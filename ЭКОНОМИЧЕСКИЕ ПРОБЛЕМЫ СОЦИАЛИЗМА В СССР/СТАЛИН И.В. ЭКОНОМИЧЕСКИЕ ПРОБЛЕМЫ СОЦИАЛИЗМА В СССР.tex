%\documentclass[oneside,final,14pt]{extreport}

\documentclass[12pt, a4paper, openany]{book}
\usepackage[left=1cm,right=1cm,top=2cm,bottom=2cm,bindingoffset=0cm]{geometry}
%\usepackage[koi8-r]{inputenc}
%\usepackage[russianb]{babel}
\usepackage{vmargin}
\setpapersize{A4}

\usepackage[T2A]{fontenc}

\usepackage[utf8x]{inputenc}

\usepackage[english, russian]{babel}

\setmarginsrb{2cm}{2cm}{2cm}{2cm}{0pt}{5mm}{0pt}{0mm}

\usepackage{indentfirst}

\usepackage{nicefrac} % For comparison

%\usepackage{xfrac} % Works better with other fonts

%\usepackage[unicode, pdftex]{hyperref}

\usepackage{lettrine}

\usepackage[usenames]{color}

\usepackage{colortbl}

\usepackage{mathtext}

\usepackage{epigraph}

\usepackage{amsmath, amsfonts, amssymb, mathrsfs}

%\usepackage{mathptmx}

%\usepackage{txfonts}

\usepackage{pxfonts}

\usepackage[pagestyles]{titlesec}

\usepackage{ebgaramond}

\usepackage{awesomebox}

\usepackage{enumitem}

\usepackage{makeidx}

\usepackage[letterspace=150]{microtype}

\makeindex

\usepackage{etoolbox}

\makeatletter

\newlength\epitextskip

\pretocmd{\@epitext}{\em}{}{}

\apptocmd{\@epitext}{\em}{}{}

\patchcmd{\epigraph}{\@epitext{#1}\\}{\@epitext{#1}\\[\epitextskip]}{}{}

\makeatother

%running fraction with slash — requires math mode.

\newcommand*\rfrac[2]{{}^{#1}\!/_{#2}}

\DeclareSymbolFont{Xlargesymbols}{OMX}{cmex}{m}{n}

\DeclareMathSymbol{\Xsum}{\mathop}{Xlargesymbols}{80}

\setlength\epigraphrule{0pt}

\setlength\epitextskip{2ex}

\setlength\epigraphwidth{.8\textwidth}

\usepackage{xfrac} % Works better with other fonts

\usepackage[colorlinks=true,linkcolor=black,urlcolor=black,bookmarksopen=true]{hyperref}

\usepackage{fancyhdr} % пакет для установки колонтитулов

\pagestyle{fancy} % смена стиля оформления страниц

\fancyhf{} % очистка текущих значений

\fancyhead[C]{\thepage} % установка верхнего колонтитула

\renewcommand{\headrulewidth}{0pt} % убрать разделительную линию

% Настройка вертикальных и горизонтальных отступов

\titlespacing{\chapter}{0pt}{5pt}{5pt}

\titlespacing{\section}{\parindent}{4mm}{4mm}

\titlespacing{\subsection}{\parindent}{3mm}{3mm}

%\renewcommand{\tabcolsep}{1cm} %% increase table column spacing

% Настройка задачи со зведочкой

\newcounter{namedlistcounter} % number the items

\newenvironment{withdot}


\newcommand{\anonsection}[1]{ \section*{#1} \addcontentsline{toc}{section}{\numberline {}#1}}

\makeatletter %%%%% <- Starting chapter without a pagebreak

\renewcommand\chapter{\par%
	
	
	
	\thispagestyle{plain}% \global\@topnum\z@
	
	
	
	\@afterindentfalse \secdef\@chapter\@schapter}

\makeatother %%%%% <- Starting chapter without a pagebreak

\titleformat{\chapter}[display]
{\normalfont\bfseries}{}{0pt}{\Large}

\newpagestyle{mystyle}{
	
	
	
	\sethead[\thepage][][]{}{}{\thepage}
	
	
	
}

\renewcommand{\rmdefault}{cmr}


\pagestyle{mystyle}

\sloppy

\begin{document}
	
	
	
	%\maketitle
	
	
	
	\begin{titlepage}
		
		
		
		
		\fontsize{36pt}{36pt}\selectfont\centering{\textbf{\textcolor{red}{\textbf{И. СТАЛИН}}}}
		
		
		
		\vspace{1.8cm}
		
		
		
		\fontsize{24pt}{24pt}\selectfont\centering{ЭКОНОМИЧКЕСКИЕ ПРОБЛЕМЫ СОЦИАЛИЗМА В СССР}
		
		
		
		
		
		
		
		
		
		
		
		\vspace{\fill}
		
		
		
		
		
		\fontsize{12pt}{14pt}\selectfont{ГОСПОЛИТИЗДАТ \ 1952}
		
		
		
		\newpage
		
		
		
		\fontsize{36pt}{36pt}\selectfont\centering{\textbf{\textbf{И. СТАЛИН}}}
		
		
		
		\vspace{1.8cm}
		
		
		
		\fontsize{24pt}{24pt}\selectfont\centering{ЭКОНОМИЧКЕСКИЕ ПРОБЛЕМЫ СОЦИАЛИЗМА В СССР}
		
		
		
		
		
		
		
		
		
		
		
		\vspace{\fill}
		
		
		
		
		
		\fontsize{12pt}{14pt}\selectfont{Г О С У Д А Р С Т В Е Н Н О Е \ И З Д А Т Е Л Ь С Т В О \\ П О Л И Т И Ч Е С К О Й \ Л И Т Е Р А Т У Р Ы \\ \textit{1952}}
		
		
		
		\thispagestyle{empty} % выключаем отображение номера для этой страницы
		
		
		
	\end{titlepage}
	
	
	
	
	
	
	
	\newpage
	
	
	
	\setcounter{secnumdepth}{0}
	
	
	
	\phantomsection
	
	
	
	\begin{center}
		
		
		
		\hangindent=10cm \hangafter=0 \textit{Участникам \\ экономической дискуссии.}
		
		
		
	\end{center}
	
	
		\thispagestyle{empty} % выключаем отображение номера для этой страницы
	
	
	\vspace{0.8cm}
	
	
	
	\section[ЗАМЕЧАНИЯ ПО ЭКОНОМИЧЕСКИМ ВОПРОСАМ, СВЯЗАННЫМ С НОЯБРЬСКОЙ ДИСКУССИЕЙ 1951 года]{\centering{ЗАМЕЧАНИЯ ПО ЭКОНОМИЧЕСКИМ ВОПРОСАМ, СВЯЗАННЫМ С НОЯБРЬСКОЙ ДИСКУССИЕЙ 1951 года}}
	
	
	
	Я получил все документы по экономической дискуссии, проведённой в связи с оценкой проекта учебника политической экономии. Получил в том числе <<Предложения по улучшению проекта учебника политической экономии>>, <<Предложения по устранению ошибок и неточностей>> в проекте, <<Справку о спорных вопросах>>.
	
	
	
	По всем этим материалам, а так же по проекту учебника считаю нужным сделать следующие замечания.
	
	
	
	\vspace{0.4cm}
	
	
	
	\subsection[1. Вопрос о характере экономических законов при социализме]{\centering{1. ВОПРОС О ХАРАКТЕРЕ ЭКОНОМИЧЕСКИХ ЗАКОНОВ ПРИ СОЦИАЛИЗМЕ.}}
	
	
	
	Некоторые товарищи отрицают объективный характер законов науки, особенно законов политической экономии при социализме. Они отрицают, что законы политической экономии отражают закономерности процессов, совершающихся независимо от воли людей. Они считают, что ввиду особой роли, предоставленной историей Советскому государству, Советское государство, его руководители могут отменить существующие законы политической экономии, могут <<сформировать>> новые законы, <<создать>> новые законы.
	
	
	
	Эти товарищи глубоко ошибаются. Они, как видно, смешивают законы науки, отражающие объективные процессы в природе или обществе, происходящие независимо от воли людей, с теми законами, которые издаются правительствами, создаются по воле людей и имеют лишь юридическую силу. Но их смешивать никак нельзя.
	
	
	
	Марксизм понимает законы науки, — всё равно идёт ли речь о законах естествознания или о законах политической экономии, — как отражение объективных процессов, происходящих независимо от воли людей. Люди могут открыть эти законы, познать их, изучить их, учитывать их в своих действиях, использовать их в интересах общества, но они не могут изменить или отменить их. Тем более они не могут сформировать или создавать новые законы науки.
	
	
	
	Значит ли это, что, например, результаты действий законов природы, результаты действий сил природы вообще неотвратимы, что разрушительные действия сил природы везде и всегда происходят со стихийно-неумолимой силой, не поддающейся воздействию людей? Нет, не значит. Если исключить астрономические, геологические и некоторые другие аналогичные процессы, где люди, если они даже познали законы их развития, действительно бессильны действовать на них, то во многих других случаях люди далеко не бессильны в смысле возможности их воздействия на процессы природы. Во всех таких случаях люди, познав законы природы, учитывая их и опираясь на них, умело применяя и используя их, могут ограничить сферу их действия, дать разрушительным силам природы другое направление, обратить разрушительные силы природы на пользу общества.
	
	
	
	Возьмём один из многочисленных примеров. В древнейшую эпоху разлив больших рек, наводнения, уничтожение в связи с этим жилищ и посевов считались неотвратимым бедствием, против которого люди были бессильны. Однако с течением времени, с развитием человеческих знаний, когда люди научились строить плотины и гидростанции, оказалось возможным отвратить от общества бедствия наводнений, казавшиеся ранее неотвратимыми. Более того, люди научились обуздывать разрушительные силы природы, так сказать оседлать их, обратить силу воды на пользу общества и использовать её для орошения полей, для получения энергии.
	
	
	
	Значит ли это, что люди тем самым отменили законы природы, законы науки, создали новые законы природы, новые законы науки? Нет, не значит. Дело в том, что вся эта процедура предотвращения действий разрушительных сил воды и использования их в интересах общества проходит без какого бы то ни было нарушения, изменения или уничтожения законов науки. Наоборот, вся эта процедура осуществляется на точном основании законов природы, законов науки, ибо какое-либо нарушение законов природы, малейшее их нарушение привело бы лишь к расстройству дела, к срыву процедуры.
	
	
	
	То же самое надо сказать о законах экономического развития, о законах политической экономии, — всё равно идёт ли речь о периоде капитализма или о периоде социализма. Здесь так же, как и в естествознании, законы экономического развития являются объективными законами, отражающими процессы экономического развития, совершающиеся независимо от воли людей. Люди могут открыть эти законы, познать их и, опираясь на них, использовать их в интересах общества, дать другое направление разрушительным действиям некоторых законов, ограничить сферу их действия, дать простор другим законам, пробивающим себе дорогу, но они не могут уничтожить их или создать новые экономические законы.
	
	
	
	Одна из особенностей политической экономии состоит в том, что её законы, в отличие от законов естествознания, недолговечны, что они, по крайней мере, большинство из них, действуют в течение определённого исторического периода, после чего они уступают место новым законам. Но они, эти законы, не уничтожаются, а теряют силу в силу новых экономических условий и сходят со сцены, чтобы уступить место новым законам, которые не создаются волею людей, а возникают на базе новых экономических условий.
	
	
	
	Ссылаются на <<Анти-Дюринг>> Энгельса, на его формулу о том, что с ликвидацией капитализма и обобществлением средств производства люди получат власть над средствами производства, что они получат свободу от гнёта общественно-экономических отношений, станут <<господами>> своей общественной жизни. Энгельс называет эту свободу <<познанной необходимостью>>. А что может означать <<познанная необходимость>>? Это означает, что люди, познав объективные законы (<<необходимость>>), будут их применять вполне сознательно в интересах общества. Именно поэтому Энгельс говорит там же, что: <<Законы их собственных общественных действий, противостоящие людям до сих пор, как чуждые, господствующие над ними законы природы, будут применяться людьми с полным знанием дела, следовательно, будут подчинены их господству>>.
	
	
	
	Как видно, формула Энгельса говорит отнюдь не в пользу тех, которые думают, что можно уничтожить при социализме существующие экономические законы и создать новые. Наоборот, она требует не уничтожения, а познания экономических законов и умелого их применения.
	
	
	
	Говорят, что экономические законы носят стихийный характер, что действия этих законов являются неотвратимыми, что общество бессильно перед ними. Это неверно. Это фетишизация законов, отдача себя в рабство законам. Доказано, что общество не бессильно перед лицом законов, что общество может, познав экономические законы и опираясь на них, ограничить сферу их действия, использовать их в интересах общества и <<оседлать>> их, как это имеет место в отношении сил природы и их законов, как это имеет место в приведённом выше примере о разливе больших рек.
	
	
	
	Ссылаются на особую роль Советской власти в деле построения социализма, которая якобы даёт ей возможность уничтожить существующие законы экономического развития и <<формировать>> новые. Это так же неверно.
	
	
	
	Особая роль Советской власти объясняется двумя обстоятельствами: во-первых, тем, что Советская власть должна была не заменить одну форму эксплуатации другой формой, как это было в старых революциях, а ликвидировать всякую эксплуатацию; во-вторых, тем, что ввиду отсутствия в стране каких-либо готовых зачатков социалистического хозяйства, она должна была создать, так сказать, на <<пустом месте>> новые, социалистические формы хозяйства.
	
	
	
	Задача эта безусловно трудная и сложная, не имеющая прецедентов. Тем не менее, Советская власть выполнила эту задачу с честью. Но она выполнила её не потому, что будто бы уничтожила существующие экономические законы и <<сформировала>> новые, а только лишь потому, что она опиралась на экономический закон \textbf{обязательного соответствия} производственных отношений характеру производительных сил. Производительные силы нашей страны, особенно в промышленности, имели общественный характер, форма же собственности была частная, капиталистическая. Опираясь на экономический закон обязательного соответствия производственных отношений характеру производительных сил, Советская власть обобществила средства производства, сделала их собственностью всего народа и тем уничтожила систему эксплуатации, создала социалистические формы хозяйства. Не будь этого закона и, не опираясь на него, Советская власть не смогла бы выполнить своей задачи.
	
	
	
	Экономический закон обязательного соответствия производственных отношений характеру производительных сил давно пробивает себе дорогу в капиталистических странах. Если он ещё не пробил себе дорогу и не вышел на простор, то это потому, что он встречает сильнейшее сопротивление со стороны отживающих сил общества.
	
	
	
	Здесь мы сталкиваемся с другой особенностью экономических законов. В отличие от законов естествознания, где открытие и применение нового закона проходит более или менее гладко, в экономической области открытие и применение нового закона, задевающие интересы отживающих сил общества, встречают сильнейшее сопротивление со стороны этих сил. Нужна, следовательно, сила, общественная сила, способная преодолеть это сопротивление. Такая сила нашлась в нашей стране в виде союза рабочего класса и крестьянства, представляющих подавляющее большинство общества. Такой силы не нашлось ещё в других, капиталистических странах. В этом секрет того, что Советской власти удалось разбить старые силы общества, а экономический закон обязательного соответствия производственных отношений характеру производительных сил получил у нас полный простор.
	
	
	
	Говорят, что необходимость планомерного (пропорционального) развития нашей страны даёт возможность Советской власти уничтожить существующие и создать новые экономические законы. Это совершенно неверно. Нельзя смешивать наши годовые и пятилетние планы с объективным экономическим законом планомерного, пропорционального развития народного хозяйства. Закон планомерного развития народного хозяйства возник как противовес закону конкуренции и анархии производства при капитализме. Он возник на базе обобществления средств производства, после того, как закон конкуренции и анархии производства потерял силу. Он вступил в действие потому, что социалистическое народное хозяйство можно вести лишь на основе экономического закона планомерного развития народного хозяйства. Это значит, что закон планомерного развития народного хозяйства даёт \textbf{возможность} нашим планирующим органам правильно планировать общественное производство. Но \textbf{возможность} нельзя смешивать с \textbf{действительностью}. Это — две разные вещи. Чтобы эту возможность превратить в действительность, нужно изучить этот экономический закон, нужно овладеть им, нужно научиться применять его с полным знанием дела, нужно составлять такие планы, которые полностью отражают требования этого закона. Нельзя сказать, что наши годовые и пятилетние планы полностью отражают требования этого экономического закона.
	
	
	
	Говорят, что некоторые экономические законы, в том числе и закон стоимости, действующие у нас при социализме, являются <<преобразованными>> или даже <<коренным образом преобразованными>> законами на основе планового хозяйства. Это тоже неверно. Нельзя <<преобразовывать>> законы, тем более <<коренным образом>>. Если можно их преобразовать, то можно и уничтожить, заменив другими законами. Тезис о <<преобразовании>> законов есть пережиток от неправильной формулы об <<уничтожении>> и <<формировании>> законов. Хотя формула о преобразовании экономических законов давно уже вошла у нас в обиход, придётся от неё отказаться в интересах точности. Можно ограничить сферу действия тех или иных экономических законов, можно предотвратить их разрушительные действия, если, конечно, они имеются, но нельзя их <<преобразовать>> или <<уничтожить>>.
	
	
	
	Следовательно, когда говорят о <<покорении>> сил природы или экономических сил, о <<господстве>> над ними и т. д., то этим вовсе не хотят сказать, что люди могут <<уничтожить>> законы науки или <<сформировать>> их. Наоборот, этим хотят сказать лишь то, что люди могут открыть законы, познать их, овладеть ими, научиться применять их с полным знанием дела, использовать их в интересах общества и таким образом покорить их, добиться господства над ними.
	
	
	
	Итак, законы политической экономии при социализме являются объективными законами, отражающими закономерность процессов экономической жизни, совершающихся независимо от нашей воли. Люди, отрицающие это положение, отрицают по сути дела науку, отрицая же науку, отрицают тем самым возможность всякого предвидения, — следовательно, отрицают возможность руководства экономической жизни.
	
	
	
	Могут сказать, что всё сказанное здесь правильно и общеизвестно, но в нём нет ничего нового и что, следовательно, не стоит тратить время на общеизвестных истин. Конечно, здесь действительно нет ничего нового, но было бы неправильно думать, что не стоит тратить время на повторение некоторых известных нам истин.
	
	
	
	Дело в том, что к нам, как руководящему ядру, каждый год подходят тысячи новых молодых кадров, они горят желанием помочь нам, горят желанием показать себя, но не имеют достаточного марксистского воспитания, не знают многих, нам хорошо известных, истин и вынуждены блуждать в потёмках. Они ошеломлены колоссальными достижениями Советской власти, им кружат голову необычайные успехи советского строя, и они начинают воображать, что Советская власть <<всё может>>, что ей <<всё нипочём>>, что она может уничтожить законы науки, сформировать новые законы. Как нам быть с этими товарищами? Как их воспитать в духе марксизма-ленинизма? Я думаю, что систематическое повторение так называемых <<общественных>> истин, терпеливое их разъяснение является одним из лучших средств марксистского воспитания этих товарищей.
	
	
	
	\vspace{0.4cm}
	
	
	
	\subsection[2. Вопрос о товарном производстве при социализме]{\centering{2. ВОПРОС О ТОВАРНОМ ПРОИЗВОДСТВЕ ПРИ СОЦИАЛИЗМЕ.}}
	
	
	
	Некоторые товарищи утверждают, что партия поступила неправильно, сохранив товарное производство после того, как она взяла власть и национализировала средства производства в нашей стране. Они считают, что партия должна была тогда уже устранить товарное производство. Они ссылаются при этом на Энгельса, который говорит:
	
	
	
	<<Раз общество возьмёт во владение средства производства, то будет устранено товарное производство, а вместе с тем и господство продуктов над производителями>> (см. <<Анти-Дюринг>>).
	
	
	
	Эти товарищи глубоко ошибаются.
	
	
	
	Разберём формулу Энгельса. Формулу Энгельса нельзя считать вполне ясной и точной, так как в ней нет указания, идёт ли речь о взятии во владение общества всех средств производства или только части средств производства, т. е. все ли средства производства переданы в общенародное достояние или только часть средств производства. Значит, эту формулу Энгельса можно понять и так и эдак.
	
	
	
	В другом месте <<Анти-Дюринга>> Энгельс говорит об овладении <<\textbf{всеми} средствами производства>>, об овладении <<\textbf{всей} совокупности средств производства>>. Значит, Энгельс в своей формуле имеет ввиду национализацию не части средств производства, а всех средств производства, т. е. передачу в общенародное достояние средств производства не только в промышленности, но и в сельском хозяйстве.
	
	
	
	Из этого следует, что Энгельс имеет ввиду такие страны, где капитализм и концентрация производства достаточно развиты не только в промышленности, но и в сельском хозяйстве для того, чтобы экспроприировать \textbf{все} средства производства страны и передать их в общенародную собственность. Энгельс считает, следовательно, что в \textbf{таких} странах следовало бы наряду с обобществлением \textbf{всех} средств производства устранить товарное производство. И это, конечно, правильно.
	
	
	
	Такой страной являлась в конце прошлого века, к моменту появления в свет <<Анти-Дюринга>>, лишь одна страна — Англия, где развитие капитализма и концентрация производства как в промышленности, так и в сельском хозяйстве были доведены до такой точки, что была возможность в случае взятия власти пролетариатом передать \textbf{все} средства производства в стране в общенародное достояние и устранить из обихода товарное производство.
	
	
	
	Я отвлекаюсь в данном случае от вопроса о значении для Англии внешней торговли с её громадным удельным весом в народном хозяйстве Англии. Я думаю, что только по изучении этого вопроса можно было бы окончательно решить вопрос о судьбе товарного производства в Англии после взятия власти пролетариатом и национализации \textbf{всех} средств производства.
	
	
	
	Впрочем, не только в конце прошлого столетия, но и в настоящее время ни одна страна ещё не достигла той степени развития капитализма и концентрации производства в сельском хозяйстве, какую наблюдаем в Англии. Что касается остальных стран, то там, несмотря на развитие капитализма в деревне, имеется ещё достаточно многочисленный класс мелких и средних собственников — производителей в деревне, судьбу которых следовало бы определить в случае взятия власти пролетариатом.
	
	
	
	Но вот вопрос: как быть пролетариату и его партии, если в той или иной стране, в том числе в нашей стране, имеются благоприятные условия для взятия власти пролетариатом и ниспровержения капитализма, где капитализм в промышленности до того концентрировал средства производства, что можно их экспроприировать и передать во владение общества, но где сельское хозяйство, несмотря на рост капитализма, до того ещё раздроблено между многочисленными мелкими и средними собственниками-производителями, что не представляется возможности ставить вопрос об экспроприации этих производителей?
	
	
	
	На этот вопрос формула Энгельса не даёт ответа. Впрочем, она и не должна отвечать на этот вопрос, так как она возникла на базе другого вопроса, а именно — вопроса о том, какова должна быть судьба товарного производства после того, как обобществлены все средства производства.
	
	
	
	Итак, как быть, если обобществлены \textbf{не все} средства производства, а только часть средств производства, а благоприятные условия для взятия власти пролетариатом имеются налицо, — следует ли взять власть пролетариату и нужно ли сразу после этого уничтожить товарное производство?
	
	
	
	Нельзя, конечно, назвать ответом мнение некоторых горе — марксистов, которые считают, что при таких условиях следовало бы отказаться от взятия власти и ждать, пока капитализм успеет разорить миллионы мелких и средних производителей, превратив их в батраков, и концентрировать средства производства в сельском хозяйстве, что только после этого можно было бы поставить вопрос о взятии власти пролетариатом и обобществлении всех средств производства. Понятно, что на такой <<выход>> не могут пойти марксисты, если они не хотят опозорить себя вконец. Нельзя так же считать ответом мнение других горе — марксистов, которые считают, что следовало бы, пожалуй, взять власть и пойти на экспроприацию мелких и средних производителей в деревне и обобществить их средства производства. На этот бессмысленный и преступный путь также не могут пойти марксисты, ибо такой путь подорвал бы всякую возможность победы пролетарской революции, отбросил бы крестьянство надолго в лагерь врагов пролетариата.
	
	
	
	Ответ на этот вопрос дал Ленин в своих трудах о <<продналоге>> и в своём знаменитом <<кооперативном плане>>.
	
	
	
	Ответ Ленина сводится коротко к следующему:
	
	
	
	а) не упускать благоприятных условий для взятия власти, взять власть пролетариату, не дожидаясь того момента, пока капитализм сумеет разорить многомиллионное население мелких и средних индивидуальных производителей;
	
	
	
	б) экспроприировать средства производства в промышленности и передать их в общенародное пользование;
	
	
	
	в) что касается мелких и средних индивидуальных производителей, объединять их постепенно в производственные кооперативы, т. е. в крупные сельскохозяйственные предприятия, колхозы.
	
	
	
	г) Развивать всемерно индустрию и подвести под колхозы современную техническую базу крупного производства, причём не экспроприировать их, а, наоборот, усиленно снабжать их первоклассными тракторами и другими машинами;
	
	
	
	д) для экономической же смычки города и деревни, промышленности и сельского хозяйства сохранить на известное время товарное производство (обмен через куплю-продажу), как \textbf{единственно приемлемую} для крестьян форму экономических связей с городом, и развернуть вовсю советскую торговлю, государственную и коллективно-колхозную, вытесняя из товарооборота всех и всяких капиталистов.
	
	
	
	История нашего социалистического строительства показывает, что этот путь развития, начертанный Лениным, полностью оправдал себя.
	
	
	
	Не может быть сомнения, что для всех капиталистических стран, имеющих более или менее многочисленный класс мелких и средних производителей, этот путь развития является единственно возможным и целесообразным для победы социализма.
	
	
	
	Говорят, что товарное производство всё же при всех условиях должно привести и обязательно приведёт к капитализму. Это неверно. Не всегда и не при всех условиях! Нельзя отождествлять товарное производство с капиталистическим производством. Это — две разные вещи. Капиталистическое производство есть высшая форма товарного производства. Товарное производство приводит к капитализму лишь в том случае, \textbf{если} существует частная собственность на средства производства, \textbf{если} рабочая сила выступает на рынок, как товар, который может купить капиталист и эксплуатировать в процессе производства, \textbf{если}, следовательно, существует в стране система эксплуатации наёмных рабочих капиталистами. Капиталистическое производство начинается там, где средства производства сосредоточены в частных руках, а рабочие, лишённые средств производства, вынуждены продавать свою рабочую силу, как товар. Без этого нет капиталистического производства.
	
	
	
	Ну, а если нет этих условий в наличии, превращающих товарное производство в капиталистическое производство, если средства производства составляют уже не частную, а социалистическую собственность, если системы наёмного труда не существует и рабочая сила не является больше товаром, если система эксплуатации давно уже ликвидирована, — как быть тогда: можно ли считать, что товарное производство всё же приведёт к капитализму? Нет, нельзя считать. А ведь наше общество является именно таким обществом, где частная собственность на средства производства, система наёмного труда, система эксплуатации давно уже не существует.
	
	
	
	Нельзя рассматривать товарное производство, как нечто самодовлеющее, независимое от окружающих экономических условий. Товарное производство старше капиталистического производства. Оно существовало при рабовладельческом строе и обслуживало его, однако не привело к капитализму. Оно существовало при феодализме и обслуживало его, однако, несмотря на то, что оно подготовило некоторые условия для капиталистического производства, не привело к капитализму. Спрашивается, почему не может товарное производство обслуживать также на известный период наше социалистическое общество, не приводя к капитализму, если иметь ввиду, что товарное производство не имеет у нас такого неограниченного и всеобъемлющего распространения, как при капиталистических условиях, что оно у нас поставлено в строгие рамки благодаря таким решающим экономическим условиям, как общественная собственность на средства производства, ликвидация системы наёмного труда, ликвидация системы эксплуатации?
	
	
	
	Говорят, что после того, как установилось в нашей стране господство общественной собственности на средства производства, а система наёмного труда и эксплуатация ликвидирована, существование товарного производства потеряло смысл, что следовало бы ввиду этого устранить товарное производство.
	
	
	
	Это также неверно. В настоящее время у нас существуют две основные формы социалистического производства: государственная — общенародная, и колхозная, которую нельзя назвать общенародной. В государственных предприятиях средства производства и продукция производства составляют всенародную собственность. В колхозных же предприятиях, хотя средства производства (земля, машины) и принадлежат государству, однако продукция производства составляет собственность отдельных колхозов, так как труд в колхозах, как и семена, — свой собственный, а землёй, которая передана колхозам в вечное пользование, колхозы распоряжаются фактически как своей собственностью, несмотря на то, что они не могут её продать, купить, сдать в аренду или заложить.
	
	
	
	Это обстоятельство ведёт к тому, что государство может распоряжаться лишь продукцией государственных предприятий, тогда как колхозной продукцией, как своей собственностью, распоряжаются лишь колхозы. Но колхозы не хотят отчуждать своих продуктов иначе как в виде товаров, в обмен на которые они хотят получить нужные им товары. Других экономических связей с городом, кроме товарных, кроме обмена через куплю-продажу, в настоящее время колхозы не приемлют. Поэтому товарное производство и товарооборот являются у нас в настоящее время такой же необходимостью, какой они были, скажем, лет тридцать тому назад, когда Ленин провозгласил необходимость всемерного разворота товарооборота.
	
	
	
	Конечно, когда вместо двух основных производственных секторов, государственного и колхозного, появится один всеобъемлющий производственный сектор с правом распоряжаться всей потребительской продукцией страны, товарное обращение с его <<денежным хозяйством>> исчезнет, как ненужный элемент народного хозяйства. Но пока этого нет, пока остаются два основных производственных сектора, товарное производство и товарное обращение должны остаться в силе, как необходимый и весьма полезный элемент в системе нашего народного хозяйства. Каким образом произойдёт создание единого объединённого сектора, путём ли простого поглощения колхозного сектора государственным сектором, что мало вероятно (ибо это было бы воспринято, как экспроприация колхозов), или путём организации единого \textbf{общенародного} хозяйственного органа (с представительством от госпромышленности и колхозов) с правом сначала учёта потребительской продукции страны, а с течением времени — также распределения продукции в порядке, скажем, продуктообмена, — это вопрос особый, требующий отдельного обсуждения.
	
	
	
	Следовательно, \textbf{наше} товарное производство представляет собой не обычное товарное производство, а товарное производство особого рода, товарное производство без капиталистов, которое имеет дело в основном с товарами объединённых социалистических производителей (государство, колхозы, кооперация), сфера действия которого ограничена предметами личного потребления, которое, очевидно, никак не может развиться в капиталистическое производство и которому суждено обслуживать совместно с его <<денежным сектором>> дело развития и укрепления социалистического производства.
	
	
	
	Поэтому совершенно не правы те товарищи, которые заявляют, что поскольку социалистическое общество не ликвидирует товарные формы производства, у нас должны быть якобы восстановлены все экономические категории, свойственные капитализму: рабочая сила, как товар, прибавочная стоимость, капитал, прибыль на капитал, средняя норма прибыли и т. п. Эти товарищи смешивают товарное производство с капиталистическим производством и полагают, что раз есть товарное производство, то должно быть и капиталистическое производство. Они не понимают, что наше товарное производство коренным образом отличается от товарного производства при капитализме.
	
	
	
	Более того, я думаю, что необходимо откинуть и некоторые другие понятия, взятые из <<Капитала>> Маркса, где Маркс занимался анализом капитализма, и искусственно приклеиваемые к нашим социалистическим отношениям. Я имею в виду, между прочим, такие понятия, как <<необходимый>> и <<прибавочный>> труд, <<необходимый>> и <<прибавочный>> продукт, <<необходимое>> и <<прибавочное>> рабочее время. Маркс анализировал капитализм для того, чтобы выяснить источник эксплуатации рабочего класса, прибавочную стоимость, и дать рабочему классу, лишённому средств производства, духовное оружие для свержения капитализма. Понятно, что Маркс пользуется при этом понятиями (категориями), вполне соответствующими капиталистическим отношениям. Но более чем странно пользоваться теперь этими понятиями, когда рабочий класс не только не лишён власти и средств производства, а наоборот, держит в своих руках власть и владеет средствами производства. Довольно абсурдно звучат теперь, при нашем строе, слова о рабочей силе, как товаре, и о <<найме>> рабочих: как будто рабочий класс, владеющий средствами производства, сам себе нанимается и сам себе продаёт свою рабочую силу. Столь же странно теперь говорить о <<необходимом>> и <<прибавочном>> труде: как будто труд в наших условиях, отданный обществу на расширение производства, развитие образования, здравоохранения, на организацию обороны и т. д., не является столь же необходимым для рабочего класса, стоящего ныне у власти, как и труд, затраченный на покрытие личных потребностей рабочего и его семьи.
	
	
	
	Следует отметить, что Маркс в своём труде <<Критика Готской программы>>, где он исследует уже не капитализм, а, между прочим, первую фазу коммунистического общества, признаёт труд, отданный обществу на расширение производства, на образование, здравоохранение, управленческие расходы, образование резервов и т. д., столь же необходимым, как и труд, затраченный на покрытие потребительских нужд рабочего класса.
	
	
	
	Я думаю, что наши экономисты должны покончить с этим несоответствием между старыми понятиями и новым положением вещей в нашей социалистической стране, заменив старые понятия новыми, соответствующими новому положению. Мы могли терпеть это несоответствие до известного времени, но теперь пришло время, когда мы должны, наконец, ликвидировать это несоответствие.
	
	
	
	\vspace{0.4cm}
	
	
	
	\subsection[3. Вопрос о законе стоимости при социализме]{\centering{3. ВОПРОС О ЗАКОНЕ СТОИМОСТИ ПРИ СОЦИАЛИЗМЕ.}}
	
	
	
	Иногда спрашивают: существует ли и действует ли у нас, при нашем социалистическом строе, закон стоимости?
	
	
	
	Да, существует и действует. Там, где есть товары и товарное производство, не может не быть и закон стоимости.
	
	
	
	Сфера действия закона стоимости распространяется у нас прежде всего на товарное обращение, на обмен товаров через куплю-продажу, на обмен главным образом товаров личного потребления. Здесь, в этой области, закон стоимости сохранят за собой, конечно, в известных пределах роль регулятора.
	
	
	
	Но действия закона стоимости не ограничиваются сферой товарного обращения. Они распространяются также на производство. Правда, закон стоимости не имеет регулирующего значения в нашем социалистическом производстве, но он всё же воздействует на производство, и этого нельзя не учитывать при руководстве производством. Дело в том, что потребительские продукты, необходимые для покрытия затрат рабочей силы в процессе производства, производятся у нас и реализуются как товары, подлежащие действию закона стоимости. Здесь именно и открывается воздействие закона стоимости на производство. В связи с этим на наших предприятиях имеют актуальное значение такие вопросы, как вопрос о хозяйственном расчёте и рентабельности, вопрос о себестоимости, вопрос о ценах и т. п. Поэтому наши предприятия не могут обойтись и не должны обходиться без учёта закона стоимости.
	
	
	
	Хорошо ли это? Не плохо. При нынешних наших условиях это действительно не плохо, так как это обстоятельство воспитывает наших хозяйственников в духе рационального ведения производства и дисциплинирует их. Не плохо, так как оно учит наших хозяйственников считать производственные величины, считать их точно и так же точно учитывать реальные вещи в производстве, а не заниматься болтовнёй об <<ориентировочных данных>>, взятых с потолка. Не плохо, так как оно учит наших хозяйственников искать, находить и использовать скрытые резервы, таящиеся в недрах производства, а не топтать их ногами. Не плохо, так как оно учит наших хозяйственников систематически улучшать методы производства, снижать себестоимость производства, осуществлять хозяйственный расчёт и добиваться рентабельности предприятий. Это — хорошая практическая школа, которая ускоряет рост наших хозяйственных кадров и превращение их в настоящих руководителей социалистического производства на нынешнем этапе развития.
	
	
	
	Беда не в том, что закон стоимости воздействует у нас на производство. Беда в том, что наши хозяйственники и плановики, за немногими исключениями, плохо знакомы с действиями закона стоимости, не изучают их и не умеют учитывать их в своих расчётах. Этим собственно и объясняется та неразбериха, которая всё ещё царит у нас в вопросе о политике цен. Вот один из многочисленных примеров. Некоторое время тому назад было решено упорядочить в интересах хлопководства соотношение цен на хлопок и на зерно, уточнить цены на зерно, продаваемое хлопкоробам, и поднять цены на хлопок, сдаваемый государству. В связи с этим наши хозяйственники и плановики внесли предложение, которое не могло не изумить членов ЦК, так как по этому предложению цена на тонну зерна предлагалось почти такая же, как цена на тонну хлопка, при этом цена на тонну зерна была приравнена к цене на тонну печёного хлеба. На замечания членов ЦК о том, что цена на тонну печёного хлеба должна быть выше цены на тонну зерна ввиду добавочных расходов на помол и выпечку, что хлопок вообще стоит намного дороже, чем зерно, о чём свидетельствуют также мировые цены на хлопок и на зерно, авторы предложения не могли сказать ничего вразумительного. Ввиду этого ЦК пришлось взять в свои руки, снизить цены на зерно и поднять цены на хлопок. Что было бы, если бы предложение этих товарищей получило законную силу? Мы разорили бы хлопкоробов и остались бы без хлопка.
	
	
	
	Значит ли, однако, всё это, что действия закона стоимости имеет у нас такой же простор, как при капитализме, что закон стоимости является у нас регулятором производства? Нет, не значит. На самом деле сфера действия закона стоимости при нашем экономическом строе строго ограничено и поставлено в рамки. Уже было сказано, что сфера действия товарного производства при нашем строе ограничено и поставлено в рамки. То же самое надо сказать о сфере действия закона стоимости. Несомненно, что отсутствие частной собственности на средства производства и обобществлении средств производства как в городе, так и в деревне, не могут не ограничивать сферу действия закона стоимости и степень его воздействия на производство.
	
	
	
	В том же направлении действует закон планомерного (пропорционального) развития народного хозяйства, заменивший собой закон конкуренции и анархии производства.
	
	
	
	В том же направлении действуют наши годовые и пятилетние планы и вообще вся наша хозяйственная политика, опирающаяся на требования закона планомерного закона развития народного хозяйства.
	
	
	
	Всё это вместе ведёт к тому, что сфера действия закона стоимости строго ограничена у нас и закон стоимости не может при нашем строе играть роль регулятора производства.
	
	
	
	Этим, собственно, и объясняется тот <<поразительный>> факт, что, несмотря на непрерывный и бурный рост нашего социалистического производства, закон стоимости не ведёт нас к кризисам перепроизводства, тогда как тот же закон стоимости, имеющий широкую сферу действия при капитализме, несмотря на низкие темпы роста производства в капиталистических странах, — ведёт к периодическим кризисам перепроизводства.
	
	
	
	Говорят, что закон стоимости является постоянным законом, обязательным для всех периодов исторического развития, что если закон стоимости и потеряет силу, как регулятор меновых отношений в период второй фазы коммунистического общества, то он сохранит на этой фазе развития свою силу, как регулятор отношений между различными отраслями производства, как регулятор распределения труда между отраслями производства.
	
	
	
	Это совершенно неверно. Стоимость, как закон стоимости, есть историческая категория, связанная с существованием товарного производства. С исчезновением товарного производства исчезнут и стоимость с её формами и закон стоимости.
	
	
	
	На второй фазе коммунистического общества количество труда, затраченного на производство продуктов, будет измеряться не окольным путём, не через посредство стоимости её форм, как это бывает при товарном производстве, а прямо и непосредственно — количеством времени, количеством часов, израсходованным на производство продуктов. Что же касается распределения труда, то распределение труда между отраслями производства будет регулироваться не законом стоимости, который потеряет силу к этому времени, а ростом потребностей общества в продуктах. Это будет общество, где производство будет регулироваться потребностями общества, а учёт потребностей общества приобретёт первостепенное значение для планирующих органов.
	
	
	
	Совершенно неправильно также утверждение, что при нашем нынешнем экономическом строе, на первой фазе развития коммунистического общества, закон стоимости регулирует будто бы <<пропорции>> распределения труда между различными отраслями производства.
	
	
	
	Если бы это было верно, то непонятно, почему у нас не развивают во — всю лёгкую промышленность, как наиболее рентабельную, преимущественно перед тяжёлой промышленностью, являющейся часто менее рентабельной, а иногда и вовсе нерентабельной?
	
	
	
	Если бы это было верно, то непонятно, почему не закрывают у нас ряд пока ещё нерентабельных предприятий тяжёлой промышленности, где труд рабочих не даёт <<должного эффекта>>, и не открывают новых предприятий безусловно рентабельной лёгкой промышленности, где труд рабочих мог бы дать <<больший эффект>>?
	
	
	
	Если бы это было верно, то непонятно, почему не перебрасывают у нас рабочих из малорентабельных предприятий, хотя и очень нужных для народного хозяйства, в предприятия более рентабельные, согласно закона стоимости, якобы регулирующего <<пропорции>> распределения труда между отраслями производства?
	
	
	
	Очевидно, что идя по стопам этих товарищей, нам пришлось бы отказаться от примата производства средств производства в пользу производства средств потребления. А что значит отказаться от примата средств производства? Это значит уничтожить возможность непрерывного роста нашего народного хозяйства, ибо невозможно осуществлять непрерывный рост народного хозяйства, не осуществляя вместе с тем примата производства средств производства.
	
	
	
	Эти товарищи забывают, что закон стоимости может быть регулятором производства лишь при капитализме, при наличии частной собственности на средства производства, при наличии конкуренции, анархии производства, кризисов перепроизводства. Они забывают, что сфера действия закона стоимости ограничена у нас наличием общественной собственности на средства производства, действием закона планомерного развития народного хозяйства, — следовательно, ограничена также нашими годовыми и пятилетними планами, являющимися приблизительным отражением требований этого закона.
	
	
	
	Некоторые товарищи делают отсюда вывод, что закон планомерного развития народного хозяйства и планирование народного хозяйства уничтожают принцип рентабельности производства. Это совершенно неверно. Дело обстоит как раз наоборот. Если взять рентабельность не с точки зрения отдельных предприятий или отраслей производства и не в разрезе одного года, а с точки зрения всего народного хозяйства и в разрезе, скажем, 10–15 лет, что было бы единственно правильным подходом к вопросу, временная и непрочная рентабельность отдельных предприятий или отраслей производства не может идти ни в какое сравнение с той высшей формой прочной и постоянной рентабельности, которую дают нам действия закона планомерного развития народного хозяйства и планирование народного хозяйства, избавляя нас от периодических экономических кризисов, разрушающих народное хозяйство и наносящих обществу колоссальный материальный ущерб, и обеспечивая нам непрерывный рост народного хозяйства с его высокими темпами.
	
	
	
	Короче: не может быть сомнения, что при наших нынешних социалистических условиях производства закон стоимости не может быть <<регулятором пропорций>> в деле распределения труда между различными отраслями производства.
	
	
	
	\vspace{0.4cm}
	
	
	
	\subsection[4. Вопрос об уничтожении противоположности между городом и деревней, между умственным и физическим трудом, а также вопрос о ликвидации различий между ними]{\centering{4. ВОПРОС ОБ УНИЧТОЖЕНИИ ПРОТИВОПОЛОЖНОСТИ МЕЖДУ ГОРОДОМ И ДЕРЕВНЕЙ, МЕЖДУ УМСТВЕННЫМ И ФИЗИЧЕСКИМ ТРУДОМ, А ТАКЖЕ ВОПРОС О ЛИКВИДАЦИИ РАЗЛИЧИЙ МЕЖДУ НИМИ.}}
	
	
	
	Заголовок этот затрагивает ряд проблем, существенно отличающихся друг от друга, однако я объединяю их в одной главе не для того, чтобы смешать их друг с другом, а исключительно для краткости изложения.
	
	
	
	Проблема уничтожения противоположности между городом и деревней, между промышленностью и сельским хозяйством представляет известную проблему, давно уже поставленную Марксом и Энгельсом. Экономической основой этой противоположности является эксплуатация деревни городом, экспроприация крестьянства и разорение большинства деревенского населения всем ходом развития промышленности, торговли, кредитной системы при капитализме. Поэтому противоположность между городом и деревней при капитализме нужно рассматривать как противоположность интересов. На этой почве возникло враждебное отношение деревни к городу и вообще к <<городским людям>>.
	
	
	
	Несомненно, что с уничтожением капитализма и системы эксплуатации, с укреплением социалистического строя в нашей стране должна была исчезнуть и противоположность интересов между городом и деревней, между промышленностью и сельским хозяйством. Оно так и произошло. Огромная помощь нашему крестьянству со стороны социалистического города, со стороны нашего рабочего класса, оказанная в деле ликвидации помещиков и кулачества, укрепила почву для союза рабочего класса и крестьянства, а систематическое снабжение крестьянства и его колхозов первоклассными тракторами и другими машинами превратило союз рабочего класса и крестьянства в дружбу между ними. Конечно, рабочее и колхозное крестьянство составляет все те же два класса, отличающиеся друг от друга по своему положению. Но это различие ни в какой мере не ослабляет их дружбу. Наоборот, их интересы лежат на одной общей линии, на одной линии укрепления социалистического строя и победы коммунизма. Не удивительно поэтому, что от былого недоверия, а тем более ненависти деревни к городу не осталось и следа.
	
	
	
	Всё это означает, что почва для противоположности между городом и деревней, между промышленностью и сельским хозяйством уже ликвидирована нынешним нашим социалистическим строем.
	
	
	
	Это, конечно, не значит, что уничтожение противоположности между городом и деревней должно повести к <<гибели больших городов>> (см. <<Анти-Дюринг>> Энгельса). Большие города не только не погибнут, но появятся ещё новые большие города, как центры наибольшего роста культуры, как центры не только большой индустрии, но и переработки сельскохозяйственных продуктов и мощного развития всех отраслей пищевой промышленности. Это обстоятельство облегчит культурный расцвет страны и приведёт к выравниванию условий быта в городе и в деревне.
	
	
	
	Аналогичное положение имеем мы с проблемой уничтожения противоположности между умственным и физическим трудом. Эта проблема так же является известной проблемой, давно поставленной Марксом и Энгельсом. Экономической основой противоположности между умственным и физическим трудом является эксплуатация людей физического труда со стороны представителей умственного труда. Всем известен разрыв, существовавший при капитализме между людей физического труда предприятий и руководящим персоналом. Известно, что на базе этого развивалось враждебное отношение рабочих к директору, к мастеру, к инженеру и другим представителям технического персонала, как их врагам. Понятно, что с уничтожением капитализма и системы эксплуатации должна была исчезнуть и противоположность интересов между физическим и умственным трудом. И она действительно исчезла при нашем современном социалистическом строе. Теперь люди физического труда и руководящий персонал являются не врагами, а товарищами-друзьями, членами единого производственного коллектива, кровно заинтересованными в преуспевании и улучшении производства. От былой вражды между ними не осталось и следа.
	
	
	
	Совершенно другой характер имеет проблема исчезновения различий между городом (промышленностью) и деревней (сельским хозяйством), между физическим и умственным трудом. Эта проблема не ставилась классиками марксизма. Это — новая проблема, поставленная практикой нашего социалистического строительства.
	
	
	
	Не является ли эта проблема надуманной, имеет ли она для нас какое-либо практическое или теоретическое значение? Нет, эту проблему нельзя считать надуманной. Наоборот, она является для нас в высшей степени серьёзной проблемой.
	
	
	
	Если взять, например, различие между сельским хозяйством и промышленностью, то оно сводится у нас не только к тому, что условия труда в сельском хозяйстве отличаются от условий труда в промышленности, но, прежде всего и главным образом к тому, что в промышленности мы имеем общенародную собственность на средства производства и продукцию производства, тогда как в сельском хозяйстве имеем не общенародную, а групповую, колхозную собственность. Уже говорилось, что это обстоятельство ведёт к сохранению товарного обращения, что только с исчезновением этого различия между промышленностью и сельским хозяйством может исчезнуть товарное производство с вытекающими отсюда последствиями. Следовательно, нельзя отрицать, что исчезновение этого существенного различия между сельским хозяйством и промышленностью должно иметь для нас первостепенное значение.
	
	
	
	То же самое нужно сказать о проблеме уничтожения существенного различия между трудом умственным и трудом физическим. Эта проблема имеет для нас также первостепенное значение. До начала разворота массового соцсоревнования рост промышленности шёл у нас со скрипом, а многие товарищи ставили даже вопрос о замедлении темпов развития промышленности. Объясняется это главным образом тем, что культурно-технический уровень рабочих был слишком низок и далеко отставал от уровня технического персонала. Дело, однако, изменилось коренным образом после того, как соцсоревнование приняло у нас массовый характер. Именно после этого промышленность пошла вперёд ускоренным темпом. Почему соцсоревнование приняло массовый характер? Потому, что среди рабочих нашлись целые группы товарищей, которые не только освоили технический минимум, но пошли дальше, стали в уровень с техническим персоналом, стали поправлять техников инженеров, ломать существующие нормы, как устаревшие, вводить новые, более современные нормы и т. п. Что было бы, если бы не отдельные группы рабочих, а большинство рабочих подняло свой культурно-технический уровень до уровня инженерно-технического персонала? Наша промышленность была бы поднята на высоту, недосягаемую для промышленности других стран. Следовательно, нельзя отрицать, что уничтожение существенного различия между умственным и физическим трудом путём поднятия культурно-технического персонала до уровня технического персонала не может не иметь для нас первостепенного значения.
	
	
	
	Некоторые товарищи утверждают, что со временем исчезнет не только существенное различие между промышленностью и сельским хозяйством, между физическим и умственным трудом, но исчезнет также \textbf{всякое} различие между ними. Это неверно. Уничтожение существенного различия между промышленностью и сельским хозяйством не может привести к уничтожению всякого различия между ними. Какое-то различие, хотя и несущественное, безусловно, останется ввиду различий в условиях работы в промышленности и в сельском хозяйстве. Даже в промышленности, если иметь в виду различные её отрасли, условия работы не везде одинаковы: условия работы, например, шахтёров отличаются от условий работы рабочих механизированной обувной фабрики, условия работы рудокопов отличаются от условий работы машиностроительных рабочих. Если это верно. То тем более должно сохраниться известное различие между промышленностью и сельским хозяйством.
	
	
	
	То же самое надо сказать насчёт различия между трудом умственным и трудом физическим. Существенное различие между ними в смысле разрыва в культурно-техническом уровне безусловно исчезнет. Но какое-то различие, хотя и несущественное, всё же сохранится, хотя бы потому, что условия работы руководящего состава предприятий не одинаковы с условиями работы рабочих.
	
	
	
	Товарищи, утверждающие обратное, опираются, должно быть, на известную формулировку в некоторых моих выступлениях, где говорится об уничтожении различия между промышленностью и сельским хозяйством, между умственным и физическим трудом, без оговорки о том, что речь идёт об уничтожении \textbf{существенного}, а не всякого различия. Но это значит, что формулировка была не точная, неудовлетворительная. Её нужно откинуть и заменить другой формулировкой, говорящей об уничтожении существенных различий и сохранении несущественных различий между умственным и физическим трудом.
	
	
	
	\vspace{0.4cm}
	
	
	
	\subsection[5. Вопрос о распаде единого мирового рынка и углублении кризиса мировой капиталистической системы]{\centering{5. ВОПРОС О РАСПАДЕ ЕДИНОГО МИРОВОГО РЫНКА И УГЛУБЛЕНИИ КРИЗИСА МИРОВОЙ КАПИТАЛИСТИЧЕСКОЙ СИСТЕМЫ.}}
	
	
	
	Наиболее важным экономическим результатом второй мировой войны и её хозяйственных последствий нужно считать распад единого всеохватывающего мирового рынка. Это обстоятельство определило дальнейшее углубление общего кризиса мировой капиталистической системы.
	
	
	
	Вторая мировая война сама была порождена этим кризисом. Каждая из двух капиталистических коалиций, вцепившихся друг в друга во время войны, рассчитывала разбить противника и добиться мирового господства. В этом они искали выход из кризиса. Соединённые Штаты Америки рассчитывали вывести из строя наиболее опасных своих конкурентов, Германию и Японию, захватить зарубежные рынки, мировые ресурсы сырья и добиться мирового господства.
	
	
	
	Однако война не оправдала этих надежд. Правда, Германия и Япония были выведены из строя, как конкуренты трёх главных капиталистических стран: США, Англии, Франции. Но наряду с этим от капиталистической системы отпали Китай и другие народно-демократические страны в Европе, образовав вместе с Советским Союзом единый и мощный социалистический лагерь, противостоящий лагерю капитализма. Экономическим результатом существования двух противоположных лагерей явилось то, что единый всеохватывающий мировой рынок распался, в результате чего мы имеем теперь два параллельных мировых рынка, тоже противостоящих друг другу.
	
	
	
	Следует отметить, что США и Англия с Францией сами содействовали, конечно, помимо своей воли, образованию и укреплению нового параллельного мирового рынка. Они подвергли экономической блокаде СССР, Китай и европейские народно-демократические страны, не вошедшие в систему <<плана Маршалла>>, думая этим ухудшить их. На деле же получилось не ухудшение, а укрепление нового мирового рынка. Всё же основное в этом деле состоит, конечно, не в экономической блокаде, а в том, что за период после войны эти страны экономически сомкнулись и наладили экономическое сотрудничество и взаимопомощь. Опыт этого сотрудничества показывает, что ни одна капиталистическая страна не могла бы оказать такой действительной и технически квалифицированной помощи народно-демократическим странам, какую оказывает им Советский Союз. Дело не только в том, что помощь эта является максимально дешёвой и технически первоклассной. Дело прежде всего в том, что в основе этого сотрудничества лежит искреннее желание помочь друг другу и добиться общего экономического подъёма. В результате мы имеем высокие темпы развития промышленности в этих странах. Можно с уверенностью сказать, что при таких темпах развития промышленности скоро дело дойдёт до того, что эти страны не только не будут нуждаться в завозе товаров из капиталистических стран, но сами почувствуют необходимость отпускать на сторону избыточные товары своего производства.
	
	
	
	Но из этого следует, что сфера приложения сил главных капиталистических стран (США, Англия, Франция) к мировым ресурсам будет не расширяться, а сокращаться, что условия мирового рынка сбыта для этих стран будут ухудшаться, а недогрузка предприятий в этих странах будет увеличиваться. В этом, собственно, и состоит углубление общего кризиса мировой капиталистической системы в связи с распадом мирового рынка.
	
	
	
	Это чувствуют сами капиталисты, ибо трудно не почувствовать потерю таких рынков, как СССР, Китай. Они стараются перекрыть эти трудности <<планом Маршалла>>, войной в Корее, гонкой вооружения, милитаризацией промышленности. Но это очень похоже на то, что утопающие хватаются за соломинку.
	
	
	
	В связи с таким положением перед экономистами встали два вопроса:
	
	
	
	а) Можно ли утверждать, что известный тезис Сталина об относительной стабильности рынков в период общего кризиса капитализма, высказанный до второй мировой войны, — всё ещё остаётся в силе?
	
	
	
	б) Можно ли утверждать, что известный тезис Ленина, высказанный им весной 1916 года, о том, что, несмотря на загнивание капитализма, <<в целом капитализм растёт неизмеримо быстрее, чем прежде>>, — всё ещё остаётся в силе?
	
	
	
	Я думаю, что нельзя этого утверждать. Виду новых условий, возникших в связи со второй мировой войной, оба тезиса нужно считать утратившими силу.
	
	
	
	\vspace{0.4cm}
	
	
	
	\subsection[6. Вопрос о неизбежности войн между капиталистическими странами]{\centering{6. ВОПРОС О НЕИЗБЕЖНОСТИ ВОЙН МЕЖДУ КАПИТАЛИСТИЧЕСКИМИ СТРАНАМИ.}}
	
	
	
	Некоторые товарищи утверждают, что в силу развития новых международных условий после второй мировой войны, войны между капиталистическими странами перестали быть неизбежными. Они считают, что противоречия между лагерем социализма и лагерем капитализма сильнее, чем противоречия между капиталистическими странами, что Соединённые Штаты Америки достаточно подчинили себе другие капиталистические страны для того, чтобы не дать им воевать между собой и ослаблять друг друга, что передовые люди капитализма достаточно научены опытом двух мировых войн, нанёсших серьёзный ущерб всему капиталистическому миру, чтобы позволить себе вновь втянуть капиталистические страны в войну между собой, — что ввиду всего этого войны между капиталистическими странами перестали быть неизбежными.
	
	
	
	Эти товарищи ошибаются. Они видят внешние явления, мелькающие на поверхности, но не видят тех глубинных сил, которые, хотя и действуют пока незаметно, но всё же будут определять ход событий.
	
	
	
	Внешне всё будто бы обстоит <<благополучно>>: Соединённые Штаты Америки посадили на паёк Западную Европу, Японию и другие капиталистические страны; Германия (Западная), Англия, Франция, Италия, Япония, попавшие в лапы США, послушно выполняют веления США. Но было бы нелепо думать, что это <<благополучие>> может сохраниться <<на веки вечные>>, что эти страны будут без конца терпеть господство и гнёт Соединённых Штатов Америки, что они не попытаются вырваться из американской неволи и стать на путь самостоятельного развития.
	
	
	
	Возьмём прежде всего Англию и Францию. Несомненно, что эти страны являются империалистическими. Несомненно, что дешёвое сырьё и обеспеченные рынки сбыта имеют для них первостепенное значение. Можно ли полагать, что они будут без конца терпеть нынешнее положение, когда американцы под шумок <<помощи>> по линии <<плана Маршалла>> внедряются в экономику Англии и Франции, стараясь превратить её в придаток экономики Соединённых Штатов Америки, когда американский капитал захватывает сырьё и рынки сбыта в англо-французских колониях и готовят, таким образом, катастрофу для высоких прибылей англо-французских капиталистов? Не вернее ли будет сказать, что капиталистическая Англия, а вслед за ней и капиталистическая Франция в конце концов будут вынуждены вырваться из объятий США и пойти на конфликт с ними для того, чтобы обеспечить себе самостоятельное положение и, конечно, высокие прибыли?
	
	
	
	Перейдём к главным побеждённым странам, к Германии (Западной), Японии. Эти страны влачат теперь жалкое существование под сапогом американского империализма. Их промышленность и сельское хозяйство, их торговля, их внешняя и внутренняя политика, весь их быт скованы американским <<режимом>> оккупации. А ведь эти страны вчера ещё были великими империалистическими державами, потрясавшими основы господства Англии, США, Франции в Европе, в Азии. Думать, что эти страны не попытаются вновь подняться на ноги, сломить <<режим>> США и вырваться на путь самостоятельного развития — значит верить в чудеса.
	
	
	
	Говорят, что противоречия между капитализмом и социализмом сильнее, чем противоречия между капиталистическими странами. Теоретически это, конечно, верно. Это верно не только теперь, в настоящее время, — это было верно также перед второй мировой войной. И это более или менее понимали руководители капиталистических стран. И всё же вторая мировая война началась не с войны с СССР, а с войны между капиталистическими странами. Почему? Потому, во-первых, что война с СССР, как с страной социализма, опаснее для капитализма, чем война между капиталистическими странами, ибо, если война между капиталистическими странами ставит вопрос только о преобладании таких-то капиталистических стран над другими капиталистическими странами, то война с СССР обязательно должна поставить вопрос о существовании самого капитализма. Потому, во-вторых, что капиталисты, хотя и шумят в целях <<пропаганды>> об агрессивности Советского Союза, сами не верят в его агрессивность, так как они учитывают мирную политику Советского Союза и знают, что Советский Союз сам не нападёт на капиталистические страны.
	
	
	
	После первой мировой войны тоже считали, что Германия окончательно выведена из строя, так же как некоторые товарищи думают теперь, что Япония и Германия выведены из строя. То же говорили и шумели в прессе о том, что Соединённые Штаты Америки посадили Европу на паёк, что Германия не может больше встать на ноги, что отныне войны между капиталистическими странами не должно быть. Однако, несмотря на это, Германия встала на ноги как великая держава через каких-либо 15–20 лет после своего поражения, вырвавшись из неволи и став на путь самостоятельного развития. При этом характерно, что не кто иной, как Англия и Соединённые Штаты Америки помогли Германии подняться экономически и поднять её военно-экономический потенциал. Конечно, США и Англия, помогая Германии подняться экономически, имели при этом в виду направить поднявшуюся Германию против Советского Союза, использовать её против страны социализма. Однако Германия направила свои силы в первую очередь против англо-франко-американского блока. И когда гитлеровская Германия объявила войну Советскому Союзу, то англо-франко-американский блок не только не присоединился к гитлеровской Германии, а, наоборот, был вынужден вступить в коалицию с СССР против гитлеровской Германии.
	
	
	
	Следовательно, борьба капиталистических стран за рынки и желание утопить своих конкурентов оказались практически сильнее, чем противоречия между лагерем капитализма и лагерем социализма.
	
	
	
	Спрашивается, какая имеется гарантия, что Германия и Япония не поднимутся вновь на ноги, что они не попытаются вырваться из американской неволи и зажить своей самостоятельной жизнью? Я думаю, что таких гарантий нет. Но из этого следует, что неизбежность войн между капиталистическими странами остаётся в силе.
	
	
	
	Говорят, что тезис Ленина о том, что империализм неизбежно порождает войны, нужно считать устаревшим, поскольку выросли в настоящее время мощные народные силы, выступающие в защиту мира, против новой мировой войны. Это неверно.
	
	
	
	Современное движение за мир имеет своей целью поднять народные массы на борьбу за сохранение мира, за предотвращение новой мировой войны. Следовательно, оно не преследует цели свержения капитализма, — оно ограничивается демократическими целями борьбы за сохранение мира. В этом отношении современное движение за сохранение за сохранение мира отличается от движения в период первой мировой войны за превращение войны империалистической в гражданскую войну, так как это последнее движение шло дальше и преследовало социалистические цели.
	
	
	
	Возможно, что при известном стечении обстоятельств, борьба за мир разовьётся кое-где в борьбу за социализм, но это будет уже не современное движение за мир, а движение за свержение капитализма.
	
	
	
	Вероятнее всего, что современное движение за мир, как движение за сохранение мира, в случае успеха приведёт к предотвращению \textbf{данной} войны, к временной её отсрочке, к временному сохранению \textbf{данного} мира, к отставке воинствующего правительства и замене его другим правительством, готовым временно сохранить мир. Это, конечно, хорошо. Даже очень хорошо. Но этого всё же недостаточно для того, чтобы уничтожить неизбежность войн вообще между капиталистическими странами. Недостаточно, так как при всех этих успехах движения в защиту мира империализм всё же сохраняется, остаётся в силе, — следовательно, остаётся в силе так же неизбежность войн.
	
	
	
	Чтобы устранить неизбежность войн, нужно уничтожить империализм.
	
	
	
	\vspace{0.4cm}
	
	
	
	\subsection[7. Вопрос об основных экономических законах современного капитализма и социализма]{\centering{7. ВОПРОС ОБ ОСНОВНЫХ ЭКОНОМИЧЕСКИХ ЗАКОНАХ СОВРЕМЕННОГО КАПИТАЛИЗМА И СОЦИАЛИЗМА.}}
	
	
	
	
	
	Как известно, вопрос об основных экономических законах капитализма и социализма несколько раз выдвигался на дискуссии. Высказывались различные мнения на этот счёт вплоть до самых фантастических. Правда, большинство участников дискуссии слабо реагировали на это дело, и никакого решения на этот счёт не было намечено. Однако никто из участников дискуссии не отрицал существования таких законов.
	
	
	
	Существует ли основной экономический закон капитализма? Да, существует. Что это за закон, в чём состоят его характерные черты? Основной экономический закон капитализма — это такой закон, который определяет не какую-либо отдельную сторону или какие-либо отдельные процессы развития капиталистического производства, а все главные стороны и все главные процессы этого развития, — следовательно, определяет существо капиталистического производства, его сущность.
	
	
	
	Не является ли закон стоимости основным экономическим законом капитализма? Нет. Закон стоимости есть прежде всего закон товарного производства. Он существовал до капитализма и продолжает существовать, как и товарное производство, после свержения капитализма, например, в нашей стране, правда, с ограниченной сферой действия. Конечно, закон стоимости, имеющий широкую сферу действия в условиях капитализма, играет большую роль в деле развития капиталистического производства, но он не только не определяет существа капиталистического производства и основ капиталистической прибыли, но даже не ставит таких проблем. Поэтому он не может быть основным экономическим законом современного капитализма.
	
	
	
	По тем же соображениям не может быть основным экономическим законом капитализма закон конкуренции и анархии производства, ил закон неравномерного развития капитализма в различных странах.
	
	
	
	Говорят, что закон средней нормы прибыли является основным экономическим законом современного капитализма. Это неверно. Современный капитализм, монополистический капитализм, не может удовлетворяться средней прибылью, которая к тому же имеет тенденцию к снижению ввиду повышения органического состава капитала. Современный монополистический капитализм требует не средней прибыли, а максимума прибыли, необходимого для того, чтобы осуществлять более или менее регулярно расширенное воспроизводство.
	
	
	
	Более всего подходит к понятию основного экономического закона капитализма закон прибавочной стоимости, закон рождения и возрастания капиталистической прибыли. Он действительно предопределяет основные черты капиталистического производства. Но закон прибавочной стоимости является слишком общим законом, не затрагивающим проблемы высшей нормы прибыли, обеспечение которой является условием развития монополистического капитализма. Чтобы восполнить этот пробел, нужно конкретизировать закон прибавочной стоимости и развить его дальше применительно к условиям монополистического капитализма, учтя при этом, что монополистический капитализм требует не всякой прибыли, а именно максимальной прибыли. Это и будет основной экономический закон современного капитализма.
	
	
	
	Главные черты и требования основного экономического закона современного капитализма можно было бы сформулировать примерно таким образом: обеспечение максимальной капиталистической прибыли путём эксплуатации, разорения и обнищания большинства населения данной страны, путём закабаления и систематического ограбления народов других стран, особенно отсталых стран, наконец, путём войн и милитаризации народного хозяйства, используемых для обеспечения наивысших прибылей.
	
	
	
	Говорят, что среднюю прибыль всё же можно бы считать вполне достаточной для капиталистического развития в современных условиях. Это неверно. Средняя прибыль есть низший предел рентабельности, ниже которого капиталистическое производство становится невозможным. Но было бы смешно думать, что воротилы современного монополистического капитализма, захватывая колонии, порабощая народы и затевая войны, стараются обеспечить себе всего лишь среднюю прибыль. Нет, не средняя прибыль, и не сверхприбыль, представляющая, как правило, всего лишь некоторое превышение над средней прибылью, а именно максимальная прибыль является двигателем монополистического капитализма. Именно необходимость получения максимальных прибылей толкает монополистический капитализм на такие рискованные шаги, как закабаление и систематическое ограбление колоний и других отсталых стран в зависимые страны, организация новых войн, являющихся для воротил современного капитализма лучшим <<бизнесом>> для извлечения максимальных прибылей, наконец, попытки завоевания мирового экономического господства.
	
	
	
	Значение основного экономического закона капитализма состоит между прочим в том, что он, определяя все важнейшие явления в области развития капиталистического способа производства, его подъёмы и кризисы, его победы и поражения, его достоинства и недостатки, — весь этот процесс его противоречивого развития, — даёт возможность понять и объяснить их.
	
	
	
	Вот один из многочисленных <<поразительных>> примеров.
	
	
	
	Всем известны факты из истории и практики капитализма, демонстрирующие бурное развитие техники при капитализме, когда капиталисты выступают как знаменосцы передовой техники, как революционеры в области развития техники производства. Но известны так же факты другого рода, демонстрирующие приостановку развития техники при капитализме, когда капиталисты выступают как реакционеры в области развития новой техники и переходят нередко на ручной труд.
	
	
	
	Чем объяснить это вопиющее противоречие? Его можно объяснить лишь основным экономическим законом современного капитализма, то есть необходимостью получения максимальных прибылей. Капитализм стоит за новую технику, когда она сулит ему наибольшие прибыли. Капитализм стоит против новой техники и за переход на ручной труд, когда новая техника не сулит больше наибольших прибылей.
	
	
	
	Так обстоит дело с основным экономическим законом современного капитализма. Существует ли основной экономический закон социализма? Да, существует. В чём состоят существенные черты и требования этого закона? Существенные черты и требования основного экономического закона социализма можно было бы сформулировать примерно таким образом: обеспечение максимального удовлетворения постоянно растущих материальных и культурных потребностей всего общества путём непрерывного роста и совершенствования социалистического производства на базе высшей техники.
	
	
	
	Следовательно: вместо обеспечения максимальных прибылей, — обеспечение максимального удовлетворения материальных и культурных потребностей общества; вместо развития производства с перерывами от подъёма к кризису и от кризиса к подъёму, — непрерывный рост производства; вместо периодических перерывов в развитии техники, сопровождающихся разрушением производительных сил общества, — непрерывное совершенствование производства на базе высшей техники.
	
	
	
	Говорят, что основным экономическим законом социализма является закон планомерного, пропорционального развития народного хозяйства. Это неверно. Планомерное развитие народного хозяйства, а значит и планирование народного хозяйства, являющееся более или менее верным отражением этого закона, сами по себе ничего не могут дать, если неизвестно, во имя какой задачи совершается плановое развитие народного хозяйства, или если задача не ясна. Закон планомерного развития народного хозяйства может дать должный эффект лишь в том случае, если имеется задача, во имя осуществления которой совершается плановое развитие народного хозяйства. Эту задачу не может дать сам закон планомерного развития народного хозяйства. Её тем более не может дать планирование народного хозяйства. Эта задача содержится в основном экономическом законе социализма в виде его требований, изложенных выше. Поэтому действия закона планомерного развития народного хозяйства могут получить полный простор лишь в том случае, если они опираются на основной экономический закон социализма.
	
	
	
	Что касается планирования народного хозяйства, то оно может добиться положительных результатов лишь при соблюдении двух условий: а) если оно правильно отражает требования закона планомерного развития народного хозяйства, б) если оно сообразуется во всём с требованиями основного экономического закона социализма.
	
	
	
	\vspace{0.4cm}
	
	
	
	\subsection[8. Другие вопросы]{\centering{8. ДРУГИЕ ВОПРОСЫ.}}
	
	
	
	1) Вопрос о внеэкономическом принуждении при феодализме.
	
	
	
	Конечно, внеэкономическое принуждение играло роль в деле укрепления экономической власти помещиков-крепостников, однако, не оно являлось основой феодализма, а феодальная собственность на землю.
	
	
	
	2) Вопрос о личной собственности колхозного двора.
	
	
	
	Неправильно было бы сказать в проекте учебника, что <<каждый колхозный двор имеет в личном пользовании корову, мелкий скот и птицу>>. На самом деле, как известно, корова, мелкий скот, птица и т. д. находятся не в личном пользовании, а в личной \textbf{собственности} колхозного двора. Выражение <<в личном пользовании>> взято, по-видимому, из Примерного Устава сельскохозяйственной артели. Но в Примерном Уставе сельскохозяйственной артели допущена ошибка. В Конституции СССР, которая разрабатывалась более тщательно, сказано другое, а именно:
	
	
	
	<<Каждый колхозный двор… имеет в личной собственности подсобное хозяйство на приусадебном участке, жилой дом, продуктивный скот, птицу и мелкий сельскохозяйственный инвентарь>>.
	
	
	
	Это, конечно, правильно.
	
	
	
	Следовало бы, кроме того, поподробнее сказать, что каждый колхозник имеет в личной собственности от одной до стольких-то коров, смотря по местным условиям, столько-то овец, коз, свиней (тоже от — до, смотря по местным условиям) и неограниченное количество домашней птицы (уток, гусей, кур, индюшек).
	
	
	
	Это подробности имеют большое значение для наших зарубежных товарищей, которые хотят знать точно, что же, собственно осталось у колхозного двора в его личной собственности, после того как осуществлена у нас коллективизация сельского хозяйства.
	
	
	
	3) Вопрос о стоимости арендной платы крестьян помещикам, а так же о стоимости расходов на покупку земли.
	
	
	
	В проекте учебника сказано, что в результате национализации земли <<крестьянство освободилось от арендных платежей помещикам в сумме около 500 миллионов рублей ежегодно>> (надо сказать <<золотом>>). Эту цифру следовало бы уточнить, так как она учитывает, как мне кажется, арендную плату не во всей России, а только в большинстве губерний России. Надо при этом иметь в виду, что в ряде окраин России арендная плата уплачивалась натурой, что, видимо, не учтено авторами учебника. Кроме того, нужно иметь в виду, что крестьянство освободилось не только от арендной платы, но и от ежегодных расходов на покупку земли. Учтено ли это в проекте учебника? Мне кажется, что не учтено, а следовало бы учесть.
	
	
	
	4) Вопрос о сращивании монополий с государственным аппаратом.
	
	
	
	Выражение <<сращивание>> не подходит. Это выражение поверхностно и описательно отмечает сближение монополий и государства, но не раскрывает экономического смысла этого сближения. Дело в том, что в процессе этого сближения происходит не просто сращивание, а подчинение государственного аппарата монополистам. Поэтому следовало бы выкинуть слово <<сращивание>> и заменить его словами <<подчинение государственного аппарата монополиям>>.
	
	
	
	5) Вопрос о применении машин в СССР.
	
	
	
	В проекте учебника сказано, что <<в СССР машины применяются во всех случаях, когда они сберегают труд обществу>>. Это не совсем то, что следовало бы сказать. Во-первых, машины в СССР всегда сберегают труд обществу, ввиду чего мы не знаем случаев, когда бы они в условиях СССР не сберегали труд обществу. Во-вторых, машины не только сберегают труд, но они вместе с тем облегчают труд работников, ввиду чего в наших условиях, в отличие от условий капитализма, рабочие с большой охотой используют машины в процессе труда.
	
	
	
	Поэтому следовало бы сказать, что нигде так охотно не применяются машины, как в СССР, ибо машины сберегают труд обществу и облегчают труд рабочих, и, так как в СССР нет безработицы, рабочие с большой охотой используют машины в народном хозяйстве.
	
	
	
	6) Вопрос о материальном положении рабочего класса в капиталистических странах.
	
	
	
	Когда говорят о материальном положении рабочего класса, обычно имеют ввиду занятых в производстве рабочих и не принимают в расчёт материальное положение так называемой резервной армии безработных. Правильно ли такое отношение к вопросу о материальном положении рабочего класса? Я думаю, что неправильно. Если существует резервная армия безработных, членам которой нечем жить, кроме как продажей своей рабочей силы, то безработные не могут не входить в состав рабочего класса, но если они входят в состав рабочего класса, их нищенское положение не может не влиять на материальное положение рабочих, занятых в производстве. Я думаю поэтому, что при характеристике материального положения рабочего класса в капиталистических странах следовало бы принять в расчёт также положение резервной армии безработных рабочих.
	
	
	
	7) Вопрос о национальном доходе.
	
	
	
	Я думаю, что следовало бы \textbf{безусловно} включить в проект учебника новую главу о национальном доходе.
	
	
	
	8) Вопрос о специальной главе в учебнике о Ленине и Сталине, как о создателях политической экономии социализма.
	
	
	
	Я думаю, что главу <<Марксистское учение о социализме. Создание В.И. Лениным и И.В. Сталиным политической экономии социализма>> следует исключить из учебника. Она совершенно не нужна в учебнике, так как ничего нового не даёт и лишь бледно повторяет то, что более подробно сказано в предыдущих главах учебника.
	
	
	
	Что касается остальных вопросов, у меня нет каких-либо замечаний к <<предложениям>> товарищей Островитянова, Леонтьева, Шепилова, Гатовского и других.
	
	
	
	\vspace{0.4cm}
	
	
	
	\subsection[9. Международное значение марксистского учебника политической экономии]{\centering{9. МЕЖДУНАРОДНОЕ ЗНАЧЕНИЕ МАРКСИСТСКОГО УЧЕБНИКА ПОЛИТИЧЕСКОЙ ЭКОНОМИИ.}}
	
	
	
	Я думаю, что товарищи не учитывают всего значения марксистского учебника политической экономии. Учебник нужен не только для нашей советской молодёжи. Он особенно нужен для коммунистов всех стран и для людей, сочувствующих коммунистам. Наши зарубежные товарищи хотят знать, каким образом мы вырвались из капиталистической неволи, каким образом преобразовали мы экономику страны в духе социализма, как мы добились дружбы с крестьянством, как мы добились того, что наша недавно ещё нищая и слабая страна превратилась в страну богатую, могущественную, что из себя представляют колхозы, почему мы, несмотря на обобществление средств производства, не уничтожаем товарного производства, денег, торговли и т. д. Они хотят знать всё это и многое другое не для простого любопытства, а для того, чтобы учиться у нас и использовать наш опыт для своей страны. Поэтому появление хорошего марксистского учебника политической экономии имеет не только внутриполитическое, но и большое международное значение.
	
	
	
	Нужен, следовательно, учебник, который мог бы служить настольной книгой революционной молодёжи не только внутри страны, но и за рубежом. Он не должен быть слишком объёмистым, так как слишком объёмистый учебник не может быть настольной книгой и его трудно будет освоить — одолеть. Но он должен содержать всё основное, касающееся как экономики нашей страны, так и экономики капитализма и колониальной системы.
	
	
	
	Некоторые товарищи предлагали во время дискуссии включить в учебник целый ряд новых глав, историки — по истории, политики — по политике, философы — по философии, экономисты — по экономике. Но это привело бы к тому, что учебник разросся бы до необъятных размеров. Этого, конечно, нельзя допустить. Учебник использует исторический метод для иллюстрации проблем политической экономии, но это ещё не значит, что мы должны превратить учебник политической экономии в историю экономических отношений.
	
	
	
	Нам нужен учебник в 500, максимум в 600 страниц, — не больше. Это будет настольная книга по марксистской политической экономии, — хороший подарок молодым коммунистам всех стран.
	
	
	
	Впрочем, ввиду недостаточного уровня марксистского развития большинства компартий западных стран, такой учебник мог бы принести большую пользу также и не молодым кадровым коммунистам этих стран.
	
	
	
	\vspace{0.4cm}
	
	
	
	\subsection[10. Пути улучшения проекта учебника политической экономии]{\centering{10. ПУТИ УЛУЧШЕНИЯ ПРОЕКТА УЧЕБНИКА ПОЛИТИЧЕСКОЙ ЭКОНОМИИ.}}
	
	
	
	Некоторые товарищи во время дискуссии слишком усердно <<разносили>> проект учебника, ругали его авторов за ошибки и упущения, утверждали, что проект не удался. Это несправедливо. Конечно, ошибки и упущения имеются в учебнике, — они почти всегда бывают в большом деле. Но как бы там ни было, подавляющее число участников дискуссии всё же признало, что проект учебника может служить основой будущего учебника и нуждается лишь в некоторых поправках и дополнениях. Действительно, стоит только сравнить проект учебника с имеющимися в обращении учебниками политической экономии, чтобы придти к выводу, что проект учебника стоит на целую голову выше существующих учебников. В том большая заслуга авторов проекта учебника.
	
	
	
	Я думаю, что для улучшения проекта учебника следовало бы назначить немногочисленную комиссию со включением туда не только авторов учебника и не только сторонников большинства участников дискуссии, но и противников большинства, ярых критиков проекта учебника.
	
	
	
	Хорошо было бы включить в комиссию также опытного статистика для проверки цифр и внесения в проект новых статистических материалов, а также опытного юриста для проверки точности формулировок.
	
	
	
	Членов комиссии следовало бы освободить временно от всякой другой работы, обеспечив их полностью в материальном отношении, с тем, чтобы они могли целиком отдаться работе над учебником.
	
	
	
	Кроме того, следовало бы назначить редакционную комиссию, скажем, из трёх человек для окончательной редакции учебника. Это необходимо так же для того, чтобы добиться единства стиля, которого нет, к сожалению, в проекте учебника.
	
	
	
	Срок представления готового учебника в ЦК — 1 год.
	
	
	
	\begin{flushright}
		
		
		
		\textbf{И. СТАЛИН}
		
		
		
	\end{flushright}
	
	
	
	\noindent{\footnotesize 1952 г. 1 февраля. }
	
	
	
	\newpage
	
		\thispagestyle{empty} % выключаем отображение номера для этой страницы
	
	\ \\
	
	
	
	\ \\
	
	
	
	\ \\
	
	
	
	\ \\
	
	
	
	\ \\
	
	
	
	\section[ОТВЕТ т-щу НОТКИНУ, Александру Ильичу]{\centering{ОТВЕТ \\ т-щу НОТКИНУ, Александру Ильичу}}
	
	
	
	
	
	Товарищ Ноткин!
	
	
	
	Я не торопился с ответом, так как поставленные Вами вопросы не считаю срочными. Тем более, что есть другие вопросы, имеющие срочный характер, которые, естественно, отвлекают внимание в сторону от вашего письма.
	
	
	
	Отвечаю по пунктам.
	
	
	
	\subsubsection*{По пункту первому.}
	
	
	
	В <<Замечаниях>> имеется известное положение о том, что общество не бессильно перед лицом законов науки, что люди могут, познав экономические законы, использовать их в интересах общества. Вы утверждаете, что это положение не может быть распространено на другие формации общества, что оно может иметь силу лишь при социализме и коммунизме, что стихийный характер экономических процессов, например, при капитализме не даёт обществу возможности использовать экономические законы в интересах общества.
	
	
	
	Это неверно. В эпоху буржуазной революции, например, во Франции буржуазия использовала против феодализма известный закон об обязательном соответствии производственных отношений характеру производительных сил, низвергла феодальные производственные отношения, создала новые, буржуазные производственные отношения и привела эти производственные отношения в соответствие с характером производительных сил, выросших в недрах феодального строя. Буржуазия сделала это не в силу особых своих способностей, а потому, что она кровно была заинтересована в этом. Феодалы сопротивлялись этому делу не в силу своей тупости, а потому, что они кровно были заинтересованы помешать осуществлению этого закона.
	
	
	
	То же самое надо сказать о социалистической революции в нашей стране. Рабочий класс использовал закон обязательного соответствия производственных отношений характеру производительных сил, ниспроверг буржуазные производственные отношения, создал новые, социалистические производственные отношения и привёл их в соответствие с характером производительных сил. Он мог это сделать не в силу своих способностей, а потому, что он кровно был заинтересован в этом деле. Буржуазия, которая из передовой силы на заре буржуазной революции успела уже превратиться в контрреволюционную силу, всячески сопротивлялась проведению этого закона в жизнь, — сопротивлялась не в силу своей неорганизованности и не потому, что стихийный характер экономических процессов толкал её на сопротивление, а главным образом потому, что она была кровно заинтересована против проведения этого закона в жизнь.
	
	
	
	Следовательно:
	
	
	
	1. Использование экономических процессов, экономических законов в интересах общества происходит в той или иной мере не только при социализме и коммунизме, но и при других формациях;
	
	
	
	2. Использование экономических законов всегда и везде при классовом обществе имеет классовую подоплёку, причём знаменосцем использования экономических законов в интересах общества всегда и везде является передовой класс, тогда как отживающие классы сопротивляются этому делу.
	
	
	
	Разница в этом деле между пролетариатом, с одной стороны, и другими классами, когда-либо совершившими на протяжении истории перевороты в производственных отношениях, с другой стороны, состоит в том, что классовые интересы пролетариата сливаются с интересами подавляющего большинства общества, ибо революция пролетариата означает не уничтожение той или иной формы эксплуатации, а уничтожение всякой эксплуатации, тогда как революция других классов, уничтожая лишь ту или иную форму эксплуатации, ограничивались рамками их узкоклассовых интересов, находящихся в противоречии с интересами большинства общества.
	
	
	
	В <<Замечаниях>> говорится о классовой подоплёке дела использования экономических законов в интересах общества. Там сказано, что <<в отличие от законов естествознания, где открытие и применение нового закона происходит более или менее гладко, в экономической области открытие и применение нового закона, задевающего интересы отживших сил общества, встречают сильнейшее сопротивление со стороны этих сил>>. Однако вы не обратили на это внимания.
	
	
	
	\subsubsection*{По пункту второму.}
	
	
	
	
	
	Вы утверждаете, что полное соответствие производственных отношений характеру производительных сил может быть достигнуто лишь при социализме и коммунизме, а при других формациях может быть осуществлено лишь неполное соответствие.
	
	
	
	Это неверно. В эпоху после буржуазной революции, когда буржуазия разрушила феодальные производственные отношения и установила буржуазные производственные отношения, безусловно были периоды, когда буржуазные производственные отношения полностью соответствовали характеру производительных сил. В противном случае капитализм не мог бы развиваться с такой быстротой, с какой он развивался после буржуазной революции.
	
	
	
	Далее, нельзя понимать в абсолютном смысле слова <<полное соответствие>>. Их нельзя понимать так, что будто бы при социализме не существует никакого отставания производственных отношений от роста производительных сил. Производительные силы являются наиболее подвижными и революционными силами производства. Они бесспорно идут впереди производственных отношений и при социализме. Производственные отношения спустя лишь некоторое время преобразуются применительно к характеру производительных сил.
	
	
	
	Как же в таком случае понимать слова <<полное соответствие>>? Их следует понимать так, что при социализме дело обычно не доходит до конфликта между производственными отношениями и производительными силами, что общество имеет возможность своевременно привести в соответствие отстающие производственные отношения с характером производительных сил. Социалистическое общество имеет возможность сделать это, потому что оно не имеет в своём составе отживающих классов, могущих организовать сопротивление. Конечно, и при социализме будут отстающие инертные силы, не понимающие необходимости изменения в производственных отношениях, но их, конечно, не трудно будет преодолеть, не доводя дела до конфликта.
	
	
	
	\subsubsection*{По пункту третьему.}
	
	
	
	
	
	Из Ваших рассуждений вытекает, что средства производства и прежде всего орудия производства, производимые нашими национализированными предприятиями, Вы рассматриваете, как товар.
	
	
	
	Можно ли рассматривать средства производства при нашем социалистическом строе, как товар? По-моему, никак нельзя.
	
	
	
	Товар есть такой продукт производства, который продаётся любому покупателю, причём при продаже товара товаровладелец теряет право собственности на него, а покупатель становится собственником товара, который может перепродать, заложить, сгноить его. Подходят ли средства производства под такое определение? Ясно, что не подходят. Во-первых, средства производства <<продаются>> не всякому покупателю, они не <<продаются>> даже колхозам, они только распределяются государством среди своих предприятий. Во-вторых, владелец средств производства — государство при передачи их тому или иному предприятию ни в какой мере не теряет права собственности на средства производства, а наоборот, полностью сохраняет его. В-третьих, директора предприятий, получившие от государства средства производства, не только не становятся их собственниками, а наоборот, утверждаются, как уполномоченные советского государства по использовании средств производства, согласно планов, преподанных государством.
	
	
	
	Как видно, средства производства при нашем строе нельзя подвести под категорию товаров.
	
	
	
	Почему же в таком случае говорят о стоимости средств производства, об их себестоимости, об их цене и т. п.?
	
	
	
	По двум причинам.
	
	
	
	Во-первых, это необходимо для калькуляции, для расчётов, для определения доходности и убыточности предприятий, для проверки и контроля предприятий. Но это всего лишь формальная сторона дела.
	
	
	
	Во-вторых, это необходимо для того, чтобы в интересах внешней торговли осуществить дело продажи средств производства иностранным государствам. Здесь, в области внешней торговли, но \textbf{только в этой области}, наши средства производства действительно являются товарами и они действительно продаются (без кавычек).
	
	
	
	Выходит таким образом, что в области внешнеторгового оборота средства производства, производимые нашими предприятиями, сохраняют свойства товаров как по существу, так и формально, тогда как в области экономического оборота внутри страны средства производства теряют свойства товаров, перестают быть товарами и выходят за пределы сферы действия закона стоимости, сохраняя лишь внешнюю оболочку товаров (калькуляция и пр.).
	
	
	
	Чем объяснить это своеобразие?
	
	
	
	Дело в том, что в наших социалистических условиях экономическое развитие происходит не в порядке переворотов, а в порядке постепенных изменений, когда старое не просто отменяется начисто, а меняет свою природу применительно к новому, сохраняя лишь свою форму, а новое не просто уничтожает старое, а проникает в старое, меняет его природу, его функции, не ломая его форму, а используя его для развития нового. Так обстоит дело не только с товарами, но и с деньгами в нашем экономическом обороте, так же и с банками, которые, теряя свои старые функции и приобретая новые, сохраняют старую форму, используемую социалистическим строем.
	
	
	
	Если подойти к делу с точки зрения формальной, с точки зрения процессов, происходящих по поверхности явлений, можно притти к неправильному выводу о том, что категории капитализма сохраняют будто бы силу в нашей экономике. Если же подойти к делу с марксистским анализом, делающим строгое различие между содержанием экономического процесса и его формой, между глубинными процессами развития и поверхностными явлениями, — то можно притти к единственно правильному выводу о том, что от старых категорий капитализма сохранилась у нас главным образом форма, внешний облик, по существу же они изменились у нас коренным образом применительно к потребностям развития социалистического народного хозяйства.
	
	
	
	\subsubsection*{По пункту четвёртому.}
	
	
	
	
	
	Вы утверждаете, что закон стоимости оказывает регулирующее воздействие на цены <<средств производства>>, изготовляемых в сельском хозяйстве и сдаваемых государству по заготовительным ценам. Вы имеете при этом в виду такие <<средства производства>>, как сырьё, например хлопок. Вы могли бы добавить к этому так же лён, шерсть и прочее сельскохозяйственное сырье.
	
	
	
	Следует прежде всего отметить, что в данном случае сельское хозяйство производит не <<средства производства>>, а одно из средств производства — сырьё. Нельзя играть словами <<средства производства>>. Когда марксисты говорят о производстве средств производства, они имеют в виду прежде всего производство орудий производства, — то, что Маркс называет <<механическими средствами труда, совокупность которых можно назвать костной и мускульной системой производства>>, составляющей <<характерные отличительные признаки определённой эпохи общественного производства>>. Ставить на одну доску часть средств производства (сырьё) и средства производства, в том числе орудия производства, — значит грешить против марксизма, ибо марксизм исходит из определённой роли орудий производства в сравнении со всеми другими средствами производства. Всякому известно, что сырье само по себе не может производить орудий производства, хотя некоторые виды сырья и необходимы, как материал для производства орудий производства, тогда как никакое сырьё не может быть произведено без орудий производства.
	
	
	
	Далее. Является ли воздействие закона стоимости на цену сырья, производимого в сельском хозяйстве, регулирующим воздействием, как это утверждаете Вы, товарищ Ноткин? Оно было бы регулирующим, если бы у нас существовала <<свободная>> игра цен на сельскохозяйственное сырьё, если бы у нас действовал закон конкуренции и анархии производства, если бы у нас не было планового хозяйства, если бы производство сырья не регулировалось планом. Но так как все эти <<если>> отсутствуют в системе нашего народного хозяйства, то воздействие закона стоимости на цену сельскохозяйственного сырья никак не может быть регулирующим. Во-первых, цены у нас на сельскохозяйственное сырьё твёрдые, установленные планом, а не <<свободные>>. Во-вторых, размеры производства сельскохозяйственного сырья определяются не стихией и не какими либо случайными элементами, а планом. В-третьих, орудия производств, необходимые для производства сельскохозяйственного сырья, сосредоточены не в руках отдельных лиц, или групп лиц, а в руках государства. Что же остаётся после этого от регулирующей роли закона стоимости? Выходит, что сам закон стоимости регулируется указанными выше фактами, свойственными социалистическому производству.
	
	
	
	Следовательно, нельзя отрицать того, что закон стоимости воздействует на образование цен сельскохозяйственного сырья, что он является одним из факторов этого дела. Но тем более нельзя отрицать и того, что это воздействие не является и не может быть регулирующим.
	
	
	
	\subsubsection*{По пункту пятому.}
	
	
	
	
	
	Говоря о рентабельности социалистического народного хозяйства, я возражал в своих <<Замечаниях>> некоторым товарищам, которые утверждают, что поскольку наше плановое народное хозяйство не даёт большого предпочтения рентабельным предприятиям и допускает существование наряду с этими предприятиями также и нерентабельных предприятий, — оно убивает будто бы самый принцип рентабельности в хозяйстве. В <<Замечаниях>> сказано, что рентабельность с точки зрения отдельных предприятий и отраслей производства не идёт ни в какое сравнение с той высшей рентабельностью, которую даёт нам социалистическое производство, избавляя нас от кризисов перепроизводства и обеспечивая нам непрерывный рост производства.
	
	
	
	Но было бы неправильно делать из этого вывод, что рентабельность отдельных предприятий и отраслей производства не имеет особой ценности и не заслуживает того, чтобы обратить на неё серьёзное внимание. Это, конечно, неверно. Рентабельность отдельных предприятий и отраслей производства имеет громадное значение с точки зрения развития нашего производства. Она должна быть учитываема как при планировании строительства, так и при планировании производства. Это — азбука нашей хозяйственной деятельности на нынешнем этапе развития.
	
	
	
	\subsubsection*{По пункту шестому.}
	
	
	
	
	
	Неясно, как нужно понимать Ваши слова, касающиеся капитализма: <<расширенное производство в сильно деформированном виде>>. Нужно сказать, что таких производств, да ещё расширенных, не бывает на свете.
	
	
	
	Очевидно, что после того, как мировой рынок раскололся и сфера приложения сил главных капиталистических стран (США, Англия, Франция) и мировым ресурсам стала сокращаться, циклический характер развития капитализма — рост и сокращение производства — должен всё же сохраниться. Однако рост производства в этих странах будет происходить на суженной базе, ибо объём производства в этих странах будет сокращаться.
	
	
	
	\subsubsection*{По пункту седьмому.}
	
	
	
	
	
	Общий кризис мировой капиталистической системы начался в период первой мировой войны, особенно в результате отпадения Советского Союза от капиталистической системы. Это был первый этап общего кризиса. В период второй мировой войны развернулся второй этап общего кризиса, особенно после отпадения от капиталистической системы народно-демократических стран в Европе и в Азии. Первый кризис в период первой мировой войны и второй кризис во время второй мировой войны нужно рассматривать не как отдельные, оторванные друг от друга самостоятельные кризисы, а как этапы развития общего кризиса мировой капиталистической системы.
	
	
	
	Является ли общий кризис мирового капитализма только политическим или только экономическим кризисом? Ни то, ни другое. Он является общим, т. е. всесторонним кризисом мировой системы капитализма, охватывающим как экономику, так и политику. При этом понятно, что в основе его лежит всё более усиливающееся разложение мировой экономической системы капитализма, с одной стороны, и растущая экономическая мощь отпавших от капитализма стран — СССР, Китая и других народно-демократических стран, с другой стороны.
	
	
	
	
	
	\begin{flushright}
		
		
		
		\textbf{И. СТАЛИН}
		
		
		
	\end{flushright}
	
	
	
	\noindent{\footnotesize 21 апреля 1952 г. }
	
	
	
	\newpage
	
		\thispagestyle{empty} % выключаем отображение номера для этой страницы
	
	\ \\
	
	
	
	\ \\
	
	
	
	\ \\
	
	
	
	\ \\
	
	
	
	\ \\
	
	
	
	\section[ОБ ОШИБКАХ т. ЯРОШЕНКО Л.Д.]{\centering{ОБ ОШИБКАХ т. ЯРОШЕНКО Л.Д.}}
	
	
	
	Членам Политбюро ЦК ВКП(б) недавно было разослано товарищем Ярошенко письмо от 20 марта сего года по ряду экономических вопросов, обсуждавшихся на известной ноябрьской дискуссии. В письме имеется жалоба его автора на то, что в основных обобщающих документах по дискуссии, так же как и в <<Замечаниях>> товарища Сталина, <<не нашла никакого отражения точка зрения>> т. Ярошенко. В записке имеется кроме того предложение т. Ярошенко о том, чтобы разрешить ему составить <<Политическую экономию социализма>> в течение одного года или полутора лет, дав ему для этого двух помощников.
	
	
	
	Я думаю, что придётся рассмотреть по существу как жалобу т. Ярошенко, так и его предложение.
	
	
	
	Начнём с жалобы.
	
	
	
	Итак, в чём состоит <<точка зрения>> т. Ярошенко, которая не получила никакого отражения в названных выше документах.
	
	
	
	\vspace{0.4cm}
	
	
	
	\subsection[I. Главная ошибка т. Ярошенко]{\centering{I. \\ ГЛАВНАЯ ОШИБКА т. ЯРОШЕНКО}}
	
	
	
	Если охарактеризовать точку зрения т. Ярошенко в двух словах, то следует сказать, что она является немарксистской, — следовательно, глубоко ошибочной.
	
	
	
	Главная ошибка т. Ярошенко состоит в том, что он отходит от марксизма в вопросе о роли производительных сил и производственных отношений в развитии общества, чрезмерно преувеличивает роль производительных сил, также чрезмерно преуменьшает роль производственных отношений и кончает дело тем, что объявляет производственные отношения частью производительных сил.
	
	
	
	Тов. Ярошенко согласен признать некоторую роль за производственными отношениями в условиях <<антагонистических классовых противоречий>>, поскольку здесь производственные отношения <<противоречат развитию производительных сил>>. Но эту роль он ограничивает отрицательной ролью, ролью фактора, тормозящего развитие производительных сил, сковывающего их развитие. Других функций, каких-либо положительных функций производственных отношений т. Ярошенко не видит.
	
	
	
	Что касается социалистического строя, где уже нет <<антагонистических классовых противоречий>> и где производственные отношения <<больше не противоречат развитию производительных сил>>, - то т. Ярошенко считает, что здесь какая бы то ни было самостоятельная роль производственных отношений исчезает, производственные отношения перестают быть серьёзным фактором развития и они поглощаются производительными силами, как часть целым. При социализме <<производственные отношения людей, говорит т. Ярошенко, входят в организацию производительных сил, как средство, как момент этой организации>> (см. письмо т. Ярошенко в Политбюро ЦК).
	
	
	
	Какова же в таком случае главная задача Политической экономии социализма? Тов. Ярошенко отвечает: <<Главная проблема Политической экономии социализма поэтому \textbf{не в том}, чтобы изучать производственные отношения людей социалистического общества, \textbf{а} в \textbf{том}, чтобы разрабатывать и развивать научную теорию организации производительных сил в общественном производстве, теорию планирования развития народного хозяйства>> (см. речь т. Ярошенко на Пленуме дискуссии).
	
	
	
	Этим, собственно, и объясняется, что т. Ярошенко не интересуется такими экономическими вопросами социалистического строя, как наличие различных форм собственности в нашей экономике, товарное обращение, закон стоимости и проч., считая их второстепенными вопросами, вызывающими лишь схоластические споры. Он прямо заявляет, что в его Политической экономии социализма <<споры о роли той или другой категории политической экономии социализма — стоимость, товар, деньги, кредит и др., - принимающие зачастую у нас схоластический характер, \textbf{заменяются} здравыми рассуждениями о рациональной организации производительных сил в общественном производстве, научном обосновании такой организации>> (см. речь т. Ярошенко на Секции Пленума дискуссии).
	
	
	
	Следовательно, политическая экономия без экономических проблем.
	
	
	
	Тов. Ярошенко думает, что достаточно наладить <<рациональную организацию производительных сил>>, чтобы переход от социализма к коммунизму произошёл без особых трудностей. Он считает, что этого вполне достаточно для перехода к коммунизму. Он прямо заявляет, что <<при социализме основная борьба за построение коммунистического общества сводится к борьбе за правильную организацию производительных сил и рациональное их использование в общественном производстве>> (см. речь на пленуме дискуссии). Тов. Ярошенко торжественно провозглашает, что <<Коммунизм — это высшая научная организация производительных сил в общественном производстве>>.
	
	
	
	Выходит, оказывается, что существо коммунистического строя исчерпывается <<рациональной организацией производительных сил>>.
	
	
	
	Из всего этого т. Ярошенко делает вывод, что не может быть единой Политической экономии для всех общественных формаций, что должны быть две политические экономии: одна — для досоциалистических общественных формаций, предметом которой является изучение производственных отношений людей, другая — для социалистического строя, предметом которой должно являться не изучение производственных, т. е. экономических, отношений, а изучение вопросов рациональной организации производительных сил.
	
	
	
	Такова точка зрения т. Ярошенко.
	
	
	
	Что можно сказать об этой точке зрения?
	
	
	
	Неверно, во-первых, что роль производственных отношений в истории общества ограничивается ролью тормоза, сковывающего развитие производительных сил. Когда марксисты говорят о тормозящей роли производственных отношений, то они имеют в виду не всякие производственные отношения, а только старые производственные отношения, которые уже не соответствуют росту производительных сил и, следовательно, тормозят их развитие. Но кроме старых производственных отношений существуют, как известно, новые производственные отношения, заменяющие собой старые. Можно ли сказать, что роль новых производственных отношений сводится к роли тормоза производительных сил? Нет, нельзя. Наоборот, новые производственные отношения являются той \textbf{главной} и решающей силой, которая собственно и определяет дальнейшее, притом мощное развитие производительных сил и без которых производительные силы обречены на прозябание, как это имеет место в настоящее время в капиталистических странах.
	
	
	
	Никто не может отрицать колоссального развития производительных сил нашей советской промышленности в течение пятилеток. Но это развитие не имело бы места, если бы мы не заменили старые, капиталистические производственные отношения в октябре 1917 года новыми, социалистическими производственными отношениями. Без этого переворота в производственных, экономических отношениях нашей страны производительные силы прозябали бы у нас так же, как они прозябают теперь в капиталистических странах.
	
	
	
	Никто не может отрицать колоссального развития производительных сил нашего сельского хозяйства за последние 20–25 лет. Но это развитие не имело бы места, если бы мы не заменили в тридцатых годах старые производственные капиталистические отношения в деревне новыми, коллективистическими производственными отношениями. Без этого производственного переворота производительные силы нашего сельского хозяйства прозябали бы так же, как они прозябают теперь в капиталистических странах.
	
	
	
	Конечно, новые производственные отношения не могут остаться и не остаются вечно новыми, они начинают стареть и впадать в противоречие с дальнейшим развитием производительных сил, они начинают терять роль главного двигателя производительных сил и превращаются в тормоз. Тогда на место таких производственных отношений, ставших уже старыми, появляются новые производственные отношения, роль которых состоит в том, чтобы быть главным двигателем дальнейшего развития производительных сил.
	
	
	
	Это своеобразие развития производственных отношений от роли тормоза производительных сил к роли главного их двигателя вперёд и от роли главного двигателя к роли тормоза производительных сил, — составляет один из главных элементов марксистской материалистической диалектики. Это знают теперь все приготовишки от марксизма. Этого не знает, оказывается, т. Ярошенко.
	
	
	
	Неверно, во-вторых, что самостоятельная роль производственных, т. е. экономических, отношений исчезает при социализме, что производственные отношения поглощаются производительными силами, что общественное производство при социализме сводится к организации производительных сил. Марксизм рассматривает общественное производство, как целое, имеющее две неразрывные стороны: производительные силы общества (отношения общества к природным силам, в борьбе с которыми оно добывает необходимые материальные блага) и производственные отношения (отношения людей к друг другу в процессе производства). Это — две различные стороны общественного производства, хотя они связаны друг с другом неразрывно. И именно потому, что они являются различными сторонами общественного производства, они могут воздействовать друг на друга. Утверждать, что одна из этих сторон может быть поглощена другой и превращена в её собственную часть, — значит серьёзнейшим образом согрешить против марксизма.
	
	
	
	Маркс говорит:
	
	
	
	<<В производстве люди воздействуют не только на природу, но и друг на друга. Они не могут производить, не соединяясь известным образом для совместной деятельности и для взаимного обмена своей деятельностью. Чтобы производить, люди вступают в определённые связи и отношения, и только через посредство этих общественных связей и отношений существует их отношение к природе, имеет место производство>> (см. <<К. Маркс и Ф. Энгельс>>, т. V, стр.429).
	
	
	
	Следовательно, общественное производство состоит из двух сторон, которые при всём том, что они неразрывно связаны друг с другом, отражают всё же два ряда различных отношений: отношения людей к природе (производительные силы) и отношения людей друг к другу в процессе производства (производственные отношения). Только наличие обеих сторон производства даёт нам общественное производство, всё равно, идёт ли речь о социалистическом строе или о других общественных формациях.
	
	
	
	Тов. Ярошенко, очевидно, не вполне согласен с Марксом. Он считает, что это положение Маркса не применимо к социалистическому строю. Именно поэтому он сводит проблему Политической экономии социализма к задаче рациональной организации производительных сил, отбрасывая прочь производственные, экономические отношения и отрывая от них производительные силы.
	
	
	
	Следовательно, вместо марксистской Политической экономии у т. Ярошенко получается что-то вроде <<Всеобщей организационной науки>> Богданова.
	
	
	
	Таким образом, взяв правильную мысль о том, что производительные силы являются наиболее подвижными и революционными силами производства, т. Ярошенко доводит эту мысль до абсурда, до отрицания роли производственных, экономических отношений при социализме, причём вместо полнокровного общественного производства у него получается однобокая и тощая технология производства, — что-то вроде Бухаринской <<общественно-организационной техники>>.
	
	
	
	Маркс говорит:
	
	
	
	<<В общественном производстве своей жизни (то есть в производстве материальных благ, необходимых для жизни людей — И. Ст.) люди вступают в определённые, необходимые, от их воли не зависящие отношения — производственные отношения, которые соответствуют определённой ступени развития их материальных производительных сил. Совокупность этих производственных отношений составляет экономическую структуру общества, реальный базис, на котором возвышается юридическая и политическая надстройка и которому соответствуют определённые формы общественного сознания>> (см. предисловия <<К критике политической экономии>>).
	
	
	
	Это значит, что каждая общественная формация, в том числе и социалистическое общество, имеет свой экономический базис, состоящий из совокупности производственных отношений людей. Встаёт вопрос, как обстоит дело у т. Ярошенко с экономическим базисом социалистического строя? Как известно, т. Ярошенко уже ликвидировал производственные отношения при социализме, как более или менее самостоятельную область, включив то малое, что осталось от них, в состав организации производительных сил. Спрашивается, имеет ли социалистический строй свой собственный экономический базис? Очевидно, что, поскольку производственные отношения исчезли при социализме, как более или менее самостоятельная сила, социалистический строй остаётся без своего экономического базиса.
	
	
	
	Следовательно, социалистический строй без своего экономического базиса. Получается довольно весёлая история…
	
	
	
	Возможен ли вообще общественный строй без своего экономического базиса? Тов. Ярошенко, очевидно, считает, что возможен. Ну, а марксизм считает, что таких общественных строев не бывает на свете.
	
	
	
	Неверно, наконец, что коммунизм есть рациональная организация производительных сил, что рациональная организация производительных сил исчерпывает существо коммунистического строя, что стоит рационально организовать производительные силы, чтобы перейти к коммунизму без особых трудностей. В нашей литературе имеется другое определение, другая формула коммунизма, а именно ленинская формула: <<Коммунизм есть советская власть плюс электрификация всей страны>>. Тов-щу Ярошенко, очевидно, не нравится ленинская формула, и он заменяет её своей собственной самодельной формулой: <<Коммунизм — это высшая научная организация производительных сил в общественном производстве>>.
	
	
	
	Во-первых, никому не известно, что из себя представляет эта, рекламируемая т-щем Ярошенко, <<высшая научная>> или <<рациональная>> организация производительных сил, каково её конкретное содержание? Тов. Ярошенко десятки раз повторяет эту мифическую формулу в своих речах на Пленуме, секциях дискуссии, в своём письме на имя членов Политбюро, но он нигде ни единым словом не пытается разъяснить, как собственно следует понимать <<рациональную организацию>> производительных сил, которая якобы исчерпывает собой сущность коммунистического строя.
	
	
	
	Во-вторых, если уж сделать выбор между двумя формулами, то следует отбросить не ленинскую формулу, являющуюся единственно правильной, а так называемую формулу тов. Ярошенко, явно надуманную и немарксистскую, взятую из Богдановского арсенала — <<Всеобщей организационной науки>>.
	
	
	
	Тов. Ярошенко думает, что стоит добиться рациональной организации производительных сил, чтобы получить изобилие продуктов и перейти к коммунизму, перейти от формулы: <<каждому по труду>> к формуле: <<каждому по потребностям>>. Это большое заблуждение, изобличающее полное непонимание законов экономического развития социализма. Тов. Ярошенко слишком просто, по-детски просто представляет условия перехода от социализма к коммунизму. Тов. Ярошенко не понимает, что нельзя добиться ни изобилия продуктов, могущего покрыть все потребности общества, ни перехода к формуле <<каждому по потребностям>>, оставляя в силе такие экономические факты, как колхозно-групповая собственность, товарное обращение и т. п. Тов. Ярошенко не понимает, что раньше, чем перейти к формуле <<каждому по потребностям>>, нужно пройти ряд этапов культурного перевоспитания общества, в течение которых труд из средства только лишь поддержания жизни будет превращён в глазах общества в первую жизненную потребность, а общественная собственность — в незыблемую и неприкосновенную основу существования общества.
	
	
	
	Для того, чтобы подготовить действительный, а не декларативный переход к коммунизму, нужно осуществить по крайней мере три основных предварительных условия.
	
	
	
	1. Необходимо, во-первых, прочно обеспечить не мифическую <<рациональную организацию>> производительных сил, а непрерывный рост всего общественного производства с преимущественным ростом производства средств производства. Преимущественный рост производства средств производства необходим не только потому, что оно должно обеспечить оборудованием как свои собственные предприятия, так и предприятия всех остальных отраслей народного хозяйства, но и потому, что без него вообще невозможно осуществить расширенное производство.
	
	
	
	2. Необходимо, во-вторых, путём постепенных переходов, осуществляемых с выгодой для колхозов и, следовательно, для всего общества, поднять колхозную собственность до уровня общенародной собственности, а товарное обращение тоже путём постепенных переходов заменить системой продуктообмена, чтобы центральная власть или другой какой-либо общественно-экономический центр мог охватить всю продукцию общественного производства в интересах общества.
	
	
	
	Тов. Ярошенко ошибается, утверждая, что при социализме нет никакого противоречия между производственными отношениями и производительными силами общества. Конечно, наши нынешние производственные отношения переживают тот период, когда они, вполне соответствуя росту производительных сил, двигают их вперёд семимильными шагами. Но было бы неправильно успокаиваться на этом и думать, что не существует никаких противоречий между нашими производительными силами и производственными отношениями. Противоречия безусловно есть и будут, поскольку развитие производственных отношений отстаёт и будет отставать от развития производительных сил. При правильной политике руководящих органов эти противоречия не могут превратиться в противоположность, и дело здесь не может дойти до конфликта между производственными отношениями и производительными силами общества. Другое дело, если мы будем проводить неправильную политику, вроде той, которую рекомендует т. Ярошенко. В этом случае конфликт будет неизбежен, и наши производственные отношения могут превратиться в серьёзнейший тормоз дальнейшего развития производительных сил.
	
	
	
	Поэтому задача руководящих органов состоит в том, чтобы своевременно подметить нарастающие противоречия и вовремя принять меры к их преодолению путём приспособления производственных отношений к росту производительных сил. Это касается прежде всего таких экономических явлений, как групповая — колхозная собственность, товарное обращение. Конечно, в настоящее время эти явления с успехом используются нами для развития социалистического хозяйства и они приносят нашему обществу несомненную пользу. Несомненно, что они будут приносить пользу и в ближайшем будущем. Но было бы непростительной слепотой не видеть, что эти явления вместе с тем уже теперь начинают тормозить мощное развитие наших производительных сил, поскольку они создают препятствия для полного охвата всего народного хозяйства, особенно сельского хозяйства, государственным планированием. Не может быть сомнения, что чем дальше, тем больше будут тормозить эти явления дальнейший рост производительных сил нашей страны. Следовательно, задача состоит в том, чтобы ликвидировать эти противоречия путём постепенного превращения колхозной собственности в общенародную собственность и введения продуктообмена — тоже в порядке постепенности — вместо товарного обращения.
	
	
	
	3. Необходимо, в-третьих, добиться такого культурного роста общества, который бы обеспечил всем членам общества всестороннее развитие их физических и умственных способностей, чтобы члены общества имели возможность получить образование, достаточное для того, чтобы стать активными деятелями общественного развития, чтобы они имели возможность свободно выбирать профессию, а не быть прикованными на всю жизнь, в силу существующего разделения труда, к одной какой-либо профессии.
	
	
	
	Что требуется для этого?
	
	
	
	Было бы неправильно думать, что можно добиться такого серьёзного культурного роста членов общества без серьёзных изменений в нынешнем положении труда. Для этого нужно прежде всего сократить рабочий день по крайней мере до 6, а потом и до 5 часов. Это необходимо для того, чтобы члены общества получили достаточно свободного времени, необходимого для получения всестороннего образования. Для этого нужно, далее, ввести общеобязательное политехническое обучение, необходимое для того, чтобы члены общества имели возможность свободно выбирать профессию и не быть прикованными на всю жизнь к одной какой-либо профессии. Для этого нужно, дальше, коренным образом улучшить жилищные условия и поднять реальную зарплату рабочих и служащих минимум вдвое, если не больше, как путём прямого повышения денежной зарплаты, так и, особенно, путём дальнейшего систематического снижения цен на предметы массового потребления.
	
	
	
	Таковы основные условия подготовки перехода к коммунизму.
	
	
	
	Только после выполнения \textbf{всех} этих предварительных условий, взятых вместе, можно будет надеяться, что труд будет превращён в глазах членов общества из обузы <<в первую жизненную необходимость>> (Маркс), что <<труд из тяжёлого бремени превратится в наслаждение>> (Энгельс), что общественная собственность будет расцениваться всеми членами общества как незыблемая и неприкосновенная основа существования общества.
	
	
	
	Только после выполнения \textbf{всех} этих предварительных условий, взятых вместе, можно будет перейти от социалистической формулы — <<от каждого по способностям, каждому по труду>> к коммунистической формуле — <<каждый по способностям, каждому по потребностям>>.
	
	
	
	Это будет коренной переход от одной экономики, от экономики социализма — к другой, высшей экономике, к экономике коммунизма.
	
	
	
	Как видно, дело с переходом от социализма к коммунизму обстоит не так просто, как это воображает т. Ярошенко.
	
	
	
	Пытаться свести всё это сложное и многообразное дело, требующее серьёзнейших экономических изменений, к <<рациональной организации производительных сил>>, как это делает т. Ярошенко, — значит подменить марксизм богдановщиной.
	
	
	
	\vspace{0.4cm}
	
	
	
	\subsection[II. Другие ошибки тов. Ярошенко]{\centering{II. \\ ДРУГИЕ ОШИБКИ тов. ЯРОШЕНКО}}
	
	
	
	1. Из своей неправильной точки зрения т. Ярошенко делает неправильные выводы о характере и предмете политической экономии.
	
	
	
	Тов. Ярошенко отрицает необходимость единой политической экономии для всех общественных формаций, исходя из того, что каждая общественная формация имеет свои специфические законы. Но он совершенно неправ, и он расходится здесь с такими марксистами, как Энгельс, Ленин.
	
	
	
	Энгельс говорит, что политическая экономия есть <<наука об условиях и формах, при которых происходит производство и обмен в \textbf{различных человеческих обществах} и при которых, соответственно этому, всякий раз происходит распределение продуктов>> (<<Анти-Дюринг>>). Следовательно, политическая экономия изучает законы экономического развития не одной какой-либо общественной формации, а различных общественных формаций.
	
	
	
	С этим, как известно, вполне согласен Ленин, который в своих критических замечаниях по поводу книжки Бухарина <<Экономика переходного периода>> сказал, что Бухарин неправ, ограничивая сферу действия политической экономии товарным и прежде всего капиталистическим производством, заметив при этом, что Бухарин делает здесь <<шаг назад против Энгельса>>.
	
	
	
	С этим вполне согласуется определение политической экономии, данное в проекте учебника политической экономии, где сказано, что политическая экономия есть наука, изучающая <<законы общественного производства и распределения материальных благ на \textbf{различных ступенях} развития человеческого общества>>.
	
	
	
	Оно и понятно. Различные общественные формации в своём экономическом развитии подчиняются не только своим специфическим экономическим законам, которые общи для всех формаций, например, таким законам, как закон об единстве производительных сил и производственных отношений в едином общественном производстве, закон об отношениях между производительными силами и производственными отношениями в процессе развития всех общественных формаций. Стало быть, общественные формации не только отделены друг от друга своими специфическими законами, но и связаны друг с другом общими для всех формаций экономическими законами.
	
	
	
	Энгельс был совершенно прав, когда он говорил:
	
	
	
	<<Чтобы всесторонне провести эту критику буржуазной политической экономии, недостаточно было знакомства с капиталистической формой производства, обмена и распределения. Нужно было также, хотя бы в общих чертах, исследовать и привлечь к сравнению те формы, которые ей предшествовали, или те, которые существуют ещё рядом с ней в менее развитых странах>> (<<Анти-Дюринг>>).
	
	
	
	Очевидно, что здесь, в этом вопросе т. Ярошенко перекликается с Бухариным.
	
	
	
	Далее. Тов. Ярошенко утверждает, что в его <<Политической экономии социализма>> <<категории политической экономии — стоимость, товар, деньги, кредит и др. — \textbf{заменяются} здравыми рассуждениями о рациональной организации производительных сил в общественном производстве>>, что, следовательно, предметом \textbf{этой} политической экономии являются \textbf{не} производственные отношения социализма, \textbf{а} <<разработка и развитие научной теории организации производственных сил, теории планирования народного хозяйства и т. п.>>, что производственные отношения при социализме теряют своё самостоятельное значение и поглощаются производительными силами, как их составная часть.
	
	
	
	Нужно сказать, что такой несусветной тарабарщины не разводил ещё у нас ни один свихнувшийся <<марксист>>. Ведь, что значит политическая экономия социализма без экономических, производственных проблем? Разве бывает на свете такая политическая экономия? Что значит заменить в политической экономии социализма экономические проблемы проблемами организации производительных сил? Это значит ликвидировать политическую экономию социализма. Тов. Ярошенко так именно и поступает, — он ликвидирует политическую экономию социализма. Здесь он полностью смыкается с Бухариным. Бухарин \textbf{говорил}, что с уничтожением капитализма должна уничтожиться политическая экономия. Тов. Ярошенко этого не говорит, но он это делает, ликвидируя политическую экономию социализма. Правда, при этом он делает вид, что не вполне согласен с Бухариным, но это — хитрость, при том хитрость копеечная. На самом деле он делает то, что проповедовал Бухарин и против чего выступал Ленин. Тов. Ярошенко плетётся по стопам Бухарина.
	
	
	
	Дальше. Тов. Ярошенко проблемы политической экономии социализма сводит к проблемам рациональной организации производительных сил, к проблемам планирования народного хозяйства и т. п. Но он глубоко заблуждается. Проблемы рациональной организации производительных сил, планирование народного хозяйства и т. п. являются не предметом политической экономии, а предметом хозяйственной политики руководящих органов. Это две различные области, которых нельзя смешивать. Тов. Ярошенко спутал эти две различные вещи и попал впросак. Политическая экономия изучает законы о развитии производственных отношений людей. Хозяйственная политика делает из этого практические выводы, конкретизирует их и строит на этом свою повседневную работу. Загружать политическую экономию вопросами хозяйственной политики значит загубить её, как науку.
	
	
	
	Предметом политической экономии являются производственные, экономические отношения людей. Сюда относятся: а) формы собственности на средства производства; б) вытекающее из этого положение различных социальных групп в производстве и их взаимоотношения, или, как говорит Маркс: <<взаимный обмен своей деятельностью>>; в) всецело зависимые от них формы распределения продуктов. Всё это вместе составляет предмет политической экономии.
	
	
	
	В этом определении отсутствует слово <<обмен>>, фигурирующее в определении Энгельса. Оно отсутствует потому, что <<обмен>> понимается многими обычно, как обмен товаров, свойственный не всем, а лишь некоторым общественным формациям, что вызывает иногда недоразумение, хотя Энгельс под словом <<обмен>> понимал не только товарный обмен. Однако, как видно, то, что Энгельс понимал под словом <<обмен>>, нашло своё место в упомянутом определении, как его составная часть. Следовательно, по своему содержанию это определение предмета политической экономии полностью совпадает с определением Энгельса.
	
	
	
	2. Когда говорят об основном экономическом законе той или иной общественной формации, обычно исходят из того, что последняя не может иметь несколько основных экономических законов, что она может иметь лишь один какой-либо основной экономический закон, именно как \textbf{основной} закон. В противном случае мы имели бы несколько основных экономических законов для каждой общественной формации, что противоречит самому понятию об основном законе. Однако т. Ярошенко с этим не согласен. Он считает, что можно иметь не один, а несколько основных экономических законов социализма. Это невероятно, но это факт. В своей речи на Пленуме дискуссии он говорит:
	
	
	
	<<Величины и соотношения материальных фондов общественного производства и воспроизводства определяются наличием и перспективой роста рабочей силы, вовлекаемой в общественное производство. Это есть основной экономический закон социалистического общества, обуславливающий структуру социалистического общественного производства и воспроизводства>>.
	
	
	
	Это первый основной экономический закон социализма.
	
	
	
	В той же речи т. Ярошенко заявляет:
	
	
	
	<<Соотношения между I и II подразделениями обуславливаются в социалистическом обществе потребностью производства средств производства в размерах, необходимых для вовлечения в общественное производство всего работоспособного населения. Это основной экономический закон социализма и в то же время это требование нашей Конституции, вытекающее из права на труд советских людей>>.
	
	
	
	Это, так сказать, второй основной экономический закон социализма.
	
	
	
	Наконец, в своём письме на имя членов Политбюро т. Ярошенко заявляет:
	
	
	
	<<Исходя из этого, существенные черты и требования основного экономического закона социализма можно сформулировать, мне кажется, примерно следующим образом: непрерывно растущее и совершенствующееся производство материальных и культурных условий жизни общества>>.
	
	
	
	Это уже третий основной экономический закон социализма.
	
	
	
	Все ли эти законы являются основными экономическими законами социализма или только один из них, а если только один из них, то какой именно, — на эти вопросы т. Ярошенко не даёт ответа в своём последнем письме на имя членов Политбюро. Формулируя основной экономический закон социализма в своём письме на имя членов Политбюро, он, надо полагать, <<забыл>>, что в своей речи на Пленуме дискуссии три месяца назад он уже сформулировал два других основных экономических закона социализма, видимо, полагая, что на эту более чем сомнительную комбинацию не обратят внимания. Но, как видно, его расчёты не оправдались.
	
	
	
	Допустим, что первых двух основных экономических законов социализма, сформулированных тов-щем Ярошенко, не существует больше, что основным экономическим законом социализма т. Ярошенко отныне считает третью его формулировку, изложенную в письме на имя членов Политбюро. Обратимся к письму т. Ярошенко.
	
	
	
	Тов. Ярошенко говорит в этом письме, что он не согласен с определением основного экономического закона социализма, данном в <<Замечаниях>> т. Сталина. Он говорит:
	
	
	
	<<Главным в этом определении является <<обеспечение максимального удовлетворения… потребностей всего общества>>. Производство показано здесь как средство для достижения этой главной цели — удовлетворения потребностей. Такое определение даёт основание полагать, что формулированный Вами основной экономический закон социализма исходит не из примата производства, а из примата потребления>>.
	
	
	
	Очевидно, что т. Ярошенко совершенно не понял существа проблемы и не видит того, что разговоры о примате потребления или производства совершенно не имеют отношения к делу. Когда говорят о примате тех или иных общественных процессов перед другими процессами, то исходят обычно из того, что оба эти процесса являются более или менее однородными. Можно и нужно говорить о примате производства средств производства перед производством средств потребления, так как и в том и в другом случае мы имеем дело с производством, следовательно, они более или менее однородны. Но нельзя говорить, неправильно было бы говорить о примате потребления перед производством или производства перед потреблением, так как производство и потребление представляют две совершенно различные области, правда, связанные друг с другом, но всё же различные области. Тов. Ярошенко очевидно не понимает, что речь идёт здесь не о примате потребления или производства, а о том, какую \textbf{цель} ставит общество перед общественным производством, какой \textbf{задаче} подчиняет оно общественное производство, скажем, при социализме. Поэтому совершенно не относятся к делу также разговоры т. Ярошенко о том, что <<основу жизни социалистического общества, как и всякого другого общества, составляет производство>>. Тов. Ярошенко забывает, что люди производят не для производства, а для удовлетворения своих потребностей. Он забывает, что производство, оторванное от удовлетворения потребностей общества, хиреет и гибнет.
	
	
	
	Можно ли вообще говорить о цели капиталистического или социалистического производства, о задачах, которым подчинено капиталистическое или социалистическое производство? Я думаю, что можно и должно.
	
	
	
	Маркс говорит:
	
	
	
	<<Непосредственной целью капиталистического производства является производство не товаров, а прибавочной стоимости, или прибыли в её развитой форме; не продукта, а прибавочного продукта. С этой точки зрения самый труд производителен лишь постольку, поскольку он создаёт прибыль или прибавочный продукт для капитала. Поскольку рабочий этого не создаёт, его труд не производителен. Масса применённого производительного труда, следовательно, представляет для капитала интерес лишь постольку, поскольку благодаря ей — или соответственно ей — растёт количество прибавочного труда; лишь постольку необходимо то, что мы называем необходимым рабочим временем. Постольку труд не даёт этого результата, он является излишним и должен быть прекращён.
	
	
	
	Цель капиталистического производства всегда состоит в создании максимума прибавочной стоимости или максимума прибавочного продукта с минимумом авансированного капитала; поскольку этот результат не достигается чрезмерным трудом рабочих, возникает тенденция капитала, состоящая в стремлении произвести данный продукт с возможно меньшей затратой, — в стремлении к сбережению рабочей силы и издержек…
	
	
	
	Сами рабочие представляются при таком понимании тем, чем они действительно являются в капиталистическом производстве, — только средствами производства, а не самоцелью и не целью производства>>. (См. <<Теории прибавочной стоимости>>, том II, часть 2).
	
	
	
	Эти слова Маркса замечательны не только в том отношении, что они коротко и точно определяют цель капиталистического производства, но и в том отношении, что они намечают ту основную цель, ту главную задачу, которая должна быть поставлена перед социалистическим производством.
	
	
	
	Следовательно, цель капиталистического производства — извлечение прибылей. Что касается потребления, оно нужно капитализму лишь постольку, поскольку оно обеспечивает задачу извлечения прибылей. Вне этого вопрос о потреблении теряет для капитализма смысл. Человек с его потребностями исчезает из поля зрения.
	
	
	
	Какова же цель социалистического производства, какова та главная задача, выполнению которой должно быть подчинено общественное производство при социализме?
	
	
	
	Цель социалистического производства не прибыль, а человек с его потребностями, то есть удовлетворение его материальных и культурных потребностей. Цель социалистического производства, как говорится в <<Замечаниях>> т. Сталина: <<обеспечение максимального удовлетворения постоянно растущих материальных и культурных потребностей всего общества>>.
	
	
	
	Тов. Ярошенко думает, что он имеет здесь дело с <<приматом>> потребления перед производством. Это, конечно, недомыслие. На самом деле мы имеем здесь дело не с приматом потребления, а с \textbf{подчинением} социалистического производства основной его цели обеспечения максимального удовлетворения постоянно растущих материальных и культурных потребностей всего общества.
	
	
	
	Следовательно, обеспечение максимального удовлетворения постоянно растущих материальных и культурных потребностей всего общества, — это \textbf{цель} социалистического производства; непрерывный рост и совершенствование социалистического производства на базе высшей техники, — это \textbf{средство} для достижения цели.
	
	
	
	Таков основной экономический закон социализма.
	
	
	
	Желая сохранить так называемый <<примат>> производства перед употреблением, т. Ярошенко утверждает, что <<основной экономический закон социализма>> состоит <<в непрерывном росте и совершенствовании производства материальных и культурных условий общества>>. Это совершенно неверно. Тов. Ярошенко грубо извращает и портит формулу, изложенную в <<Замечаниях>> т. Сталина. У него производство из средства превращается в цель, а обеспечение максимального удовлетворения постоянно растущих материальных и культурных потребностей общества — исключается. Получается рост производства для роста производства, производство, как самоцель, а человек с его потребностями исчезает из поля зрения тов-ща Ярошенко.
	
	
	
	Поэтому неудивительно, что вместе с исчезновением человека, как цели социалистического производства, исчезают в <<концепции>> тов-ща Ярошенко последние остатки марксизма.
	
	
	
	Таким образом у т. Ярошенко получился не <<примат>> производства перед потреблением, а что-то вроде <<примата>> буржуазной идеологии перед идеологией марксистской.
	
	
	
	3. Особо стоит вопрос о марксовой теории воспроизводства. Тов. Ярошенко утверждает, что марксова теория воспроизводства является теорией только лишь капиталистического воспроизводства, что она не содержит чего-либо такого, что могло бы иметь силу для других общественных формаций, в том числе для социалистической общественной формации. Он говорит:
	
	
	
	<<Перенесение схемы воспроизводства Маркса, разработанной им для капиталистического хозяйства, на социалистическое общественное производство является продуктом догматического понимания учения Маркса и противоречит сущности его учения>>. (см. речь т. Ярошенко на Пленуме дискуссии).
	
	
	
	Он утверждает, далее, что <<Схема воспроизводства Маркса не соответствует экономическим законам социалистического общества и не может служить основой для социалистического воспроизводства>>. (См. там же).
	
	
	
	Касаясь марксовой теории простого воспроизводства, где устанавливается определённое соотношение между производством средств производства (I-ое подразделение) и производством средств потребления (II-е подразделение), т. Ярошенко говорит:
	
	
	
	<<Соотношение между первым и вторым подразделениями не обусловливается в социалистическом обществе формулой Маркса В+М первого подразделения и С второго подразделения. В условиях социализма указанная взаимосвязь в развитии между первым и вторым подразделениями не должна иметь места>>. (См. там же).
	
	
	
	Он утверждает, что <<Разработанная Марксом теория о соотношениях I и II подразделений неприемлема в наших социалистических условиях, так как в основе теории Маркса лежит капиталистическое хозяйство с его законами>>. (См. письмо т. Ярошенко на имя членов Политбюро).
	
	
	
	Так разносит т. Ярошенко марксову теорию воспроизводства.
	
	
	
	Конечно, марксова теория воспроизводства, выработанная в результате изучения законов капиталистического производства, отражает специфику капиталистического производства и, естественно, облечена в форму товарно-капиталистических стоимостных отношений. Иначе и не могло быть. Но видеть в марксовой теории воспроизводства только эту форму, и не замечать её основы, не замечать её основного содержания, имеющего силу не только для капиталистической общественной формации, — значит ничего не понять в этой теории. Если бы т. Ярошенко понимал что-либо в этом деле, то он понял бы и ту очевидную истину, что марксовы схемы воспроизводства отнюдь не исчерпываются отражением специфики капиталистического производства, что они содержат вместе с тем целый ряд основных положений воспроизводства, имеющих силу для всех общественных формаций, в том числе и особенно для социалистической общественной формации. Такие основные положения марксовой теории воспроизводства, как положение о разделении общественного производства на производство средств производства и производство средств потребления; положение о преимущественном росте производства средств производства при расширенном воспроизводстве; положение о соотношении между I и II подразделении; положение о прибавочном продукте, как единственном источнике накопления; положение об образовании и назначении общественных фондов; положение о накоплении, как единственном источнике расширенного производства, — все эти основные положения марксовой теории воспроизводства являются теми самыми положениями, которые имеют силу не только для капиталистической формации и без применения которых не может обойтись ни одно социалистическое общество при планировании народного хозяйства. Характерно, что сам т. Ярошенко, так высокомерно фыркающий на марксовы <<схемы воспроизводства>>, вынужден сплошь и рядом прибегать к помощи этих <<схем>> при обсуждении вопросов социалистического воспроизводства.
	
	
	
	А как смотрели на это дело Ленин, Маркс?
	
	
	
	Всем известны критические замечания Ленина на книгу Бухарина <<Экономика переходного периода>>. В этих замечаниях Ленин, как известно, признал, что марксова формула соотношения между I и II подразделениями, против которой ополчается т. Ярошенко, остаётся в силе как для социализма, так и для <<чистого коммунизма>>, т. е. для второй фазы коммунизма.
	
	
	
	Что касается Маркса, то он, как известно, не любил отвлекаться в сторону от изучения законов капиталистического производства и не занимался в своём <<Капитале>> вопросом о применимости его схем воспроизводства к социализму. Однако в 20 главе II тома <<Капитала>> в рубрике <<постоянный капитал подразделения I>>, где он трактует об обмене продуктов I подразделения внутри этого подразделения, Маркс как бы мимоходом замечает, что обмен продуктов в этом подразделении протекал бы при социализме с таким же постоянством, как при капиталистическом производстве. Маркс говорит:
	
	
	
	<<Если бы производство было общественным, а не капиталистическим, то ясно, что продукты подразделения I в целях воспроизводства не с меньшим постоянством распределялись бы как средства производства между отраслями производства этого подразделения: одна часть непосредственно осталась бы в той сфере производства, из которой она вышла как продукт, напротив, другая переходила бы в другие места производства, и таким образом между различными местами производства этого подразделения установилось бы постоянное движение в противоположных направлениях>> (см. Маркс <<Капитал>>, т.2, изд. 8-е, стр.307).
	
	
	
	Следовательно, Маркс вовсе не считал, что его теория воспроизводства имеет силу только лишь для капиталистического производства, хотя он и занимался исследованием законов капиталистического производства. Наоборот, он, как видно, исходил из того, что его теория воспроизводства может иметь силу и для социалистического производства.
	
	
	
	Следует отметить, что Маркс в <<Критике Готской программы>> при анализе экономики социализма и переходного периода к коммунизму исходит из основных положений своей теории воспроизводства, считая их очевидно обязательными для коммунистического строя.
	
	
	
	Следует также отметить, что Энгельс в своём <<Анти-Дюринге>>, критикуя <<социалистическую систему>> Дюринга и характеризуя экономику социалистического строя, также исходит из основных положений теории воспроизводства Маркса, считая их обязательными для коммунистического строя.
	
	
	
	Таковы факты.
	
	
	
	Выходит, что и здесь, в вопросе о воспроизводстве, т. Ярошенко, несмотря на его развязный тон в отношении <<схем>> Маркса, оказался вновь на мели.
	
	
	
	4. Своё письмо на имя членов Политбюро т. Ярошенко кончает предложением — поручить ему составить <<Политическую экономию социализма>>. Он пишет:
	
	
	
	<<Исходя из изложенного мною на пленарном заседании, секции и в настоящем письме определения предмета науки политической экономии социализма, используя марксистский диалектический метод, я могу в течение года, не более полтора года, при помощи двух человек, разработать теоретические решения основных вопросов политической экономии социализма; изложить марксистскую, ленинско-сталинскую теорию политической экономии социализма, теорию, которая превратит эту науку в действенное орудие борьбы народа за коммунизм>>.
	
	
	
	Нельзя не признать, что т. Ярошенко не страдает скромностью. Более того, пользуясь стилем некоторых литераторов, можно сказать: <<даже совсем наоборот>>.
	
	
	
	Выше уже говорилось, что т. Ярошенко смешивает политическую экономию социализма с хозяйственной политикой руководящих органов. То, что он считает предметом политической экономии социализма — рациональная организация производительных сил, планирование народного хозяйства, образование общественных фондов и т. д. — является не предметом политической экономии социализма, а предметом хозяйственной политики руководящих органов.
	
	
	
	Я уже не говорю о том, что серьёзные ошибки, допущенные т. Ярошенко, и его немарксистская <<точка зрения>> не располагает к тому, чтобы дать т. Ярошенко такое поручение.
	
	
	
	\subsubsection*{\centering{* * *}}
	
	
	
	Выводы:
	
	
	
	1) Жалоба т. Ярошенко на руководителей дискуссии лишена смысла, так как руководители дискуссии, будучи марксистами, не могли отразить в своих обобщающих документах немарксистскую <<точку зрения>> т. Ярошенко;
	
	
	
	2) просьбу т. Ярошенко поручить ему написать Политическую экономию социализма — нельзя считать серьёзной, хотя бы потому, что от неё разит хлестаковщиной.
	
	
	
	
	
	\begin{flushright}
		
		
		
		\textbf{И. СТАЛИН}
		
		
		
	\end{flushright}
	
	
	
	\noindent{\footnotesize 22 мая 1952 г.}
	
	
	
	\newpage
	
		\thispagestyle{empty} % выключаем отображение номера для этой страницы
	
	\ \\
	
	
	
	\ \\
	
	
	
	\ \\
	
	
	
	\ \\
	
	
	
	\ \\
	
	
	
	\section[ОТВЕТ ТОВАРИЩАМ САНИНОЙ А.В. и ВЕНЖЕРУ В.Г.]{\centering{ОТВЕТ ТОВАРИЩАМ САНИНОЙ А.В. и ВЕНЖЕРУ В.Г.}}
	
	
	
	Я получил ваши письма. Как видно, авторы этих писем глубоко и серьёзно изучают проблемы экономики нашей страны. В письмах имеется немало правильных формулировок и интересных соображений. Однако наряду с этим там имеются и некоторые серьёзные теоретические ошибки. В настоящем ответе я думаю остановиться на этих именно ошибках.
	
	
	
	\vspace{0.4cm}
	
	
	
	\subsection[1. Вопрос о характере экономических законов социализма]{\centering{1. ВОПРОС О ХАРАКТЕРЕ ЭКОНОМИЧЕСКИХ ЗАКОНОВ СОЦИАЛИЗМА.}}
	
	
	
	Т.т. Санина и Венжер утверждают, что <<только благодаря сознательному действию советских людей, занятых материальным производством, и возникают экономические законы социализма>>. Это положение совершенно неправильно.
	
	
	
	Существуют ли закономерности экономического развития объективно, вне нас, независимо от воли и сознания людей? Марксизм отвечает на этот вопрос положительно. Марксизм считает, что законы политической экономии социализма являются отражением в головах людей объективных закономерностей, существующих вне нас. Но формула т.т. Саниной и Венжера отвечает на этот вопрос отрицательно. Это значит, что эти товарищи становятся на точку зрения неправильной теории, утверждающей, что законы экономического развития при социализме <<создаются>>, <<преобразуются>> руководящими органами общества. Иначе говоря, они рвут с марксизмом и становятся на путь субъективного идеализма.
	
	
	
	Конечно, люди могут открыть эти объективные закономерности, познать их и, опираясь на них, использовать их в интересах общества. Но они не могут ни <<создавать>> их, ни <<преобразовывать>>.
	
	
	
	Допустим, что мы стали на минутку на точку зрения неправильной теории, отрицающей существование объективных закономерностей в экономической жизни при социализме и провозглашающей возможность <<создания>> экономических законов, <<преобразования>> экономических законов. К чему это привело бы? Это привело бы к тому, что мы попали бы в царство хаоса и случайностей, мы лишили бы себя возможности не то, что понять, а просто разобраться в этом хаосе случайностей.
	
	
	
	Это привело бы к тому, что мы ликвидировали бы политическую экономию как науку, ибо наука не может жить и развиваться без признания объективных закономерностей, без изучения этих закономерностей. Ликвидировав же науку, мы лишили бы себя возможности предвидеть ход событий в экономической жизни страны, то есть мы лишили бы себя возможности наладить хотя бы самое элементарное экономическое руководство.
	
	
	
	В конечном счёте мы оказались бы во власти произвола <<экономических>> авантюристов, готовых <<уничтожить>> законы экономического развития и <<создать>> новые законы без понимания и учёта объективных закономерностей.
	
	
	
	Всем известна классическая формулировка марксистской позиции по этому вопросу, данная Энгельсом в его <<Анти-Дюринге>>:
	
	
	
	<<Общественные силы, подобно силам природы, действуют слепо, насильственно, разрушительно, пока мы не познали их и не считаемся с ними. Но раз мы познали их, изучили их действие, направление и влияние, то только от нас самих зависит подчинять их всё более и более нашей воле и с помощью их достигать наших целей. Это в особенности относится к современным могучим производительным силам. Пока мы упорно отказываемся понимать их природу и характер, — а этому пониманию противятся капиталистический способ производства и его защитники, — до тех пор производительные силы действуют вопреки нам, против нас, до тех пор они властвуют над нами, как это подробно показано выше. Но раз понята их природа, они могут превратиться в руках ассоциированных производителей из демонических повелителей в покорных слуг. Здесь та же разница, что между разрушительной силой электричества в молниях грозы и укрощённом электричеством в телеграфном аппарате и дуговой лампе, та же разница, что между пожаром и огнём, действующим на службу человеку. Когда с современными производительными силами станут обращаться сообразно с их познанной, наконец, природой, общественная анархия в производстве заменится общественно-планомерным регулированием производства, рассчитанного на удовлетворение потребностей как целого общества, так и каждого его члена. Тогда капиталистический способ присвоения, при котором продукт порабощает сперва производителя, а затем и присвоителя, будет заменён новым способом присвоения продуктов, основанным на самой природе современных средств производства: с одной стороны, прямым общественным присвоением продуктов в качестве средств для поддержания и расширения производства, а с другой — прямым индивидуальным присвоением их в качестве средств к жизни и наслаждению>>.
	
	
	
	\vspace{0.4cm}
	
	
	
	\subsection[2. Вопрос о мерах повышения колхозной собственности до уровня общенародной собственности]{\centering{2. ВОПРОС О МЕРАХ ПОВЫШЕНИЯ КОЛХОЗНОЙ СОБСТВЕННОСТИ ДО УРОВНЯ ОБЩЕНАРОДНОЙ СОБСТВЕННОСТИ.}}
	
	
	
	Какие мероприятия необходимы для того, чтобы поднять колхозную собственность, которая является, конечно, не общенародной собственностью, до уровня общенародной (<<национальной>>) собственности?
	
	
	
	Некоторые товарищи думают, что необходимо просто национализировать колхозную собственность, объявив её общенародной собственностью, по примеру того, как это было сделано в своё время с капиталистической собственностью. Это предложение совершенно неправильно и безусловно неприемлемо. Колхозная собственность есть социалистическая собственность, и мы никак не можем обращаться с ней, как с капиталистической собственностью. Из того, что колхозная собственность является не общенародной собственностью, ни в коем случае не следует, что колхозная собственность не является социалистической собственностью.
	
	
	
	Эти товарищи полагают, что передача собственности отдельных лиц и групп в собственность государства является единственной или во всяком случае лучшей формой национализации. Это неверно. На самом деле передача в собственность государства является не единственной и даже не лучшей формой национализации, а первоначальной формой национализации, как правильно говорит об этом Энгельс в <<Анти-Дюринге>>. Безусловно, что пока существует государство, передача в собственность государства является наиболее понятной первоначальной формой национализации. Но государство будет существовать не на веки — вечные. С расширением сферы действия социализма в большинстве стран мира государство будет отмирать и, конечно, в связи с этим отпадёт вопрос о передаче имущества отдельных лиц и групп в собственность государству. Государство отомрёт, а общество останется. Следовательно, в качестве преемника общенародной собственности будет выступать уже не государство, которое отомрёт, а само общество в лице его центрального, руководящего экономического органа.
	
	
	
	Что же в таком случае нужно предпринять, чтобы поднять колхозную собственность до уровня общенародной собственности?
	
	
	
	В качестве основного мероприятия для такого повышения колхозной собственности т.т. Санина и Венжер предлагают: продать в собственность колхозам основные орудия производства, сосредоточенные в машинно-тракторных станциях, разгрузить таким образом государство от капитальных вложений в сельское хозяйство и добиться того, чтобы сами колхозы несли на себе ответственность за поддержание и развитие машинно-тракторных станций. Они говорят:
	
	
	
	<<Было бы неправильно полагать, что колхозные вложения должны будут главным образом направляться на нужды культуры колхозной деревни, а на нужды сельскохозяйственного производства по-прежнему основную массу вложений должно будет производить государство. А не верней ли будет освободить государство от этого бремени, ввиду полной способности колхозов принять это бремя всецело на себя. У государства найдётся немало дел для вложения своих средств в целях создания в стране изобилия предметов потребления>>.
	
	
	
	Для обоснования этого предложения авторы предложения выдвигают несколько доводов.
	
	
	
	Во-первых. Ссылаясь на слова Сталина о том, что средства производства не продаются даже колхозам, авторы предложения подвергают сомнению это положения Сталина, заявляя, что государство всё же продаёт средства производства колхозам, такие средства производства, как мелкий инвентарь, вроде кос и серпов, мелких двигателей и т. д. Они считают, что если государство продаёт колхозам эти средства производства, то оно могло бы продать и все другие средства производства вроде машин МТС.
	
	
	
	Этот довод несостоятелен. Конечно, государство продаёт колхозам мелкий инвентарь, как это полагается по Уставу сельскохозяйственной артели и по Конституции. Но можно ли ставить на одну доску мелкий инвентарь и такие основные средства производства в сельском хозяйстве, как машины МТС, или, скажем, земля, которая тоже ведь является одним из основных средств производства в сельском хозяйстве. Ясно, что нельзя. Нельзя, так как мелкий инвентарь ни в какой степени не решает судьбу колхозного производства, тогда как такие средства производства, как машины МТС и земля, всецело решают судьбу сельского хозяйства в наших современных условиях.
	
	
	
	Нетрудно понять, что когда Сталин говорил о том, что средства производства не продаются колхозам, он имел в виду не мелкий инвентарь, а основные средства сельскохозяйственного производства: машины МТС, земля. Авторы играют словами <<средства производства>> и смешивают две различные вещи, не замечая, что они попадают впросак.
	
	
	
	Во-вторых. Т.т. Санина и Венжер ссылаются далее на то, что в период массового колхозного движения — в конце 1929 и в начале 1930 годов ЦК ВКП(б) сам стоял за то, чтобы передать в собственность колхозам машинно-тракторные станции, требуя от колхозов погасить стоимость машинно-тракторных станций в течение трёх лет. Они считают, что хотя тогда это дело и провалилось <<ввиду бедности>> колхозов, но теперь, когда колхозы стали богатыми, можно было бы вернуться к этой политике — продаже колхозам МТС.
	
	
	
	Этот довод также несостоятелен. В ЦК ВКП(б) действительно было принято решение о продаже МТС колхозам в начале 1930 года. Решение это было принято по предложению группы ударников — колхозников в виде опыта, в виде пробы, с тем, чтобы в ближайшее время вернуться к этому вопросу и вновь его рассмотреть. Однако первая же проверка показала нецелесообразность этого решения, и через несколько месяцев, а именно в конце 1930 года решение было отменено.
	
	
	
	Дальнейший рост колхозного движения и развитие колхозного строительства окончательно убедили как колхозников, так и руководящих работников, что сосредоточение основных орудий сельскохозяйственного производства в руках государства, в руках машинно-тракторных станций, является единственным средством обеспечения высоких темпов роста колхозного производства.
	
	
	
	Мы все радуемся колоссальному росту сельскохозяйственного производства нашей страны, росту зернового производства, производства хлопка, льна, свёклы и т. д. Где источник этого роста? Источник этого роста в современной технике, в многочисленных современных машинах, обслуживающих все эти отрасли производства. Дело тут не только в технике вообще, а в том, что техника не может стоять на одном месте, она должна всё время совершенствоваться, что старая техника должна выводиться из строя и заменяться новой, а новая — новейшей. Без этого немыслим поступательный ход нашего социалистического земледелия, немыслимы ни большие урожаи, ни изобилие сельскохозяйственных продуктов. Но что значит вывести из строя сотни тысяч колёсных тракторов и заменить их гусеничными, заменить десятки тысяч устаревших комбайнов новыми, создать новые машины, скажем, для технических культур? Это значит нести миллиардные расходы, которые могут окупиться лишь через 6–8 лет. Могут ли поднять эти расходы наши колхозы, если даже они являются миллионерами? Нет, не могут, так как они не в состоянии принять на себя миллиардные расходы, которые могут окупиться лишь через 6–8 лет. Эти расходы может взять на себя только государство, ибо оно и только оно в состоянии принять на себя убытки от вывода из строя старых машин и замены их новыми, ибо оно и только оно в состоянии терпеть эти убытки в течение 6–8 лет с тем, чтобы по истечении этого срока возместить произведённые расходы.
	
	
	
	Что значит поле всего этого требовать продажи МТС в собственность колхозам? Это значит вогнать в большие убытки и разорить колхозы, подорвать механизацию сельского хозяйства, снизить темпы колхозного производства.
	
	
	
	Отсюда вывод: предлагая продажу МТС в собственность колхозам, т.т. Санина и Венжер делают шаг назад в сторону отсталости и пытаются повернуть назад колесо истории.
	
	
	
	Допустим на минутку, что мы приняли предложение т.т. Саниной и Венжера и стали продавать в собственность колхозам основные орудия производства, машинно-тракторные станции. Что из этого получилось бы?
	
	
	
	Из этого получилось бы, во-первых, что колхозы стали бы собственниками основных орудий производства, то есть они попали бы в исключительное положение, какого не имеет в нашей стране ни одно предприятие, ибо, как известно, даже национализированные предприятия не являются у нас собственниками орудий производства. Чем можно обосновать это исключительное положение колхозов, какими соображениями прогресса, продвижения вперёд? Можно ли сказать, что такое положение способствовало бы повышению колхозной собственности до уровня общенародной собственности, что оно ускорило бы переход нашего общества от социализма к коммунизму? Не вернее ли будет сказать, что такое положение могло бы лишить отдалить колхозную собственность от общенародной собственности и привело бы не к приближению к коммунизму, а наоборот, к удалению от него?
	
	
	
	Из этого получилось бы, во-вторых, расширение сферы действия товарного обращения, ибо колоссальное количество орудий сельскохозяйственного производства попало бы в орбиту товарного обращения. Как думают т.т. Санина и Венжер, может ли способствовать расширение сферы товарного обращения нашему продвижению к коммунизму? Не вернее ли будет сказать, что оно может лишь затормозить наше продвижение к коммунизму?
	
	
	
	Основная ошибка т.т. Саниной и Венжера состоит в том, что они не понимают роли и значения товарного обращения при социализме, не понимают, что товарное обращение несовместимо с перспективой перехода от социализма к коммунизму. Они, видимо, думают, что можно и при товарном обращении перейти от социализма к коммунизму, что товарное обращение не может помешать этому делу. Это глубокое заблуждение, возникшее на базе непонимания марксизма.
	
	
	
	Критикуя <<хозяйственную коммуну>> Дюринга, действующую в условиях товарного обращения, Энгельс в своём <<Анти-Дюринге>> убедительно доказал, что наличие товарного обращения неминуемо должно привести так называемые <<хозяйственные коммуны>> Дюринга к возрождению капитализма. Т.т. Санина и Венжер, видимо, не согласны с этим. Тем хуже для них. Ну, а мы, марксисты, исходим из известного марксистского положения о том, что переход от социализма к коммунизму и коммунистический принцип распределения продуктов по потребностям исключают всякий товарный обмен, следовательно превращение продуктов в товары, а вместе с тем и превращение их в стоимость. Так обстоит дело с предложением и доводами т.т. Саниной и Венжера.
	
	
	
	Что же в конце концов следует предпринять, чтобы повысить колхозную собственность до уровня общенародной собственности?
	
	
	
	Конечно есть предприятие необычное. Колхоз работает на земле и обрабатывает землю, которая давно уже является не колхозной, а общенародной собственностью. Следовательно, колхоз не является собственником обрабатываемой земли.
	
	
	
	Далее. Колхоз работает при помощи основных орудий производства, представляющих не колхозную, а общенародную собственность. Следовательно, колхоз не является собственником основных орудий производства.
	
	
	
	Далее. Колхоз — предприятие кооперативное, он пользуется трудом своих членов и распределяет доходы среди членов по трудодням, причём у колхоза есть свои семена, которые ежегодно возобновляются и идут в производство.
	
	
	
	Спрашивается: Чем же собственно владеет колхоз, где та колхозная собственность, которой он может распоряжаться вполне свободно, по собственному усмотрению? Такой собственностью является продукция колхоза, продукция колхозного производства: зерно, мясо, масло, овощи, хлопок, свёкла, лён и т. д., не считая построек и личного хозяйства колхозников на усадьбе. Дело в том, что значительная часть этой продукции, излишки колхозного производства поступают на рынок и включаются таким образом в систему товарного обращения. Это именно обстоятельство и мешает сейчас поднять колхозную собственность до уровня общенародной собственности. Поэтому именно с этого конца и нужно развернуть работу по повышению колхозной собственности до уровня общенародной. Чтобы поднять колхозную собственность до уровня общенародной собственности, нужно выключить излишки колхозного производства из системы товарного обращения и включить их в систему продуктообмена между государственной промышленностью и колхозами. В этом суть.
	
	
	
	У нас нет ещё развитой системы продуктообмена, но есть зачатки продуктообмена в виде <<отоваривания>> сельскохозяйственных продуктов. Как известно, продукция хлопководческих, льноводческих, свекловичных и других колхозов уже давно <<отоваривается>>, правда, <<отоваривается>> не полностью, частично, но всё же <<отоваривается>>. Заметим мимоходом, что слово <<отоваривание>> неудачное слово, его следовало бы заменить продуктообменом. Задача состоит в том, чтобы эти зачатки продуктообмена организовать во всех отраслях сельского хозяйства и развить их в широкую систему продуктообмена с тем, чтобы колхозы получали за свою продукцию не только деньги, а главным образом необходимые изделия. Такая система потребует громадного увеличения продукции, отпускаемой городом деревне, поэтому её придётся вводить без особой торопливости, по мере накопления городских изделий. Но вводить её нужно неуклонно, без колебаний, шаг за шагом сокращая сферу действия товарного обращения и расширяя сферу действия продуктообмена.
	
	
	
	Такая система, сокращая сферу действия товарного обращения, облегчит переход от социализма к коммунизму. Кроме того, она даст возможность включить основную собственность колхозов, продукцию колхозного производства в общую систему общенародного планирования.
	
	
	
	Это и будет реальным и решающим средством для повышения колхозной собственности до уровня общенародной собственности при наших современных условиях.
	
	
	
	Выгодна ли такая система для колхозного крестьянства? Безусловно выгодна. Выгодна, так как колхозное крестьянство будет получать от государства гораздо больше продукции и по более дешёвым ценам, чем при товарном обращении. Всем известно, что колхозы, имеющие с правительством договора о продуктообмене (<<отоваривания>>), получают несравненно больше выгод, чем колхозы, не имеющие таких договоров. Если систему продуктообмена распространить на все колхозы в стране, то эти выгоды станут достоянием всего нашего колхозного крестьянства.
	
	
	
	
	
	\begin{flushright}
		
		
		
		\textbf{И. СТАЛИН}
		
		
		
	\end{flushright}
	
	
	
	\noindent{{\footnotesize 1952 г. 28 сентября.}}
	
	
	
	
	
	
	
	\newpage
	
	
	
	\tableofcontents
	
	
	
	\thispagestyle{empty} %
	
	
	
	
	
	\newpage
	
	
	
	\setcounter{secnumdepth}{0}
	
	
	
	\phantomsection
	
	
	
	\begin{center}
		
		
		
		\ \\
		
		
		
		\ \\
		
		
		
		\ \\
		
		
		
		\ \\
		
		
		
		\ \\
		
		
		
		\ \\
		
		
		
		\ \\
		
		
		
		\ \\
		
		
		
		\ \\
		
		
		
		\ \\
		
		
		
		\ \\
		
		
		
		\ \\
		
		
		
		\ \\
		
		
		
		\ \\
		
		
		
		\ \\
		
		
		
		\ \\
		
		
		
		\ \\
		
		
		
		\ \\
		
		
		
		\ \\
		
		
		
		\ \\
		
		
		
		\ \\
		
		
		
		\ \\
		
		
		
		\ \\
		
		
		
		\ \\
		
		
		
		\ \\
		
		
		
		\ \\
		
		
		
		\ \\
		
		
		
		\ \\
		
		
		
		\ \\
		
		
		
		
		
		\parbox{17,2cm}{%
			
			
			
			\sloppy\setlength\parfillskip{0pt}
			
			
			
			Подписано в печать с матриц 25/Х 1952 г. Тираж 10 млн. 50 тыс. (2050001 — 3050000) экз.
			
			
			
			М-47290. Формат 82$\times$108$\rfrac{1}{32}$. 4,92 и. л. 1$\rfrac{1}{9}$ бум. л. 4,2 уч. — изд. л. Заказ № 409.
			
			
			
		}
		
		
		
		
		
		
		
		Цена на обложке 1 р., в ледериновом переплёте 3 р.
		
		
		
		
		
		\hrulefill
		
		
		
		2-я типография <<Печатный Двор>> имени А. М. Горького Главполиграфиздата при Совете Министров СССР. Ленинград, Гатчинская 26.
		
		
		
		
		
		
		
		
		
		
		
		
		
		
		
		
		
		
		
		
		
		
		
	\end{center}
	
	
	
	\thispagestyle{empty} % выключаем отображение номера для этой страницы
	
	
	
	\newpage
	
	
	
	\setcounter{secnumdepth}{0}
	
	
	
	\phantomsection
	
	
	
	\ \\
	\ \\
	\ \\
	\ \\
	
	
	
	\hangindent=2cm \hangafter=0 \textbf{ИНФОРМАЦИЯ ОБ ОЦИФРОВКЕ}
	
	
	
	\hangindent=2cm \hangafter=0 \textbf{Источник}: \href{https://rutracker.org/forum/viewtopic.php?t=5603632}{https://rutracker.org/forum/viewtopic.php?t=5603632}
	
	
	\hangindent=2cm \hangafter=0 \textbf{Версия}: 1.0 — от 28 января 2026
	
	
	\hangindent=2cm \hangafter=0 \textbf{Автор}: lord199 (\href{mailto: lord199@mail.ru}{lord199@mail.ru})
	
	
	\hangindent=2cm \hangafter=0 Если вы нашли какие-то опечатки и другие неточности, просьба связаться по электронному адресу.
	
	
	
	
	
	
	
	
	
	\thispagestyle{empty} % выключаем отображение номера для этой страницы
	
	
	
\end{document}






