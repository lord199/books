\documentclass[12pt, a4paper, openany]{book}
\usepackage[utf8]{inputenc}
\usepackage[russian]{babel}
\usepackage{graphicx}
\special{papersize=a5}
\usepackage[left=2.5cm,right=2.5cm,top=2cm,bottom=2cm,bindingoffset=0cm]{geometry}
\usepackage{indentfirst}
\usepackage{blindtext}
\usepackage[pagestyles]{titlesec}
\usepackage{fancyhdr}
\usepackage{lipsum}
\usepackage[usenames]{color}
\usepackage{colortbl}
\usepackage{xfrac} % Works better with other fonts
\usepackage{graphicx}
\usepackage{epigraph}
\usepackage{nicefrac} % For comparison
\usepackage{xfrac} % Works better with other fonts
\usepackage[colorlinks=true,linkcolor=black,urlcolor=black,bookmarksopen=true]{hyperref}
\usepackage[nottoc]{tocbibind}
\usepackage{ragged2e}
%\usepackage{times} 

\usepackage[open,openlevel=1]{bookmark}
\newcommand{\anonsection}[1]{ \section*{#1} \addcontentsline{toc}{section}{\numberline {}#1}} 
 
 \makeatletter %%%%% <---- Starting chapter without a pagebreak
\renewcommand\chapter{\par%
	\thispagestyle{plain}% \global\@topnum\z@
	\@afterindentfalse \secdef\@chapter\@schapter}
\makeatother %%%%% <---- Starting chapter without a pagebreak
\titleformat{\chapter}[display]
{\normalfont\bfseries}{}{0pt}{\Large}

\newpagestyle{mystyle}{
	\sethead[\thepage][][]{}{}{\thepage}	
}


\pagestyle{mystyle}



\addto\captionsrussian{% Replace "english" with the language you use
	\renewcommand{\contentsname}%
	{Содержание}%
}

% Настройка вертикальных и горизонтальных отступов
\titlespacing{\chapter}{0pt}{4pt}{4pt}
\titlespacing{\section}{\parindent}{1mm}{1mm}
\titlespacing{\subsection}{\parindent}{1mm}{1mm}

\begin{document}

%\maketitle
\begin{titlepage}

\fontsize{14pt}{14pt}\selectfont\centering{\textbf{АКАДЕМИЯ НАУК СССР}}

\vspace{0.3cm}

\fontsize{12pt}{12pt}\selectfont\centering{\textbf{ОРДЕНА ДРУЖБЫ НАРОДОВ \\ ИНСТИТУТ ЭТНОГРАФИИ ИМЕНИ Н. Н. МИКЛУХО-МАКЛАЯ}}

\vspace{0.3cm}

\fontsize{12pt}{12pt}\selectfont\centering{\textbf{ИНСТИТУТ ЛАТИНСКОЙ АМЕРИКИ}}

\vspace{0.8cm}

\fontsize{18pt}{18pt}\selectfont\centering{\textbf{И. Р. ГРИГУЛЕВИЧ}}

\vspace{0.8cm}
	
\fontsize{30pt}{30pt}\selectfont\centering{\textbf{\textcolor{red}{\textbf{ЭРНЕСТО}}}}

\fontsize{36pt}{36pt}\selectfont\centering{\textbf{\textcolor{red}{\textbf{ЧЕ ГЕВАРА}}}}

\fontsize{24pt}{24pt}\selectfont\centering{\textcolor{red}{И \\ РЕВОЛЮЦИОННЫЙ \\ ПРОЦЕСС \\ В ЛАТИНСКОЙ \\ АМЕРИКЕ}}

\vspace{0.8cm}
\fontsize{12pt}{18pt}\selectfont\centering{\textbf{Ответственный редактор \\ О. Т. ДАРУСЕНКОВ}}




\vspace{\fill}


	\fontsize{12pt}{14pt}\selectfont{\textbf{ИЗДАТЕЛЬСТВО <<НАУКА>>\\МОСКВА 1984}}

\newpage
\justify
Эрнесто Че Гевара — известный латиноамериканский революционер, видный участник Кубинской революции, погиб во время партизанских действий в горах Боливии в октябре 1967 г. О нем написано много книг на разных языках. В новой монографии прослеживается жизненный путь Че Гевары в тесной связи с развитием революционного процесса в странах Латинской Америки. Автор рисует своего героя такпм, каким он был в действительности: искренним, честным революционером, иногда заблуждавшимся, по всегда предельно преданным великим идеалам социализма и коммунизма, за которые отдал жизнь.

\vspace{1.8cm}
\centering{Рецензенты:

К. М. ОБЫДЕН, В. А. БОРОДАЕВ, Ю. П. ГАВРИКОВ}
		\thispagestyle{empty} % выключаем отображение номера для этой страницы
\end{titlepage}



\newpage

\setcounter{secnumdepth}{0} 

\phantomsection

\hangindent=5cm \hangafter=0  Он оставил нам свои революционные идеи, он оставил нам свои революционные достоинства, он оставил свой характер, свою волю, свою настойчивость, своё трудолюбие. Словом, он оставил нам свой пример!


\begin{flushright}
	ФИДЕЛЬ КАСТРО РУС
\end{flushright}

\vspace{0.8cm}

\section[Предисловие]{\centering{Предисловие}}


Революцию творят массы. Онп порождают и революционных вождей — ярких по своим способностям и поведению личностей, как правило отличающихся искренностью, мужеством, беспредельной преданностью делу, которому служат. Жизнь вождей насыщена всякого рода коллизиями, столкновениями, часто заканчивается трагически. Но грядущие поколения не забывают их имен, чтят их память, стремятся впитать в себя все лучшее, что было в них и в их деятельности.

История Латипской Америки выдвинула немало выдающихся борцов за народное дело. В колониальную эпоху это индейцы Кауполикан, Куаутемок, Тупак Амару. В XIX в. — Симон Боливар, Бернардо О'Хиггинс, Хосе Марти и десятки других героев борьбы за независимость. XX век навсегда занес в почетную книгу истории имена Панчо Вильи, Эмилиано Сапаты, Аугусто Сесара Сандино, Фиделя Кастро Рус, Эрнесто Че Гевары, Камило Сьенфуэгоса, Сальвадора Альенде и многих других.

Наша книга — об Эрнесто Че{\footnote{Че — характерное для аргентинцев междометие, выражавшее и удивление, и восторг, и печаль, и нежность, и одобрение, и протест, стало сначала прозвищем Эрнесто Гевары, а потом боевым псевдонимом, сросшимся с его именем и фамилией. После победы Кубинской революции, будучи президентом Национального банка, Гевара подписался на новых банкнотах Кубы «Че», вызвав возмущение контрреволюционеров. В ответ Эрнесто сказал: «Для меня Че означает самое важное, самое дорогое в моей жизни. Иначе и быть не могло. Ведь мои имя и фамилия — нечто маленькое, частное, незначительное».}} Геваре и его месте в Кубинской революции и в революционном процессе Латинской Америки. В. И. Ленин, которого оппортунисты обвиняли в том, что Советская Россия развивается не по привычным «марксистским» схемам, указывал, что революция — это не Невский проспект. Жизнь Эрнесто Че Гевары, развитие революционного процесса в Латинской Америке, история Кубинской революции еще раз подтверждают эту истину. Революции, хотя развиваются согласно одним и тем же закономерностям, идут обычно путями, которые трудно заранее предугадать как самим революционерам, так и их противникам. Если бы было иначе, то революция, вероятно, не могла бы победить, ведь враг, зная зараиее пути ее развития, сумел бы легко задушить революционный процесс в зародыше.

14 июня 1983 г. Эрнесто Че Геваре исполнилось бы 55 лет. Он родился 14 июня 1928 г. в аргентинском городе Росарио. Для тех, кто знал Эрнесто, трудно вообразить его ножилым, а тем более старым человеком. Всем своим обликом он олицетворял молодость — мужественную, бесстрашную, жаждущую героических свершений, подвигов, побед. Вместе с тем он терпеть не мог трескучих фраз, псевдоре-волюционной позы, самолюбования, рисовки. Против этого он боролся оружием смеха, иногда даже издевки. «Революционерами из кафе» презрительно называл он болтунов, рассуждающих о революции с позиций постороннего зрителя и критикана. Че Гевара считал настоящим революционером только того, кто принимал активное участие в революционной борьбе, в строительстве нового общества, однако в нем не было ничего от аскета, исключительной личности, героя, возвышающегося над толпой. Он не страдал комплексом отчужденности, ему были чужды и прочие комплексы, столь характерные для многих мелкобуржуазных интеллектуалов, вступающих на революционную стезю. А ведь Че тоже являлся одним из них. Врач, книжник, писатель, публицист, он был «стопроцентным интеллигентом», но в первую очередь он был коммунистом, и это определяло в нем все остальное.

Жизнь Че Гевары была яркой и драматичной. Аргентинец, он связал свою судьбу с Кубинской революцией, вместе с Фиделем Кастро участвовал в партизанских боях, стал крупнейшим специалпстом по вопросам герильи — партизанской войны, а после победы революции занимал высшие должности в партии и правительстве Республики Куба. В 1965 г. ушел со всех постов, а в 1966 г. оказался в Боливии, где во главе отряда добровольцев-партизан вел военные действия против реакционного режима и его союзников — империалистов США. 9 октября 1967 г. Че был убит американскими советниками, действовавшими в этой стране.

Гибель Эрпесто Че Гевары вызвала возмущение и негодование во всем мире. В то время популярность героя достигла апогея. Это породило беспокойство в Вашингтоне, где приложили немало усилий, чтобы исказить подлинный образ Че. Его пытались выставить супергероем-одиночкой, революционером-самоубийцей, выдавали за анархиста, троцкиста, последователя Мао Цзэдуна, как это делает, например, выполняя поручепие ЦРУ, Дапиэль Джеймс{\footnote{\textit{James D.} Che Guevara: A biography. L., 1970.}} в биографии Че. Этот автор, пытающийся всячески исказить и принизить образ Че в угоду тем, по приказу кого он был убит, с наивным притворством вопрошает в своей книге: «Почему столь широкий и глубокий ум, как Эрнесто Гевара, не обратился к опыту стран, где предпринимались пли по крайней мере намечались попытки предпринять другие, мирные решения социального вопроса? Если его ненависть к Соединенным Штатам исключала возможность объективного изучения американского общества, то почему не обратился он к опыту таких стран, как Швеция, где осуществлялись социальные эксперименты, более близкие его настроениям? Почему он оказался неспособным смотреть на вещи шире, не сквозь призму парализующей латиноамериканские страны монокультуры? Почему его ум в столь раннем возрасте исключил иные решения и иные ответы на извечные вопросы человечества?»{\footnote{Ibid., p. 123.}}. Автор воздерживается от ответа па эти патетические вопросы. Ведь ответ может быть только один: причину того, что Че избрал путь социальной революции, следует искать в политике порабощения и произвола, которую на протяжении десятилетий проводили в Западном полушарии империалисты США, Их монополии, банки, тресты захватили основные богатства стран Латинской Америки. Пентагон, госдепартамент, ЦРУ сделали нормой вмешательство в политическую жизнь этих государств. Правящие круги Соединенных Штатов боятся не только «коммунистической революции» в Латинской Америке, но и любой серьезной буржуазной реформы, если она задевает интересы их монополий, бьет по карману магнатов Уолл-стрита.

	
\newpage
\tableofcontents

\thispagestyle{empty} % 

\newpage

\setcounter{secnumdepth}{0}  

\phantomsection	

\ \\
\ \\
\ \\
\ \\

\hangindent=2cm \hangafter=0  	\textbf{ИНФОРМАЦИЯ ОБ ОЦИФРОВКЕ}

\hangindent=2cm \hangafter=0  	\textbf{Источник}:  \href{https://rutracker.org/forum/viewtopic.php?t=3908219}{https://rutracker.org/forum/viewtopic.php?t=3908219} 

\hangindent=2cm \hangafter=0  	\textbf{Версия}: 1.0 - от 16 ноября 2025

\hangindent=2cm \hangafter=0  	\textbf{Автор}: lord199 (\href{mailto:lord199@mail.ru}{lord199@mail.ru})

\hangindent=2cm \hangafter=0  Если вы нашли какие-то опечатки и другие неточности, просьба связаться по электронному адресу.




\thispagestyle{empty} % выключаем отображение номера для этой страницы

\end{document}
