\documentclass[12pt, a4paper, openany]{book}
\usepackage[utf8]{inputenc}
\usepackage[russian]{babel}
\usepackage{graphicx}
\special{papersize=a5}
\usepackage[left=2.5cm,right=2.5cm,top=2cm,bottom=2cm,bindingoffset=0cm]{geometry}
\usepackage{indentfirst}
\usepackage{blindtext}
\usepackage[pagestyles]{titlesec}
\usepackage{fancyhdr}
\usepackage{lipsum}
\usepackage[usenames]{color}
\usepackage{colortbl}
\usepackage{xfrac} % Works better with other fonts
\usepackage{graphicx}
\usepackage{epigraph}
\usepackage{nicefrac} % For comparison
\usepackage{xfrac} % Works better with other fonts
\usepackage[colorlinks=true,linkcolor=black,urlcolor=black,bookmarksopen=true]{hyperref}
\usepackage[nottoc]{tocbibind}
%\usepackage{times} 

\usepackage[open,openlevel=1]{bookmark}
\newcommand{\anonsection}[1]{ \section*{#1} \addcontentsline{toc}{section}{\numberline {}#1}} 
 
 \makeatletter %%%%% <---- Starting chapter without a pagebreak
\renewcommand\chapter{\par%
	\thispagestyle{plain}% \global\@topnum\z@
	\@afterindentfalse \secdef\@chapter\@schapter}
\makeatother %%%%% <---- Starting chapter without a pagebreak
\titleformat{\chapter}[display]
{\normalfont\bfseries}{}{0pt}{\Large}

\newpagestyle{mystyle}{
	\sethead[\thepage][][]{}{}{\thepage}	
}


\pagestyle{mystyle}



\addto\captionsrussian{% Replace "english" with the language you use
	\renewcommand{\contentsname}%
	{Содержание}%
}

% Настройка вертикальных и горизонтальных отступов
\titlespacing{\chapter}{0pt}{4pt}{4pt}
\titlespacing{\section}{\parindent}{1mm}{1mm}
\titlespacing{\subsection}{\parindent}{1mm}{1mm}

\begin{document}

%\maketitle
\begin{titlepage}
	
\fontsize{18pt}{18pt}\selectfont\centering{\textbf{\textcolor{red}{}}}

\fontsize{80pt}{80pt}\selectfont\centering{\textbf{И. СТАЛИН}}

\vspace{2cm}

\fontsize{60pt}{40pt}\selectfont\centering{\textbf{О ВЕЛИКОЙ ОТЕЧЕСТВЕННОЙ ВОЙНЕ СОВЕТСКОГО СОЮЗА}}
\vspace{3cm}


\fontsize{14pt}{16pt}\selectfont\centering{\textit{Издание шестое, стереотипное}} 


\vspace{\fill}


	\fontsize{12pt}{14pt}\selectfont{\textbf{ВОЕННОЕ ИЗДАТЕЛЬСТВО \\ МИНИСТЕРСТВА ВООРУЖЁННЫХ СИЛ СОЮЗА СССР \\ МОСКВА 1948}}



\end{titlepage}





\newpage
\tableofcontents

\thispagestyle{empty} % 

\newpage

\setcounter{secnumdepth}{0} 

\phantomsection



\section[ВЫСТУПЛЕНИЕ ПО РАДИО 3 июля 1941 года]{\centering{ВЫСТУПЛЕНИЕ ПО РАДИО \\ \textit{3 июля 1941 года}}}

\begin{center}
\textit{Товарищи! Граждане!}

\textit{Братья и сёстры!}

\textit{Бойцы нашей армии и флота!}
 
\end{center}

К вам обращаюсь я, друзья мои!

Вероломное военное нападение гитлеровской Германии на нашу Родину, начатое 22 июня, — продолжается. Несмотря на героическое сопротивление Красной Армии, несмотря на то, что лучшие дивизии врага и лучшие части его авиации уже разбиты и нашли себе могилу на полях сражения, враг продолжает лезть вперёд, бросая на фронт новые силы. Гитлеровским войскам удалось захватить Литву, значительную часть Латвии, западную часть Белоруссии, часть Западной Украины. Фашистская авиация расширяет районы действия своих бомбардировщиков, подвергая бомбардировкам Мурманск, Оршу, Могилёв, Смоленск, Киев, Одессу, Севастополь. Над нашей Родиной нависла серьёзная опасность.

Как могло случиться, что наша славная Красная Армия сдала фашистским войскам ряд наших городов и районов? Неужели немецко-фашистские войска в самом деле являются непобедимыми войсками, как об этом трубят неустанно фашистские хвастливые пропагандисты?

Конечно, нет! История показывает, что непобедимых армий нет и не бывало. Армию Наполеона считали непобедимой, но она была разбита попеременно русскими, английскими, немецкими войсками. Немецкую армию Вильгельма в период первой империалистической войны тоже считали непобедимой армией, но она несколько раз терпела поражения от русских и англо-французских войск и, наконец, была разбита англо-французскими войсками. То же самое нужно сказать о нынешней немецко-фашистской армии Гитлера. Эта армия не встречала ещё серьёзного сопротивления на континенте Европы. Только на нашей территории встретила она серьёзное сопротивление. И если в результате этого сопротивления лучшие дивизии немецко-фашистской армии оказались разбитыми нашей Красной Армией, то это значит, что гитлеровская фашистская армия так же может быть разбита и будет разбита, как были разбиты армии Наполеона и Вильгельма.

Что касается того, что часть нашей территории оказалась всё же захваченной немецко-фашистскими войсками, то это объясняется главным об разом тем, что война фашистской Германии против СССР началась при выгодных условиях для немецких войск и невыгодных для советских войск. Дело в том, что войска Германии, как страны, ведущей войну, были уже целиком отмобилизованы, и 170 дивизий, брошенных Германией против СССР и придвинутых к границам СССР, находились в состоянии полной готовности, ожидая лишь сигнала для выступления, тогда как советским войскам нужно было ещё отмобилизоваться и придвинуться к границам. Немалое значение имело здесь и то обстоятельство, что фашистская Германия неожиданно и вероломно нарушила пакт о ненападении, заключённый в 1939 году между ней и СССР, не считаясь с тем, что она будет признана всем миром стороной нападающей. Понятно, что наша миролюбивая страна, не желая брать на себя инициативу нарушения пакта, не могла стать на путь вероломства.

Могут спросить: как могло случиться, что Советское правительство пошло на заключение пакта о ненападении с такими вероломными людьми и извергами, как Гитлер и Риббентроп? Не была ли здесь допущена со стороны Советского правительства ошибка? Конечно, нет! Пакт о ненападении есть пакт о мире между двумя государствами. Именно такой пакт предложила нам Германия в 1939 году. Могло ли Советское правительство отказаться от такого предложения? Я думаю, что ни одно миролюбивое государство не может отказаться от мирного соглашения с соседней державой, если во главе этой державы стоят даже такие изверги и людоеды, как Гитлер и Риббентроп. И это, конечно, при одном непременном условии — если мирное соглашение не задевает ни прямо, ни косвенно территориальной целостности, независимости и чести миролюбивого государства. Как известно, пакт о ненападении между Германией и СССР является именно таким пактом.

Что выиграли мы, заключив с Германией пакт о ненападении? Мы обеспечили нашей стране мир в течение полутора годов и возможность подготовки своих сил для отпора, если фашистская Германия рискнула бы напасть на нашу страну вопреки пакту. Это определённый выигрыш для нас и проигрыш для фашистской Германии.

Что выиграла и что проиграла фашистская Германия, вероломно разорвав пакт и совершив нападение на СССР? Она добилась этим некоторого выигрышного положения для своих войск в течение короткого срока, но она проиграла политически, разоблачив себя в глазах всего мира, как кровавого агрессора. Не может быть сомнения, что этот непродолжительный военный выигрыш для Германии является лишь эпизодом, а громадный политический выигрыш для СССР является серьёзным и длительным фактором, на основе которого должны развернуться решительные военные успехи Красной Армии в войне с фашистской Германией.

Вот почему вся наша доблестная армия, весь наш доблестный военно-морской флот, все наши лётчики-соколы, все народы нашей страны, все лучшие люди Европы, Америки и Азии, наконец, все лучшие люди Германии — клеймят вероломные действия германских фашистов и сочувственно относятся к Советскому правительству, одобряют поведение Советского правительства и видят, что наше дело правое, что враг будет разбит, что мы должны победить.

В силу навязанной нам войны наша страна вступила в смертельную схватку со своим злейшим и коварным врагом — германским фашизмом. Наши войска героически сражаются с врагом, вооружённым до зубов танками и авиацией. Красная Армия и Красный Флот, преодолевая многочисленные трудности, самоотверженно бьются за каждую пядь советской земли. В бой вступают главные силы Красной Армии, вооружённые тысячами танков и самолётов. Храбрость воинов Красной Армии — беспримерна. Наш отпор врагу крепнет и растёт. Вместе с Красной Армией на защиту Родины подымается весь советский народ.

Что требуется для того, чтобы ликвидировать опасность, нависшую над нашей Родиной, и какие меры нужно принять для того, чтобы разгромить врага?



\end{document}
