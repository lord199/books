
%\documentclass[oneside,final,14pt]{extreport}
\documentclass[12pt, a4paper, openany]{book}
\usepackage[left=2cm,right=2cm,top=1.5cm,bottom=1.5cm,bindingoffset=0cm]{geometry}
%\usepackage[koi8-r]{inputenc}
\usepackage[russianb]{babel}
\usepackage{vmargin}
\setpapersize{A4}
\usepackage[T2A]{fontenc}
\usepackage[utf8x]{inputenc} % more recent versions (at least>=2004-17-10)
%\usepackage[russian]{babel}
%\setmarginsrb{2cm}{1.5cm}{1cm}{1.5cm}{0pt}{0mm}{0pt}{13mm}
\usepackage{indentfirst}
\usepackage{nicefrac} % For comparison
%\usepackage{xfrac} % Works better with other fonts
%\usepackage[unicode, pdftex]{hyperref}
\usepackage{lettrine}
\usepackage[usenames]{color}
%\special{papersize=a5}
\usepackage{colortbl}
\usepackage[pagestyles]{titlesec}
\usepackage{xfrac} % Works better with other fonts
\usepackage[colorlinks=true,linkcolor=black,urlcolor=black,bookmarksopen=true]{hyperref}


% Настройка вертикальных и горизонтальных отступов
\titlespacing{\chapter}{0pt}{5pt}{5pt}
\titlespacing{\section}{\parindent}{4mm}{4mm}
\titlespacing{\subsection}{\parindent}{3mm}{3mm}

\newcommand{\anonsection}[1]{ \section*{#1} \addcontentsline{toc}{section}{\numberline {}#1}} 

\makeatletter %%%%% <---- Starting chapter without a pagebreak
\renewcommand\chapter{\par%
	\thispagestyle{plain}% \global\@topnum\z@
	\@afterindentfalse \secdef\@chapter\@schapter}
\makeatother %%%%% <---- Starting chapter without a pagebreak
\titleformat{\chapter}[display]
{\normalfont\bfseries}{}{0pt}{\Large}

\newpagestyle{mystyle}{
	\sethead[\thepage][][]{}{}{\thepage}	
}

\pagestyle{mystyle}

\sloppy
\begin{document}

	
	\begin{titlepage}
		
		\begin{center}
			%\vfill
				{\Huge\bf МАТЕМАТИЧЕСКАЯ ЛОГИКА И ОСНОВАНИЯ МАТЕМАТИКИ\\}
			%\vfill
			\topskip0pt
			\vspace*{\fill}
			
			
			{\large\bf Ю.И. МАТИЯСЕВИЧ\\}
			\ \\
			\ \\
			{\Huge\bf ДЕСЯТАЯ ПРОБЛЕМА ГИЛЬБЕРТА\\}
			\ \\
			\ \\
			\ \\
			\vspace*{\fill}
			
			\vfill
			
				\begin{flushleft}
			
			Москва
		
			Издательская фирма
			
			<<Физико-математическая литература>>
			
			ВО <<Наука>>
			
			1\ 9\ 9\ 3
			
		\end{flushleft}
		\end{center}
		
	\end{titlepage}
	
	\thispagestyle{empty} % выключаем отображение номера для этой страницы
	
	\newpage
	

\setcounter{secnumdepth}{0}
	
	\section[ПРЕДИСЛОВИЕ]{\center \textbf{ПРЕДИСЛОВИЕ}}
	
На Втором Международном конгрессе математиков в Париже Давид Гильберт [1900] сделал свой знаменитый доклад «Математематические проблемы», содержащий 23 проблемы или, точнее, 23 группы родственных проблем, которые 19-й век оставлял в  наследие 20-му. Проблема под номером десять была посвящена  диофантовым уравнениям. 


	\begin{center}
	10. ЗАДАЧА О РАЗРЕШИМОСТИ ДИОФАНТОВА УРАВНЕНИЯ. 
\end{center}

\hangindent=0,6cm \hangafter=0 {\small Пусть задано диофантово уравнение с произвольными неизвестными и целыми рациональными числовыми коэффициентами.  Указать способ, при помощи которого возможно после конечного  числа операций установить, разрешимо ли это уравнение в целых  рациональных числах. }

\ \\

Под «способом», который предлагает найти Д. Гильберт, в настоящее время подразумевают «алгоритм». В начале века, когда проблемы формулировались, ещё не было математически строго общего понятия алгоритма. Отсутствие такого понятия  не могло само по себе служить препятствием к положительному решению 10-й проблемы Гильберта, поскольку про конкретные алгоритмы всегда было ясно, что они действительно дают требуемый общий способ решения соответствующих проблем. 

В 30-е годы в работах К. Гёделя, А. Чёрча, А. М. Тьюринга и других логиков было выработано строгое общее понятие алгоритма, которое дало принципиальную возможность устанавливать алгоритмическую неразрешимость, т. е. доказывать невозможность алгоритма с требуемыми свойствами. Тогда же были найдены первые примеры алгоритмически неразрешимых проблем, сначала в самой математической логике, а затем и в других разделах математики. 

Таким образом, теория алгоритмов создала необходимые предпосылки для попыток доказать неразрешимость 10-й проблемы Гильберта. Первые работы в этом направлении были опубликованы в начале 50-х годов, а в 1970 году исследования завершились «отрицательным решением» 10-й проблемы Гильберта. 

В случае 10-й проблемы Гильберта, как и в случае других проблем, долго ожидавших своего решения, не меньшее, а пожалуй, большее значение имеет математический аппарат, развитый для решения проблемы и находящий затем другие приложения, порой неожиданные. Основной технический результат, полученный при доказательстве неразрешимости 10-й проблемы Гильберта — это теорема о совпадении класса диофантовых множеств и класса перечислимых множеств. В качестве одного из следствий этой теоремы, формулировка которого не содержит специальных терминов, приведем следующее: можно явно указать полином от многих переменных с целыми коэффициентами такой, что множество всех его положительных значений, принимаемых при целочисленных значениях переменных, есть в точности множество всех простых чисел. 

Настоящая книга посвящена алгоритмической неразрешимости 10-й проблемы Гильберта и родственным вопросам; многочисленные частичные результаты, полученные в направлении положительного решения 10-й проблемы Гильберта, здесь почти не рассматриваются. 

Отрицательное решение 10-й проблемы Гильберта излагали (с разной степенью детализации) многие авторы, в частности: Азра [1971], Белл и Маховер [1977]. Гермес [1972. 1978], Девис [1973а, 1974]. Захаров [1970, 1986], Капланский [1977], Манин [1973, 1977], Маргенштерн [1981]. Матиясевич [1972а], Мияйлович. Маркович и Дошен [1986], Руохонен [1972, 1980], Саломаа [1985], Смориньский [1987], Сусман [1971]. Такахаши [1974], Фенстад [1971], Хавел [1973], Хиросе [1973]. 



	
	
	\newpage
	\tableofcontents
	
	\thispagestyle{empty} % 
	
	\newpage
	
	\setcounter{secnumdepth}{0}
	
	\phantomsection
	
		\section*{Описание}
	
	{\bf Название:} Десятая проблема Гильберта
	
{\bf Автор:} Юрий Владимирович Матиясевич
	
{\bf Издательство:} М.: Физмалит. 1993. - 224 с. - ISBN 5-02-014326-X
	
		{\bf Аннотация:} Дается полное доказательство алгоритмической неразрешимости 10-й проблемы Гильберта, касающейся диофантовых уравнений, вместе с необходимыми сведениями из теории алгоритмов и теории чисел, а также приложения развитой для этого техники к другим массовым проблемам теории чисел, алгебры, анализа, теоритического программирования.
		
		Для математиков, в том числе аспирантов и студентов старших курсов.
		
		Библиогр. 247 назв.
		
\ \\

		Р е ц е н з е н т
		
		Доктор физико-математических наук С.И. Адян
		\thispagestyle{empty} % выключаем отображение номера для этой страницы

	
\end{document}


