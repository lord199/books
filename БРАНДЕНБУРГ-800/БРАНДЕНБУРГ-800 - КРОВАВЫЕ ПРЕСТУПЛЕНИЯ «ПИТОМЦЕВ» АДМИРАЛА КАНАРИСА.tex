\documentclass[12pt, a4paper, openany]{book}
\usepackage[utf8]{inputenc}
\usepackage[russian]{babel}
\usepackage{graphicx}
\special{papersize=a5}
\usepackage[left=2.5cm,right=2.5cm,top=2cm,bottom=2cm,bindingoffset=0cm]{geometry}
\usepackage{indentfirst}
\usepackage{blindtext}
\usepackage[pagestyles]{titlesec}
\usepackage{fancyhdr}
\usepackage{lipsum}
\usepackage[usenames]{color}
\usepackage{colortbl}
\usepackage{xfrac} % Works better with other fonts
\usepackage{graphicx}
\usepackage{epigraph}
\usepackage{nicefrac} % For comparison
\usepackage{xfrac} % Works better with other fonts
\usepackage[colorlinks=true,linkcolor=black,urlcolor=black,bookmarksopen=true]{hyperref}
\usepackage[nottoc]{tocbibind}
%\usepackage{times} 

\usepackage[open,openlevel=1]{bookmark}
\newcommand{\anonsection}[1]{ \section*{#1} \addcontentsline{toc}{section}{\numberline {}#1}} 
 
 \makeatletter %%%%% <---- Starting chapter without a pagebreak
\renewcommand\chapter{\par%
	\thispagestyle{plain}% \global\@topnum\z@
	\@afterindentfalse \secdef\@chapter\@schapter}
\makeatother %%%%% <---- Starting chapter without a pagebreak
\titleformat{\chapter}[display]
{\normalfont\bfseries}{}{0pt}{\Large}

\newpagestyle{mystyle}{
	\sethead[\thepage][][]{}{}{\thepage}	
}


\pagestyle{mystyle}



\addto\captionsrussian{% Replace "english" with the language you use
	\renewcommand{\contentsname}%
	{Содержание}%
}



\begin{document}

%\maketitle
\begin{titlepage}

			\topskip0pt
\vspace*{\fill}

\fontsize{30pt}{30pt}\selectfont\centering{\textbf{«БРАНДЕНБУРГ-800»: КРОВАВЫЕ ПРЕСТУПЛЕНИЯ «ПИТОМЦЕВ» АДМИРАЛА КАНАРИСА}}
\vspace{3cm}



\vspace{\fill}




\end{titlepage}





\newpage
\tableofcontents

\thispagestyle{empty} % 

\newpage

\setcounter{secnumdepth}{0} 

\phantomsection



\section[Докладная записка начальника ОКР «Смерш» 61-й армии полковника В.И. Бударева начальнику УКР «Смерш» 1-го Белорусского фронта генерал-лейтенанту А.А. Вадису «О частях немецкого полка <<Бранденбург-800>> действующих перед фронтом 61-й армии». 14 июня 1944 г.]{\centering{Докладная записка начальника ОКР «Смерш» 61-й армии полковника В.И. Бударева начальнику УКР «Смерш» 1-го Белорусского фронта генерал-лейтенанту А.А. Вадису «О частях немецкого полка <<Бранденбург-800>> действующих перед фронтом 61-й армии». 14 июня 1944 г.}}

\begin{flushright}
	\underline{СОВЕРШЕННО СЕКРЕТНО}
\end{flushright}

\underline{ОТДЕЛ КОНТРРАЗВЕДКИ НКО <<СМЕРШ>> 61 АРМИИ 1 БЕЛОРУССКОГО ФРОНТА}

\begin{flushright}
	\underline{<<14>> июня 1944г.}
\end{flushright}

В процеесе изучения разведывательных и карательных органов противника дислоцирующихся перед фронтом 61 Армии, допросами военнопленных и данным полученным через партизансние отряды, были установлены батальоны принадлежащие немециому полку <<Бранденбург-800>>. Указанный полк проходит по ориентировке Главного Управления Контрразведки <<СМЕРШ>>  как имеющий специальное назначение по организации террора, диверсий и разложению войск противника и подчиняющийся непосредственно германскому разведоргану при ставке верховного командотания <<Абвер-II>>.

Исходя из этого в работе нед военнопленными германской армии нами большое внимание уделялось выявлению пленных принадлежащих подразделениям полка <<Брандербург-800>> с целью разоблачения среди них агентуры немецкой разведки, лиц чинивших злодеяния над Советскими гражданами, а такие получению развернутых данных о построении, структуре и назначении самого полка.

30 мая 1944 года на армейском сборном пункте военнопленных были установлены, а затем задержаны и арестованы, взятые в плен разведчики одатые в красноармейскую одежду:

фельдфебель 9 роты 3 батальона дивизии <<Бранденбург>>


\hangindent=5 cm \hangafter=0 \noindent КЕРСТГЕС Губерт, 1922 года рождения‚ уроженец и житель д. Бюлинген округ Аахен /Германия/, из крестьян, немец, беспартийный, образование 8 классов, холост

и оберефрейтор той-же роты -


\hangindent=5 cm \hangafter=0 \noindent КОХ Генрих Генрихович, 1917 года рождения, уроженец 6. Республики немцев-Поволжья, франкский кантон, д. Титель, по национальности немец, гражданин СССР, изменник Родины принявший германское подданство , по специальности шофер, беспартийный, образование 4 класса, холост.

Допросами указанных военнопленных устанавливается, что в конце 1940 года на базе оформированного перед нападением Германии на Бельгию и Голландию специального батальона при ставке германского верховного главнокомандования был создан полк особого назначения <<Брандербург-800>>.

В начале войны Германии против Советского Союза основные подразделения полка <<Брандербург-800>> были переброшены на Советско-германский фронт. Полк тогда состоял из 4 батальонов четырехротного состава и роты связи. В период наступления выполнял специальную задачу по захвату важных стратегических объектов /мостов, заводов, ж.д. станций/ и удерживал их до подхода основных немецких частей.

Для выполненич заданий отряжы полка переодевались в форму военнослужащих Красной Армии, частично вооружались советским оружием и действовали под видом частей Красной Армии. В ряде случаев солдаты забрасывались в расположение наших частей переодетыми в гражданскую одежду и снабжались проводниками-переводчиками хорошо владеющими русским языком.

В период отступления немцев, подразделения полка разрушали коммуникации и военные объекты, организовывали грабежи местного населения.

При переходе немцев к обороне подразделения полка участвуют в экспедициях против партизан, ведут разведывательную работу по похищению военнослужащих Красной Армии с переднего края обороны.



\end{document}
