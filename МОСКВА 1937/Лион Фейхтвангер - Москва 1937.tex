\documentclass[12pt, a4paper, openany]{book}
\usepackage[utf8]{inputenc}
\usepackage[russian]{babel}
\usepackage{graphicx}
\special{papersize=a5}
\usepackage[left=2.5cm,right=2.5cm,top=2cm,bottom=2cm,bindingoffset=0cm]{geometry}
\usepackage{indentfirst}
\usepackage{blindtext}
\usepackage[pagestyles]{titlesec}
\usepackage{fancyhdr}
\usepackage{lipsum}
\usepackage[usenames]{color}
\usepackage{colortbl}
\usepackage{xfrac} % Works better with other fonts
\usepackage{graphicx}
\usepackage{epigraph}
\usepackage{nicefrac} % For comparison
\usepackage{xfrac} % Works better with other fonts
\usepackage[colorlinks=true,linkcolor=black,urlcolor=black,bookmarksopen=true]{hyperref}
\usepackage[nottoc]{tocbibind}
%\usepackage{times} 

\usepackage[open,openlevel=1]{bookmark}
\newcommand{\anonsection}[1]{ \section*{#1} \addcontentsline{toc}{section}{\numberline {}#1}} 
 
 \makeatletter %%%%% <---- Starting chapter without a pagebreak
\renewcommand\chapter{\par%
	\thispagestyle{plain}% \global\@topnum\z@
	\@afterindentfalse \secdef\@chapter\@schapter}
\makeatother %%%%% <---- Starting chapter without a pagebreak
\titleformat{\chapter}[display]
{\normalfont\bfseries}{}{0pt}{\Large}

\newpagestyle{mystyle}{
	\sethead[\thepage][][]{}{}{\thepage}	
}


\pagestyle{mystyle}



\addto\captionsrussian{% Replace "english" with the language you use
	\renewcommand{\contentsname}%
	{Содержание}%
}

% Настройка вертикальных и горизонтальных отступов
\titlespacing{\chapter}{0pt}{4pt}{4pt}
\titlespacing{\section}{\parindent}{1mm}{1mm}
\titlespacing{\subsection}{\parindent}{1mm}{1mm}

\begin{document}

%\maketitle
\begin{titlepage}
	
\fontsize{18pt}{18pt}\selectfont\centering{\textbf{\textcolor{red}{}}}

\vspace{5cm}
\fontsize{30pt}{30pt}\selectfont\centering{\textbf{Лион Фейхтвангер}}

\vspace{3cm}

\fontsize{24pt}{30pt}\selectfont\centering{\textbf{МОСКВА \\ 1937}}
\vspace{3cm}

%\fontsize{14pt}{16pt}\selectfont\centering{Стенограмма публичной лекции\\тов. С. Е. БЕЛИНКОВА, прочитанной\\14 августа 1945 года в Лекционном\\зале в Москве} 


\vspace{\fill}


	\fontsize{12pt}{14pt}\selectfont{\textbf{ГОСПОЛИТИЗДАТ\\1937}}

\newpage

\thispagestyle{empty}

\fontsize{18pt}{18pt}\selectfont\centering{\textbf{\textcolor{red}{}}}

\fontsize{18pt}{18pt}\selectfont\centering{ЛИОН ФЕЙХТВАНГЕР}

\vspace{2cm}

\fontsize{24pt}{30pt}\selectfont\centering{\textbf{МОСКВА 1937}}

\vspace{1cm}

\fontsize{20pt}{26pt}\selectfont\centering{\textbf{ОТЧЕТ О ПОЕЗДКЕ}}

\fontsize{18pt}{22pt}\selectfont\centering{ДЛЯ МОИХ ДРУЗЕЙ}
\vspace{1cm}

\fontsize{12pt}{20pt}\selectfont\centering{ПЕРЕВОД С НЕМЕЦКОГО}

\vspace{\fill}


\fontsize{12pt}{14pt}\selectfont{ГОСУДАРСТВЕННОЕ ИЗДАТЕЛЬСТВО \\ <<ХУДОЖЕСТВЕННАЯ ЛИТЕРАТУРА>>}

\fontsize{10pt}{14pt}\selectfont{МОСКВА \ \ 1937}

\vspace{1cm}

\end{titlepage}





\newpage
\tableofcontents

\thispagestyle{empty} % 

\newpage

\setcounter{secnumdepth}{0} 

\phantomsection

\section[От издательства]{ОТ ИЗДАТЕЛЬСТВА}


Изданная в Амстердаме на немецком языке книжка Лиона Фейхтвангера <<Москва 1937>>, в которой автор, на основе личных впечатлений и наблюдений от поездки в СССР, дает оценку современного положения СССР, его политической, хозяйственной и культурной жизни, представляет несомненный интерес. Книжка содержит ряд ошибок и неправильных оценок. В этих ошибках легко может разобраться советский читатель. Тем не менее книжка представляет интерес и значение, как попытка честно и добросовестно изучить Советский Союз.

Фейхтвангер принадлежит к числу тех немногих некоммунистических писателей на Западе, которые не боятся правды, не сложили оружия перед фашизмом, а продолжают борьбу с ним. В то время, когда буржуазные разбойники пера, в угоду капитализму и фашизму, состязаются в фабрикации отравленной лжи и клеветы против СССР, Фейхтвангер старается доискаться объективной правды об СССР и понять его особенности.

\newpage

\section[Предисловие]{ПРЕДИСЛОВИЕ}

\subsection*{Цель этой книги}

Эти страницы следовало бы, собственно, озаглавить <<Москва, январь, 1937 год>>. Ведь жизнь в Москве течет с такой быстротой, что некоторые утверждения становятся спустя несколько месяцев уже неправильными. Я бродил по Москве с людьми, хорошо ее знающими; пробыв в отсутствии каких-нибудь полгода, они теперь, глядя на нее, покачивали головой: неужели это наш город? Несмотря на это, я все же даю этой книге заглавие <<Москва, 1937 год>>. Я позволю себе такую неопределенность в дате, потому что я не стремлюсь к точной объективной передаче виденного мною; после десятинедельного пребывания такая попытка была бы нелепа. Я хочу только изложить свои личные впечатления для друзей, жадно набрасывающихся на меня с вопросами: <<Ну, что Вы думаете о Москве? Что Вы там, в Москве, видали?>>


\subsection*{Несколько неправильно нарисованная мною картина}
Так как я сознаю, что предлагаемые мною суждения субъективны, я хочу рассказать о том, с какими ожиданиями и опасениями я ехал в Советский Союз. Пусть каждый читатель сам установит, насколько мой взгляд был затемнен предвзятыми мнениями и чувствами.


\subsection*{Вера в разум}

Я пустился в путь в качестве <<симпатизирующего>>. Да, я симпатизировал с самого начала эксперименту, поставившему себе целью построить гигантское государство только на базисе разума, и ехал в Москву с желанием, чтобы этот эксперимент был удачным. Как бы мало я ни был склонен исключать из частной жизни человека его логическое, нелогическое и чувства, как бы я ни находил жизнь, построенную на одной чистой логике, однообразной и скучной, все же я глубоко убежден в том, что общественная организация, если она хочет развиваться и процветать, должна строиться на основах разума и здравых суждений. Мы с содроганием видели на примере Центральной Европы, что получается, когда фундаментом государства и законов хотят сделать не разум, а чувства и предрассудки. Мировая история мне всегда представлялась великой длительной борьбой, которую ведет разумное меньшинство с большинством глупцов. В этой борьбе я стал на сторону разума, и потому я симпатизировал великому опыту, предпринятому Москвой, с самого его возникновения.

\subsection*{Недоверие и сомнение}
Однако с самого начала к моим симпатиям примешивались сомнения. Практический социализм мог быть построен только посредством диктатуры класса, и Советский Союз был в самом деле государством диктатуры. Но я писатель, писатель по призванию, а это означает, что я испытываю страстную потребность свободно выражать все, что я чувствую, думаю, вижу, переживаю, невзирая на лица, на классы, партии и идеологии, и поэтому при всей моей симпатии я все же чувствовал недоверие к Москве. Правда, Советский Союз выработал демократическую, свободную конституцию; но люди, заслуживающие доверия, говорили мне, что эта свобода на практике имеет весьма растрепанный и исковерканный вид, а вышедшая перед самым моим отъездом небольшая книга Андре Жида только укрепила мои сомнения.

\subsection*{Потемкинские деревни}
Итак,  к границам  Советского Союза  я  подъезжал полный любопытства, сомнений и  симпатий.  Почетная встреча, оказанная мне в Москве, увеличила  мою неуверенность.  Мои  хорошие  знакомые,  люди  обычно вполне  разумные, совершенно  теряли здравый  ум,  когда  оказывались  среди немецких фашистов, осыпавших их почестями, и я спрашивал  себя, неужели и  я позволю тщеславию изменить мой взгляд на вещи и людей. Кроме того, я говорил себе, что мне, несомненно, будут показывать  только положительное и что мне, человеку, не знакомому  с языком, трудно будет разглядеть то, что скрыто под прикрашенной внешностью.

\subsection*{Нападки, вы званные  недостатком  комфорта}
     
С другой стороны, множество мелких неудобств, осложняющих повседневный московский быт  и мешающих видеть важное,  легко   могло   привести  человека  к  несправедливому  и   слишком отрицательному  суждению.  Я  очень  скоро понял,  что причиной неправильной оценки, данной Москве великим писателем Андре Жидом, были именно такого рода мелкие неприятности. По этому в Москве я приложил много усилий к тому, чтобы неустанно контролировать  свои взгляды и выправлять их то в  ту, то в другую сторону  с  тем,  чтобы  приятные  или  неприятные  впечатления  момента  не оказывали влияния на мое окончательное суждение.


\subsection*{Дальнейшие  трудности на пути к правильному суждению}

Иногда же наивная гордость и  усердие советских  людей  мешали мне найти  правильное  решение. Цивилизация   Советского   Союза  совсем   молода.   Она   достигнута  ценой беспримерных  трудностей и лишений,  поэтому, когда  к  москвичам  приезжает гость, мнением  которого - справедливо или несправедливо  - они дорожат, они немедленно начинают забрасывать его вопросами:  как  Вам нравится то, что Вы скажете по поводу этого? Кроме того, я попал в Москву  в  неспокойное время. Фашистские  вожди вели угрожающие  речи на тему  о  войне против  Советского Союза;  в  Испании  и  на  границах Монголии  шла борьба; в  Москве слушался политический  процесс, сильно взволновавший  массы.  Следовательно, вопросов накопилось  немало,  и   москвичи  на   них  не  скупились.  Я  же,  человек медлительный в  своих оценках, люблю мысленно обсудить все "за" и "против" и не  тороплюсь   выражать  свое  мнение,  если  не   считаю   его  достаточно продуманным. Вполне естественно, что не все в Москве мне понравилось, а  мое писательское честолюбие требует от меня откровенного выражения моего  мнения - склонность, причинившая мне немало неудобств. Итак, я, будучи в  Советском Союзе, не хотел умалчивать  о недостатках, где-либо замеченных мною.  Однако найти этим неблагоприятным отзывам нужную форму  и слова, которые, не будучи бестактными, имели бы достаточно определенный смысл, представляло  не всегда легкую задачу для почетного гостя в такое напряженное время.

\subsection*{Откровенность за откровенность}

Я мог с удовлетворением констатировать, что  моя откровенность  в  Москве  не  вызвала  обиды. Газеты  помещали  мои замечания на видном  месте, хотя, возможно,  правящим лицам  они не особенно нравились. В этих заметках я  высказывался за большую терпимость в некоторых областях,   выражал   свое  недоумение   по   поводу   иной  раз   безвкусно преувеличенного культа Сталина  и говорил насчет  того, что  следовало  бы с большей  ясностью  раскрыть,  какими  мотивами  руководствовались обвиняемые второго троцкистского процесса, признаваясь в содеянном. И в частных беседах руководители  страны  относились  к  моей  критике с  вниманием  и  отвечали откровенностью  на откровенность. Именно  потому, что свое мнение я  выражал неприкрыто,  я  получил  сведения, которые  в противном случае мне  едва  ли удалось бы получить.

\subsection*{Нужно  ли  вы ступать с положительной оценкой Советского  Союза?}

После моего возвращения на Запад передо мной встал вопрос, должен ли я  говорить о том, что я видел в Советском Союзе? Это не являлось бы проблемой, если бы я, как   другие,  увидел   в  Советском  Союзе  много  отрицательного  и   мало положительного. Мое выступление встретили бы с ликованием. Но я  заметил там больше света, чем тени, а Советский Союз не любят и слышать хорошее о нем не хотят. Мне тотчас же было на это указано. Я не очень часто выступал в печати Советского  Союза со своими впечатлениями. Мои выступления  составили  менее двухсот  строк, при этом они отнюдь  не заключали  в себе только похвалу; но даже это немногое было здесь, на Западе, ввиду того, что оно не представляло безоговорочного отрицания, искажено и  опошлено. Должен ли я был  продолжать говорить о Советском Союзе?

\subsection*{Лучше  не надо}

Усталый и  возбужденный виденным и слышанным, я  сказал себе в  первые дни  после моего возвращения, что моя  задача не говорить,  а изображать  в  образах,  и  я  решил  молчать  и  ждать,  пока  пережитое не воплотится в образы, которые можно запечатлеть.

\subsection*{Но как писатель я все  же это  делаю}

Однако вскоре  другие соображения одержали верх. Советский Союз ведет борьбу с многими врагами, и его союзники оказывают  ему  только  слабую  поддержку.  Тупость,  злая  воля  и косность стремятся к  тому,  чтобы опорочить, оклеветать, отрицать все  плодотворное, возникающее на Востоке. Но писатель, увидевший великое, не  смеет уклоняться от  дачи  свидетельских показаний, если даже  это великое непопулярно  и его слова будут многим неприятны.

Поэтому я и свидетельствую.


\end{document}
