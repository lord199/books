
%\documentclass[oneside,final,14pt]{extreport}
\documentclass[12pt, a4paper, openany]{book}
\usepackage[left=2.5cm,right=2.5cm,top=2cm,bottom=2cm,bindingoffset=0cm]{geometry}
%\usepackage[koi8-r]{inputenc}
\usepackage[russianb]{babel}
\usepackage{vmargin}
\setpapersize{A4}
\usepackage[T2A]{fontenc}
\usepackage[utf8x]{inputenc}  % more recent versions (at least>=2004-17-10)
%\usepackage[russian]{babel}
%\setmarginsrb{2cm}{1.5cm}{1cm}{1.5cm}{0pt}{0mm}{0pt}{13mm}
\usepackage{indentfirst}
\usepackage{nicefrac} % For comparison
%\usepackage{xfrac}    % Works better with other fonts
%\usepackage[unicode, pdftex]{hyperref}
\usepackage{lettrine}
\usepackage[usenames]{color}
\special{papersize=a5}
\usepackage{colortbl}
\usepackage[pagestyles]{titlesec}
\usepackage{xfrac}    % Works better with other fonts
\usepackage[colorlinks=true,linkcolor=black,urlcolor=black,bookmarksopen=true]{hyperref}


% Настройка вертикальных и горизонтальных отступов
\titlespacing{\chapter}{0pt}{5pt}{5pt}
\titlespacing{\section}{\parindent}{4mm}{4mm}
\titlespacing{\subsection}{\parindent}{3mm}{3mm}

\newcommand{\anonsection}[1]{ \section*{#1} \addcontentsline{toc}{section}{\numberline {}#1}} 

\makeatletter %%%%% <---- Starting chapter without a pagebreak
\renewcommand\chapter{\par%
	\thispagestyle{plain}% \global\@topnum\z@
	\@afterindentfalse \secdef\@chapter\@schapter}
\makeatother %%%%% <---- Starting chapter without a pagebreak
\titleformat{\chapter}[display]
{\normalfont\bfseries}{}{0pt}{\Large}

\newpagestyle{mystyle}{
	\sethead[\thepage][][]{}{}{\thepage}	
}

\pagestyle{mystyle}

\sloppy
\begin{document}

	
	\begin{titlepage}
		
		\begin{center}
			%\vfill
			
			%\vfill
			\topskip0pt
			\vspace*{\fill}
			
			
			{\large\bf Б. М. ХАЛОША\\}
			\ \\
			\ \\
			{\Huge\bf НАТО И АТОМ\\}
			\ \\
			\ \\
			\ \\
			(Ядерная политика Североатлантического блока) 
			\vspace*{\fill}    
			
			\vfill
			
				\begin{flushright}
			
			Издательство
		
			«Знание»
		
			Москва 1975
			
		\end{flushright}
		\end{center}
		
	\end{titlepage}
	
	\thispagestyle{empty} % выключаем отображение номера для этой страницы
	
	\newpage
	

\setcounter{secnumdepth}{0}  
	
	\section[Введение]{\center ВВЕДЕНИЕ}
	
Начало 70-х годов нынешнего столетия войдет в историю международных отношений как время глубоких  политических перемен на мировой арене.

Благодаря настойчивой борьбе Советского Союза, других социалистических государств, всех миролюбивых сил, а  также реализму, «проявленному в политике ряда   капиталистических стран, достигнуты серьезные сдвиги в сторону разрядки напряженности на международной арене. Совершается исторический поворот от «холодной войны» и опасной напряженности к совместным усилиям по укреплению мира и развитию взаимовыгодного сотрудничества.

Вся история советской внешней политики от Декрета о мире, принятого в 1917 году, и до Программы мира, выдвинутой XXIV съездом КПСС, и последующего успешного осуществления этой Программы свидетельствует о настойчивой и целеустремленной борьбе Советского Союза за мир, за широкое внедрение в практику международных отношений ленинских принципов мирного сосуществования.

В то же время ход мировых событий убедительно показывает, что на пути движения человечества к прочному миру и сотрудничеству между народами существуют еще немалые препятствия. Силы империализма, реакции и войны не   сложили оружие. Они упорно противятся процессу оздоровления международной обстановки, пытаются возродить времена «холодной войны» и опасной напряженности в мире. «Мы реалисты, — говорил Л. И. Брежнев в докладе «О   пятидесятилетии Союза Советских Социалистических республик», — и хорошо видим, что влиятельные круги в мире империализма все еще не отказались от попыток проводить политику «с позиции силы». Все еще продолжается развязанная ими гонка вооружений, создающая угрозу миру. Мы и наши союзники, естественно, не можем не делать из этого необходимых выводов. Но миролюбивый курс нашей внешней политики   неизменен, и в современной обстановке возможности миролюбивых сил в их борьбе против сил агрессии и войны велики как никогда»{\footnote{Л. И. Брежнев. О пятидесятилетии Союза Советских Социалистических Республик. М., Политиздат, 1972, стр. 42.}}.

Как известно, одним из главных орудий подрывной деятельности агрессивных кругов империализма против мира и безопасности народов был и остается Североатлантический блок (НАТО){\footnote{НАТО — сокращенное название, происходящее от начальных букв официального английского наименования этого военного блока —  «Организация Североатлантического договора». (North Atlantic Treaty Organization—NATO).}}.

Напомним, что договор о создании этой замкнутой военно-политической группировки западных держав был подписан 4 апреля 1949 года в Вашингтоне. Первоначально в НАТО вошли Соединенные Штаты Америки, Великобритания,   Франция, Бельгия, Голландия, Люксембург, Канада, Италия, Португалия, Норвегия, Дания и Исландия. В феврале 1952 года в Североатлантический союз были включены Турция и Греция, а в мае 1955 года — Федеративная Республика Германии.

НАТО занимает центральное место среди империалистических военных союзов, созданных для борьбы с мировой социалистической системой, национально-освободительным, рабочим и демократическим движением.

За двадцать пять лет существования НАТО наиболее агрессивные милитаристские круги этого блока использовали и используют его для того, чтобы превратить Западную Европу в вооруженный лагерь и военный плацдарм. Через систему НАТО они добивались и добиваются усиления гонки  вооружений, увеличения вооруженных сил в западноевропейских странах, раздувания ими военных бюджетов, милитаризации их экономики.

И сейчас влиятельные круги в странах — членах НАТО всячески стремятся помешать положительным изменениям, происходящим на международной арене, навязать подход к решению важнейших политических проблем современности с обанкротившейся «позиции силы». Процесс разрядки, все более развивающийся в Европе, тревожит представителей военно-промышленного комплекса, милитаристские круги НАТО, которые усматривают в нем угрозу самому   существованию этой военной группировки. Вооруженная интервенция НАТО против независимой суверенной Республики Кипр, предпринятая летом 1974 г., — новое свидетельство   агрессивного характера этой военной группировки.

Особенно опасный характер для дела мира и безопасности народов имели и имеют разрабатываемые и осуществляемые в НАТО вот уже на протяжении многих лет различные варианты приобщения участников этой военной группировки к ядерному оружию и стратегии.

Вопрос о создании в рамках НАТО так называемых «многосторонних ядерных сил» (МЯС) «неизменно фигурировал на многочисленных совещаниях и встречах ответственных государственных и военных деятелей стран НАТО, состоявшихся в 60-е годы.

Создание ядерных сил НАТО явилось бы серьезной угрозой для человечества. Оно (неизмеримо увеличило бы опасность возникновения мировой термоядерной войны, послужило бы стимулом к еще большему усилению гонки ракетно-ядерного вооружения и шагом на пути широкого распространения атомного оружия среди империалистических держав. Ядерное оружие с его неизбежным спутником — военной опасностью расползлось бы по нашей планете.

Планы создания объединенных ядерных сил НАТО представляли особую опасность для дела мира во всем мире, потому что в случае их реализации к спусковым кнопкам ядерных сил НАТО оказались бы в конечном итоге допущенными милитаристские, реваншистские круги ФРГ — единственной страны в Европе, которая при правительствах Аденауэра — Эрхарда — Кизингера открыто заявляла о своих территориальных притязаниях к своим соседям и намерениях пересмотреть итоги второй мировой войны.

Планы создания МЯС встретили самый решительный и резкий отпор со стороны Советского Союза и других социалистических стран, что явилось важнейшим фактором, воспрепятствовавшим реализации этих опасных замыслов. Создание «многосторонних ядерных сил» НАТО натолкнулось на решительное сопротивление международной общественности и серьезные разногласия между самими атлантическими партнерами. В этих условиях Соединенные Штаты Америки вынуждены были отказаться от планов создания «многосторонних ядерных сил» НАТО.

Срыв планов создания ядерных сил НАТО привел к тому, что наиболее милитаристские круги европейских участников этого блока были вынуждены ограничить свои ядерные притязания участием в работе постоянных ядерных органов НАТО (Комитет по вопросам ядерной обороны, Группа ядерного планирования), разрабатывающих ядерную стратегию, но не обладающих контролем над самим ядерным оружием.

Это, в частности, создало необходимые условия для подписания Договора о нераспространении ядерного оружия, вступившего в силу 5 марта 1970 года, и явившегося большой победой внешней политики Советского Союза, других социалистических стран, всех миролюбивых сил. Поэтому хотя события, связанные с разработкой планов создания МЯС, и оказались «а недавно перевернутой странице истории, — они тем не менее теснейшим образом связаны с международными событиями сегодняшнего дня.

Отказ США от создания «многосторонних ядерных сил» НАТО явился одним из убедительных примеров того, что империализм, учитывая мощь Советского Союза, вынужден идти на уступки и заключение соглашений, способствующих разрядке международной напряженности. Гонка ракетно-ядерных вооружений, как показывает практика, не в состоянии обеспечить Соединенным Штатам военного и политического превосходства. Это побуждает реалистически мыслящие круги США становиться на путь заключения соглашений с СССР об их ограничении на основе признания принципа равной одинаковой безопасности сторон. Свидетельством этого являются заключенные в результате первой советско-американской встречи на высшем уровне в Москве в мае 1972 года Договор об ограничении систем противоракетной обороны и Временное соглашение о некоторых мерах в области ограничения стратегических наступательных вооружений. Эти соглашения проложили путь к дальнейшим мерам по ограничению гонки вооружений и уменьшению опасности ядерной войны. Важное значение в этом отношении имеют подписанные между СССР и США в июне 1973 года во время визита Генерального секретаря ЦК КПСС Л. И. Брежнева в США «Основные принципы переговоров о дальнейшем ограничении стратегических наступательных вооружений». Исключительное значение в этом отношении имеет и подписанное во время этого визита Соглашение между СССР и США о предотвращении ядерной войны.

На состоявшейся с 27 «июня иго 3 июля 1974 года в Советском Союзе третьей советско-американской встрече на высшем уровне первостепенное место занимали проблемы дальнейшего уменьшения опасности войны и сдерживания гонки вооружений. Участники встречи достигли договоренности об ограничении систем противоракетной обороны двух стран, о согласованном ограничении подземных испытаний ядерного оружия, о дальнейших усилиях, направленных на ограничение стратегических наступательных вооружений, о принятии мер, нацеленных на исключение химического оружия из арсеналов государств.

Несомненно, провал планов создания объединенных ядерных сил НАТО, предусматривавших передачу ядерного оружия в собственность НАТО, и создание вместо этого постоянных ядерных органов НАТО без передачи ядерного оружия в распоряжение блока представляет меньшую опасность для дела мира, чем прежние планы создания ядерных сил НАТО. Однако попытки руководства НАТО представить эти органы как «чисто консультативные» и полностью безопасные учреждения несостоятельны. Их практическая деятельность убедительно показывает, что эти органы являются, по сути дела, наиболее непосредственной, конкретной формой приобщения участников НАТО к ядерному планированию, которая когда-либо осуществлялась за время существования НАТО, являются ядерным штабом НАТО, в котором разрабатываются планы атомной войны. Следует отметить, что союзники Соединенных Штатов по НАТО не отказались от надежд на получение доступа к ядерному оружию в будущем посредством создания так называемых «европейских ядерных сил». Этот вариант доступа к ядерному оружию стал в настоящее время одним из главных в расчетах противников разрядки напряженности в Европе.

Советский Союз, неуклонно и последовательно проводя в жизнь Программу мира, принятую XXIV съездом КПСС, выступает за углубление разрядки и расширение мирного сотрудничества государств. Вместе с тем, как подчеркнул апрельский (1973 г.) Пленум ЦК КПСС, в нашей стране полностью осознается необходимость постоянной бдительности и готовности давать отпор любым проискам агрессивных реакционных кругов империализма. Советский Союз внимательно следит за всеми опасными тенденциями в политике Североатлантического блока. Он принимает все необходимые меры, чтобы пресечь ядерные притязания милитаристских кругов НАТО, добиться безусловного соблюдения Договора о нераспространении ядерного оружия и не допустить расползания этого оружия в мире.

Осуществление этой задачи — одно из важных условий предотвращения ядерной войны и обеспечения прочного мира на нашей планете.

Сейчас, когда противники разрядки напряженности вновь пытаются возродить различные варианты доступа участников НАТО к ядерному оружию, важно знать как недавнюю историю, так и нынешнюю ядерную политику Североатлантического блока.

Как возникли планы создания объединенных ядерных сил НАТО и какие политические и стратегические дели преследовали их инициаторы? В чем проявились острейшие межимпериалистические противоречия в блоке по этому вопросу и почему провалился проект МЯС? В каких формах осуществляется ныне приобщение участников НАТО к «ядерному планированию» в блоке и какие это влечет за собой последствия? Имеют ли, наконец, партнеры США по НАТО шансы получить доступ к ядерному оружию в будущем?

С этими и некоторыми другими вопросами, связанными с деятельностью Североатлантического блока, призвана ознакомить читателя настоящая работа.
	\newpage
	
	\section[Глава первая. ПОДГОТОВКА ПЛАНОВ ЯДЕРНОГО ВООРУЖЕНИЯ НАТО]{\center ГЛАВА ПЕРВАЯ.\\ \textbf{ПОДГОТОВКА ПЛАНОВ ЯДЕРНОГО ВООРУЖЕНИЯ НАТО}}	
	\subsection[Стратегия «меча и щита» и проект «четвертой ядерной державы»]{\center СТРАТЕГИЯ «МЕЧА И ЩИТА» И ПРОЕКТ «ЧЕТВЕРТОЙ ЯДЕРНОЙ ДЕРЖАВЫ»}

	Планы ядерного вооружения НАТО возникли не на пустом месте: они были подготовлены всем предшествующим, развитием политики и стратегии этого агрессивного военного блока.
	
	
	\newpage
	\tableofcontents
	
	\thispagestyle{empty} % 
	
	\newpage
	
	\setcounter{secnumdepth}{0}  
	
	\phantomsection
	
		\section*{Описание}
	
	{\bf Название:} НАТО И АТОМ  (Ядерная политика Североатлантического блока) 
	
{\bf Автор:} Борис Михайлович Халоша
	
{\bf Издательство:} Москва: «Знание» 1975
	
		{\bf Редактор:} \textit{К. М. Чушкова}
	
		{\bf Художественный редактор:} \textit{В. И. Пантелеев}
	
		{\bf Технический редактор:} \textit{М. Т. Столярова}
	
		{\bf Корректор:} \textit{О. Ю. Мигун}
	
		{\bf Аннотация:} В книге рассматриваются различные аспекты политики ядерного вооружения Североатлантического блока (НАТО). В ней показывается, как менялась эта политика в связи с новым  соотношением сил на мировой арене, рассматриваются попытки руководства блока создать объединенные ядерные силы НАТО, выявившиеся противоречия вокруг этого проекта и его провал. Значительное место в книге уделяется анализу деятельности нынешних органов ядерного планирования НАТО, раскрывается их опасный для дела мира и безопасности народов характер. Автор освещает широкий круг проблем, связанных с борьбой Советского Союза, всех миролюбивых сил за запрещение ядерного оружия, предотвращение его распространения, создание безъядерных зон в различных районах мира. 
		\thispagestyle{empty} % выключаем отображение номера для этой страницы

	
\end{document}


