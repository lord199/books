
%\documentclass[oneside,final,14pt]{extreport}
\documentclass[12pt, a4paper, openany]{book}
\usepackage[left=2.5cm,right=2.5cm,top=2cm,bottom=2cm,bindingoffset=0cm]{geometry}
%\usepackage[koi8-r]{inputenc}
\usepackage[russianb]{babel}
\usepackage{vmargin}
\setpapersize{A4}
\usepackage[T2A]{fontenc}
\usepackage[utf8x]{inputenc}  % more recent versions (at least>=2004-17-10)
%\usepackage[russian]{babel}
%\setmarginsrb{2cm}{1.5cm}{1cm}{1.5cm}{0pt}{0mm}{0pt}{13mm}
\usepackage{indentfirst}
\usepackage{nicefrac} % For comparison
%\usepackage{xfrac}    % Works better with other fonts
%\usepackage[unicode, pdftex]{hyperref}
\usepackage{lettrine}
\usepackage[usenames]{color}
\special{papersize=a5}
\usepackage{colortbl}
\usepackage[pagestyles]{titlesec}
\usepackage{xfrac}    % Works better with other fonts
\usepackage[colorlinks=true,linkcolor=black,urlcolor=black,bookmarksopen=true]{hyperref}


% Настройка вертикальных и горизонтальных отступов
\titlespacing{\chapter}{0pt}{5pt}{5pt}
\titlespacing{\section}{\parindent}{4mm}{4mm}
\titlespacing{\subsection}{\parindent}{3mm}{3mm}

\newcommand{\anonsection}[1]{ \section*{#1} \addcontentsline{toc}{section}{\numberline {}#1}} 

\makeatletter %%%%% <---- Starting chapter without a pagebreak
\renewcommand\chapter{\par%
	\thispagestyle{plain}% \global\@topnum\z@
	\@afterindentfalse \secdef\@chapter\@schapter}
\makeatother %%%%% <---- Starting chapter without a pagebreak
\titleformat{\chapter}[display]
{\normalfont\bfseries}{}{0pt}{\Large}

\newpagestyle{mystyle}{
	\sethead[\thepage][][]{}{}{\thepage}	
}

\pagestyle{mystyle}

\sloppy
\begin{document}

	
	\begin{titlepage}
		
		\begin{center}
			%\vfill
			
			%\vfill
			\topskip0pt
			\vspace*{\fill}
			
			
			{\large\bf Б. М. ХАЛОША\\}
			\ \\
			\ \\
			{\Huge\bf НАТО И АТОМ\\}
			\ \\
			\ \\
			\ \\
			(Ядерная политика Североатлантического блока) 
			\vspace*{\fill}    
			
			\vfill
			
				\begin{flushright}
			
			Издательство
		
			«Знание»
		
			Москва 1975
			
		\end{flushright}
		\end{center}
		
	\end{titlepage}
	
	\thispagestyle{empty} % выключаем отображение номера для этой страницы
	
	\newpage
	

\setcounter{secnumdepth}{0}  
	
	\section[Введение]{\center ВВЕДЕНИЕ}
	
Начало 70-х годов нынешнего столетия войдет в историю международных отношений как время глубоких политических перемен на мировой арене.

Благодаря настойчивой борьбе Советского Союза, других социалистических государств, всех миролюбивых сил, а также реализму, «проявленному в политике ряда капиталистических стран, достигнуты серьезные сдвиги в сторону разрядки напряженности на международной арене. Совершается исторический поворот от «холодной войны» и опасной напряженности к совместным усилиям по укреплению мира и развитию взаимовыгодного сотрудничества.

Вся история советской внешней политики от Декрета о мире, принятого в 1917 году, и до Программы мира, выдвинутой XXIV съездом КПСС, и последующего успешного осуществления этой Программы свидетельствует о настойчивой и целеустремленной борьбе Советского Союза за мир, за широкое внедрение в практику международных отношений ленинских принципов мирного сосуществования.

В то же время ход мировых событий убедительно показывает, что на пути движения человечества к прочному миру и сотрудничеству между народами существуют еще немалые препятствия. Силы империализма, реакции и войны не сложили оружие. Они упорно противятся процессу оздоровления международной обстановки, пытаются возродить времена «холодной войны» и опасной напряженности в мире. «Мы реалисты, — говорил Л. И. Брежнев в докладе «О пятидесятилетии Союза Советских Социалистических республик», — и хорошо видим, что влиятельные круги в мире империализма все еще не отказались от попыток проводить политику «с позиции силы». Все еще продолжается развязанная ими гонка вооружений, создающая угрозу миру. Мы и наши союзники, естественно, не можем не делать из этого необходимых выводов. Но миролюбивый курс нашей внешней политики неизменен, и в современной обстановке возможности миролюбивых сил в их борьбе против сил агрессии и войны велики как никогда»{\footnote{Л. И. Брежнев. О пятидесятилетии Союза Советских Социалистических Республик. М., Политиздат, 1972, стр. 42.}}.

Как известно, одним из главных орудий подрывной деятельности агрессивных кругов империализма против мира и безопасности народов был и остается Североатлантический блок (НАТО){\footnote{НАТО — сокращенное название, происходящее от начальных букв официального английского наименования этого военного блока — «Организация Североатлантического договора». (North Atlantic Treaty Organization—NATO).}}.

Напомним, что договор о создании этой замкнутой военно-политической группировки западных держав был подписан 4 апреля 1949 года в Вашингтоне. Первоначально в НАТО вошли Соединенные Штаты Америки, Великобритания, Франция, Бельгия, Голландия, Люксембург, Канада, Италия, Португалия, Норвегия, Дания и Исландия. В феврале 1952 года в Североатлантический союз были включены Турция и Греция, а в мае 1955 года — Федеративная Республика Германии.

НАТО занимает центральное место среди империалистических военных союзов, созданных для борьбы с мировой социалистической системой, национально-освободительным, рабочим и демократическим движением.

За двадцать пять лет существования НАТО наиболее агрессивные милитаристские круги этого блока использовали и используют его для того, чтобы превратить Западную Европу в вооруженный лагерь и военный плацдарм. Через систему НАТО они добивались и добиваются усиления гонки вооружений, увеличения вооруженных сил в западноевропейских странах, раздувания ими военных бюджетов, милитаризации их экономики.

И сейчас влиятельные круги в странах — членах НАТО всячески стремятся помешать положительным изменениям, происходящим на международной арене, навязать подход к решению важнейших политических проблем современности с обанкротившейся «позиции силы». Процесс разрядки, все более развивающийся в Европе, тревожит представителей военно-промышленного комплекса, милитаристские круги НАТО, которые усматривают в нем угрозу самому существованию этой военной группировки. Вооруженная интервенция НАТО против независимой суверенной Республики Кипр, предпринятая летом 1974 г., — новое свидетельство агрессивного характера этой военной группировки.

Особенно опасный характер для дела мира и безопасности народов имели и имеют разрабатываемые и осуществляемые в НАТО вот уже на протяжении многих лет различные варианты приобщения участников этой военной группировки к ядерному оружию и стратегии.

Вопрос о создании в рамках НАТО так называемых «многосторонних ядерных сил» (МЯС) «неизменно фигурировал на многочисленных совещаниях и встречах ответственных государственных и военных деятелей стран НАТО, состоявшихся в 60-е годы.

Создание ядерных сил НАТО явилось бы серьезной угрозой для человечества. Оно (неизмеримо увеличило бы опасность возникновения мировой термоядерной войны, послужило бы стимулом к еще большему усилению гонки ракетно-ядерного вооружения и шагом на пути широкого распространения атомного оружия среди империалистических держав. Ядерное оружие с его неизбежным спутником — военной опасностью расползлось бы по нашей планете.

Планы создания объединенных ядерных сил НАТО представляли особую опасность для дела мира во всем мире, потому что в случае их реализации к спусковым кнопкам ядерных сил НАТО оказались бы в конечном итоге допущенными милитаристские, реваншистские круги ФРГ — единственной страны в Европе, которая при правительствах Аденауэра — Эрхарда — Кизингера открыто заявляла о своих территориальных притязаниях к своим соседям и намерениях пересмотреть итоги второй мировой войны.

Планы создания МЯС встретили самый решительный и резкий отпор со стороны Советского Союза и других социалистических стран, что явилось важнейшим фактором, воспрепятствовавшим реализации этих опасных замыслов. Создание «многосторонних ядерных сил» НАТО натолкнулось на решительное сопротивление международной общественности и серьезные разногласия между самими атлантическими партнерами. В этих условиях Соединенные Штаты Америки вынуждены были отказаться от планов создания «многосторонних ядерных сил» НАТО.

Срыв планов создания ядерных сил НАТО привел к тому, что наиболее милитаристские круги европейских участников этого блока были вынуждены ограничить свои ядерные притязания участием в работе постоянных ядерных органов НАТО (Комитет по вопросам ядерной обороны, Группа ядерного планирования), разрабатывающих ядерную стратегию, но не обладающих контролем над самим ядерным оружием.

Это, в частности, создало необходимые условия для подписания Договора о нераспространении ядерного оружия, вступившего в силу 5 марта 1970 года, и явившегося большой победой внешней политики Советского Союза, других социалистических стран, всех миролюбивых сил. Поэтому хотя события, связанные с разработкой планов создания МЯС, и оказались «а недавно перевернутой странице истории, — они тем не менее теснейшим образом связаны с международными событиями сегодняшнего дня.

Отказ США от создания «многосторонних ядерных сил» НАТО явился одним из убедительных примеров того, что империализм, учитывая мощь Советского Союза, вынужден идти на уступки и заключение соглашений, способствующих разрядке международной напряженности. Гонка ракетно-ядерных вооружений, как показывает практика, не в состоянии обеспечить Соединенным Штатам военного и политического превосходства. Это побуждает реалистически мыслящие круги США становиться на путь заключения соглашений с СССР об их ограничении на основе признания принципа равной одинаковой безопасности сторон. Свидетельством этого являются заключенные в результате первой советско-американской встречи на высшем уровне в Москве в мае 1972 года Договор об ограничении систем противоракетной обороны и Временное соглашение о некоторых мерах в области ограничения стратегических наступательных вооружений. Эти соглашения проложили путь к дальнейшим мерам по ограничению гонки вооружений и уменьшению опасности ядерной войны. Важное значение в этом отношении имеют подписанные между СССР и США в июне 1973 года во время визита Генерального секретаря ЦК КПСС Л. И. Брежнева в США «Основные принципы переговоров о дальнейшем ограничении стратегических наступательных вооружений». Исключительное значение в этом отношении имеет и подписанное во время этого визита Соглашение между СССР и США о предотвращении ядерной войны.

На состоявшейся с 27 «июня иго 3 июля 1974 года в Советском Союзе третьей советско-американской встрече на высшем уровне первостепенное место занимали проблемы дальнейшего уменьшения опасности войны и сдерживания гонки вооружений. Участники встречи достигли договоренности об ограничении систем противоракетной обороны двух стран, о согласованном ограничении подземных испытаний ядерного оружия, о дальнейших усилиях, направленных на ограничение стратегических наступательных вооружений, о принятии мер, нацеленных на исключение химического оружия из арсеналов государств.

Несомненно, провал планов создания объединенных ядерных сил НАТО, предусматривавших передачу ядерного оружия в собственность НАТО, и создание вместо этого постоянных ядерных органов НАТО без передачи ядерного оружия в распоряжение блока представляет меньшую опасность для дела мира, чем прежние планы создания ядерных сил НАТО. Однако попытки руководства НАТО представить эти органы как «чисто консультативные» и полностью безопасные учреждения несостоятельны. Их практическая деятельность убедительно показывает, что эти органы являются, по сути дела, наиболее непосредственной, конкретной формой приобщения участников НАТО к ядерному планированию, которая когда-либо осуществлялась за время существования НАТО, являются ядерным штабом НАТО, в котором разрабатываются планы атомной войны. Следует отметить, что союзники Соединенных Штатов по НАТО не отказались от надежд на получение доступа к ядерному оружию в будущем посредством создания так называемых «европейских ядерных сил». Этот вариант доступа к ядерному оружию стал в настоящее время одним из главных в расчетах противников разрядки напряженности в Европе.

Советский Союз, неуклонно и последовательно проводя в жизнь Программу мира, принятую XXIV съездом КПСС, выступает за углубление разрядки и расширение мирного сотрудничества государств. Вместе с тем, как подчеркнул апрельский (1973 г.) Пленум ЦК КПСС, в нашей стране полностью осознается необходимость постоянной бдительности и готовности давать отпор любым проискам агрессивных реакционных кругов империализма. Советский Союз внимательно следит за всеми опасными тенденциями в политике Североатлантического блока. Он принимает все необходимые меры, чтобы пресечь ядерные притязания милитаристских кругов НАТО, добиться безусловного соблюдения Договора о нераспространении ядерного оружия и не допустить расползания этого оружия в мире.

Осуществление этой задачи — одно из важных условий предотвращения ядерной войны и обеспечения прочного мира на нашей планете.

Сейчас, когда противники разрядки напряженности вновь пытаются возродить различные варианты доступа участников НАТО к ядерному оружию, важно знать как недавнюю историю, так и нынешнюю ядерную политику Североатлантического блока.

Как возникли планы создания объединенных ядерных сил НАТО и какие политические и стратегические дели преследовали их инициаторы? В чем проявились острейшие межимпериалистические противоречия в блоке по этому вопросу и почему провалился проект МЯС? В каких формах осуществляется ныне приобщение участников НАТО к «ядерному планированию» в блоке и какие это влечет за собой последствия? Имеют ли, наконец, партнеры США по НАТО шансы получить доступ к ядерному оружию в будущем?

С этими и некоторыми другими вопросами, связанными с деятельностью Североатлантического блока, призвана ознакомить читателя настоящая работа.
	\newpage
	
	\section[Глава первая. ПОДГОТОВКА ПЛАНОВ ЯДЕРНОГО ВООРУЖЕНИЯ НАТО]{\center ГЛАВА ПЕРВАЯ.\\ \textbf{ПОДГОТОВКА ПЛАНОВ ЯДЕРНОГО ВООРУЖЕНИЯ НАТО}}	
	\subsection[Стратегия «меча и щита» и проект «четвертой ядерной державы»]{\center СТРАТЕГИЯ «МЕЧА И ЩИТА» И ПРОЕКТ «ЧЕТВЕРТОЙ ЯДЕРНОЙ ДЕРЖАВЫ»}

	Планы ядерного вооружения НАТО возникли не на пустом месте: они были подготовлены всем предшествующим, развитием политики и стратегии этого агрессивного военного блока.
	
	Чтобы глубже понять политику НАТО в вопросах ракетно-ядерного вооружения, необходимо проследить ее в «динамике». Известно, что подготовка к ядерной войне всегда находилась в центре внимания организаторов Североатлантического военного союза. Уже вскоре после того, как был создан Североатлантический блок, была одобрена основная стратегическая концепция блока, впоследствии получившая наименование стратегии «меча и щита».
	
	Что она собой представляла?
	
	Сущность этой стратегической концепции, предварительно разработанной в США, состояла в том, что в НАТО было намечено четкое разграничение военных обязанностей. Стратегическая атомная бомбардировка возлагалась на Соединенные Штаты Америки (роль «меча»), а роль основных поставщиков пехоты отводилась западноевропейским участникам НАТО (роль «щита»). Основное содержание этого плана было довольно подробно изложено бывшим тогда председателем комитета начальников штабов США генералом Брэдли в его нашумевшем выступлении 29 июля 1949 года в комиссии по иностранным делам палаты представителей. Он заявил: «Их (держав, подписавших Атлантический пакт) стратегия основана на предполагаемых факторах. Во-первых, на Соединенные Штаты будут возложены стратегические бомбардировки. Высший приоритет в объединенной обороне отводится, поставкам атомных бомб. Во-вторых, военно-морской флот США и морские державы Западного союза будут проводить важнейшие военно-морские операции, включая охрану военных путей... В-третьих, главы объединенного штаба признают, что основное ядро наземных сил будет поставляться странами Европы с помощью других стран по мере осуществления ими мобилизации».
	
	Этот стратегический замысел, положенный в основу всего военного планирования блока, был одобрен на сессии Совета НАТО, состоявшейся 6 января 1950 года в Вашингтоне.
	
	Стратегия «меча и щита», по существу, закрепляла за Соединенными Штатами господствующую роль в империалистическом лагере, поскольку США выступали в роли единственного ядерного арсенала и «гаранта» Запада.
	
	Эта стратегия являлась военным эквивалентом провозглашенной примерно в то же время общей агрессивной внешнеполитической доктрины американского империализма — доктрины «сдерживания международного коммунизма» и была призвана обеспечить военный базис для практического осуществления последней. Но фактически в то время, когда США добились одобрения другими участниками НАТО концепции «меча и щита», эта концепция в отношении социалистических стран уже «не имела реальной основы. Советский Союз в самый короткий срок ликвидировал монополию США на атомное оружие. Это был первый серьезный удар по военно-стратегическим замыслам империалистов. Однако, все еще надеясь на длительное сохранение в НАТО монополии в области ядерного оружия и рассматривая ее как важнейшее орудие своего господства в блоке, Соединенные Штаты Америки в течение ряда лет после создания НАТО не передавали это оружие своим союзникам и не делились с ними секретом производства этого оружия. Закон об атомной энергии, известный под названием закона Мак-Магона и принятый в 1946 году, прямо ссылаясь на «очевидное значение» ядерного оружия для достижения внешнеполитических и военных целей США, не разрешал передачу американского атомного оружия, а также информации о «производстве или применении» этого оружия иностранным государствам. Подобное положение отражало расстановку сил в империалистическом лагере, сложившуюся в первые годы после второй мировой войны, когда США еще занимали господствующие экономические позиции в капиталистическом мире, обладали монополией на ядерное оружие. Поэтому в 1951 году в законе Мак-Магона были даже еще более усилены статьи, касающиеся секретности. «Особые отношения» в ядерной области установились лишь между США и Англией в связи с превращением последней в атомную державу и допуском ее к части атомных секретов США, хотя сам факт овладения Англией в октябре 1952 года ядерным оружием и начавшееся атомное соперничество между США и Англией свидетельствовали о том, что расчеты Вашингтона на длительное сохранение своей монополии на ядерное оружие даже в западном мире оказались несостоятельными.
	
	Ставка правящих кругов США на ядерное оружие в НАТО особенно возросла после прихода в 1952 году к власти правительства республиканцев и замены доктрины «сдерживания» еще более реакционной и агрессивной доктриной «освобождения» социалистических государств или «отбрасывания» коммунизма. Военным эквивалентом этой новой внешнеполитической доктрины США являлась военная доктрина «массированного возмездия», игравшая на протяжении середины и второй половины 50-х годов определяющую роль в системе стратегических мероприятий американского империализма. Доктрина «массированного возмездия» подразумевала применение ядерного оружия в любом военном конфликте с социалистическими странами.
	
	Все это остро ставило перед Соединенными Штатами Америки вопрос о том, как осуществить в этих условиях и на наиболее выгодных для себя основах политику ядерного вооружения стран — участниц Североатлантической военной группировки. На сессии Совета НАТО, состоявшейся в апреле 1953 года, впервые был поставлен вопрос об атомном вооружении Североатлантического союза. В коммюнике следующей сессии Совета НАТО, состоявшейся в декабре 1953 года, уже прямо указывалось на необходимость изыскать «решение проблемы атомного оружия». На сессии Совета НАТО, состоявшейся в декабре 1954 года в Париже, был одобрен специальный доклад военного комитета относительно «наиболее эффективной формы военной мощи НАТО... с учетом развития в области современных видов вооружения и техники», то есть в области ядерного оружия. Этот доклад был положен в качестве «основы» для военного планирования и военных приготовлений стран — участниц Североатлантического союза. На сессии Совета НАТО, состоявшейся в декабре 1956 года, Соединенные Штаты объявили уже о своем намерении поставить в ближайшие годы для вооружения армий стран — участниц НАТО и, в частности, западногерманского "бундесвера так называемое тактическое ракетное оружие — неуправляемые снаряды «Онест Джон», самолеты-снаряды («Матадор» и зенитные «ракеты «Найк», могущие нести как обычные, так и ядерные заряды.
	
	Новый этап в американской политике ракетно-ядерного вооружения НАТО наступил в 1957 году.
	
	Почему это произошло?
	
	
	Изумительный рывок, сделанный в том году Советским Союзом в области науки и ракетной техники и нашедший свое выражение в создании им межконтинентальной баллистической ракеты и в запуске первых в мире искусственных спутников земли, развеял прежнее представление о военном превосходстве и «неуязвимости» Соединенных Штатов Америки. Стало совершенно очевидно, что их территория так же уязвима в случае войны, как и территория любой другой страны — участницы НАТО.
	
	Все это остро ставило перед американскими военными кругами новые проблемы в области проведения политики ракетно-ядерного вооружения НАТО.
	
	Соединенные Штаты, стремясь в этих условиях компенсировать свое серьезное отставание от Советского Союза в области производства межконтинентальных баллистических ракет, приложили в первую очередь усилия к созданию ракетно-ядерных баз на территории своих западноевропейских партнеров по НАТО. Для этого на сессии Совета НАТО с участием глав правительств, состоявшейся в Париже 16—19 декабря 1957 года, Соединенные Штаты провели решение — «предоставить в распоряжение верховного главнокомандующего вооруженными силами союзников в Европе баллистические снаряды среднего радиуса действия» и создать в Западной Европе «запасы ядерных боевых зарядов». Американский журнал «Форчун» безо всяких обиняков писал тогда по поводу этого решения: «НАТО, оснащенная баллистическими ракетами средней дальности полета и ядерными зарядами, обещает быть исключительно полезной организацией в такой период, когда американские вооруженные силы... будут переоснащаться межконтинентальными баллистическими ракетами». Указывая на подлинное назначение американских ракетно-ядерных баз на территориях западноевропейских стран—участниц НАТО, журнал прямо писал, что «для США это критически важная дуга во всемирной системе баз, которая позволяет нашей стране расположить всю свою ядерную ударную мощь в пределах досягаемости советской державы».
	
	Взяв курс на размещение своего ракетно-ядерного оружия среднего радиуса действия на территории западноевропейских стран — участниц НАТО, Соединенные Штаты Америки разработали вместе с тем сложную систему мер, направленную на то, чтобы сохранить в своих руках монополию на ядерное оружие в НАТО, использовать ее для дальнейшего усиления зависимости своих партнеров в военном отношении.
	
	Каким образом это достигалось?
	
	По сообщениям, просочившимся в западную печать, на боннской и парижской сессиях Совета НАТО, состоявшихся в мае и декабре 1957 года рассматривался специальный атомный план для НАТО. Сущность этого плана заключалась в следующем: предоставляя своим партнерам по НАТО ядерное оружие — атомные бомбы, ракеты, управляемые снаряды, Соединенные Штаты решили сохранить под своим односторонним контролем ядерные боевые заряды от этого оружия. Ядерные боевые заряды хотя и были доставлены на территории стран — участниц НАТО, но продолжали находиться под прямым контролем верховного главнокомандующего вооруженными силами НАТО в Европе, каковым неизменно являлся и является американский генерал. И хотя на территории стран — участниц НАТО и действовала так называемая система «двойного контроля» (предусматривавшая право вето на запуск ракетно-ядерного оружия правительством страны, на территории которого оно было расположено) в условиях сохранения одностороннего американского контроля над ядерными боезарядами, она теряла свое значение; решение о применении ядерного оружия в Европе фактически могло приниматься только Вашингтоном.
	
	Посредством такой системы Соединенные Штаты Америки фактически узурпировали в своих руках право единолично решать вопрос о развязывании войны с применением ракетно-ядерного оружия, рассматривая это как важнейший фактор сохранения своего господства в НАТО. Именно упомянутые выше условия использования американского ракетно-ядерного оружия и были положены в основу соглашений о создании баз для ракет среднего радиуса действия, которые Соединенные Штаты Америки заключили в 1958 и 1959 годах с Англией, Италией и Турцией. Проанализируем одно из этих соглашений — соглашение между США и Англией. В соответствии с этим соглашением о создании американских ракетно-ядерных баз на территории Англии, заключенном 22 февраля 1958 года, Соединенные Штаты взяли на себя обязательство поставить Англии «согласованное количество ракетных снарядов среднего радиуса действия и относящееся к нему специализированное оборудование», а также «оказать необходимую помощь» для скорейшего освоения английским персоналом американских ракетных снарядов (статья 1). По условиям этого соглашения (статья 8) ядерные боевые заряды от американского ракетного оружия, поставляемого Англии, должны были «оставаться полностью в распоряжении Соединенных Штатов, под их охраной и контролем».
	
	Но один шаг в области ракетно-ядерного вооружения неизбежно повлек за собой и другие.
	
	Какие именно?
	
	Взяв курс на размещение своего ракетно-ядерного оружия на территории стран — участниц НАТО, Соединенные Штаты оказались перед необходимостью передать своим союзникам по блоку и определенную сумму секретных сведений о ядерном и ракетном оружии во имя дальнейшего атомного вооружения НАТО. С этой целью они произвели ряд последовательных ослаблений установленных законом Мак-Магона ограничений в области обмена атомной информацией с союзниками. Наиболее значительное ослабление ограничений было произведено в 1958 году, когда были приняты новые дополнения к закону об атомной энергии. Эти дополнения предусматривали заключение подлежавших утверждению конгрессом США специальных соглашений с союзниками США относительно более широкого обмена: а) неядерными частями оружия и военных реакторов, а также ядерным сырьем; б) секретной информацией с целью содействия отдельным странам и таким военным организациям, как НАТО, в совершенствовании своей военной подготовки; в) секретными сведениями и сырьем для производства оружия со странами, уже освоившими технику его производства.
	
	В соответствии с новым законом об атомной энергии Соединенные Штаты в различное время заключили со странами — участницами НАТО — Англией, Францией, Западной Германией, Италией, Голландией, Бельгией, Грецией, Турцией и Канадой двусторонние соглашения, предусматривавшие передачу технической информации о ядерной энергии, деталях и рабочих частях ядерного оружия, а также обучение персонала этих стран обращению с ядерным оружием.
	
	Так, вслед за предоставлением странам НАТО средств доставки ядерного оружия Соединенные Штаты сделали еще один серьезный шаг на пути приобщения к этому оружию своих союзников по блоку. Однако на всех этапах ослабления ограничений США ревностно сохраняли контроль над боевыми ядерными зарядами целиком в своих руках, надеясь таким путем продлить свою монополию на ядерное оружие в системе НАТО и сохранить политическую и военную зависимость своих союзников.
	
	Подобные условия предоставления американского ракетно-ядерного оружия странам НАТО, при которых решение вопроса о применении этого оружия фактически завесило от Вашингтона, вызывали резкое недовольство со стороны различных политических и военных кругов этих стран. Достаточно указать, например, на позицию, занятую в этом вопросе правительством Франции, которое отказалось от предоставления территории страны под ракетно-ядерные базы США, выдвинув требование о том, чтобы США согласились передать Франции полный контроль как над ракетами, так и над ядерными зарядами к ним. Не решилось дать согласие на создание ракетно-ядерных баз на своей территории и правительство Греции. Отрицательную позицию в этом вопросе заняли также Скандинавские страны — участницы НАТО — Дания и Норвегия.
	
	Между тем стремление военных кругов США иметь в Европе ракетно-ядерное оружие, нацеленное на СССР и другие социалистические страны, было так велико, что для его удовлетворения США готовы были пойти на некоторый компромисс с требованиями своих союзников и прежде всего западногерманских милитаристов, допустив их хотя бы к частичному контролю над этим оружием.
	
	Таким путем Соединенные Штаты рассчитывали сгладить обострившиеся противоречия в НАТО, преодолеть сопротивление западноевропейских стран размещению американского ракетно-ядерного оружия на их территориях. Немаловажное значение в ряду соображений, которыми руководствовались при этом правящие круги США, был расчет и на то, чтобы, связав между собой союзников по НАТО наличием совместных ядерных сил, предотвратить такое положение, когда созданное или создаваемое отдельными участниками блока ядерное оружие оказалось бы вне влияния и контроля США.
	
	Подобные планы и расчеты правящих кругов США, по сути дела, отражали ту новую расстановку сил в империалистическом лагере, которая сложилась к началу 60-х годов. Позиции США в этом лагере ослабли: к концу 50-х годов в результате неравномерности развития капиталистических государств удельный вес США в промышленном производстве капиталистического мира и мировой торговле значительно уменьшился за счет укрепления позиций стран Западной Европы и Японии.
	
	Изменение соотношения сил в экономической области не замедлило сказаться и на политических взаимоотношениях между главными империалистическими державами: все более отчетливо стали выступать претензии европейских союзников США по НАТО и прежде всего Франции, ФРГ на большую роль в разработке «глобальной» стратегии НАТО. Эти державы все больше стали рваться к обладанию собственным ядерным оружием. Недвусмысленные заявления тогдашнего правительства Франции о намерении создать собственные «ударные ядерные силы», совершенно не связанные с НАТО и независимые от нее, подкрепленные первыми взрывами французского ядерного оружия в Сахаре в начале 1960 года, явились подтверждением того, что такая возможность стала вполне реальной. Громко звучал голос Бонна, также стремившегося приобщиться к атомной бомбе. Появление атомного оружия у других империалистических держав — участниц НАТО повлекло бы за собой дальнейшее ослабление позиций США в капиталистическом мире и, главное, потерю ими возможности единолично решать вопрос о применении ядерного оружия. Перед находившимся в то время у власти в США правительством Эйзенхауэра со всей остротой встал вопрос о том, каким путем в этих условиях закрепить руководящее положение США в Североатлантическом блоке, как смягчить противоречия, начавшие раздирать этот блок, особенно по вопросам ядерной политики и стратегии.
	
	
	
	В этих условиях в Соединенных Штатах начала оформляться идея о создании ядерных сил НАТО или, по распространенному тогда выражению, о превращении блока в «четвертую атомную державу» (наряду с США, Англией и Советским Союзом).
	
	На сессии Совета НАТО, состоявшейся в Париже 16—18 декабря 1960 года, тогдашний государственный .секретарь США Гертер официально внес предложение об образовании так называемых ядерных сил НАТО. Эти предложения предусматривали передачу в распоряжение командования НАТО пяти американских подводных лодок, оснащенных ракетами «Поларис» с ядерными зарядами, создание в Западной Европе специальных баз для таких подводных лодок, сформирование «смешанных команд» из американцев и их западноевропейских союзников по обслуживанию и использованию этого оружия. Кроме того, Соединенные Штаты предложили своим союзникам по НАТО «купить» 100 ракет «Поларис», которые должны были быть размещены на различных военных судах.
	
	В рамках НАТО предполагалось создать руководящую группу из представителей США, Англии, Франции, ФРГ и, возможно, Канады с весьма широкими функциями, в которые входило бы планирование ядерной войны против СССР и других социалистических государств.
	
	Государственный департамент, Министерство обороны и комиссия по атомной энергии с целью практического осуществления плана создания ядерных сил НАТО намеревались обратиться к новому конгрессу, который должен был собраться в начале января 1961 года с просьбой одобрить передачу Соединенными Штатами своим союзникам контроля над некоторым ядерным оружием.
	
	Таким образом, правительство Эйзенхауэра незадолго до своего ухода «с политической сцены вплотную подошло к тому, чтобы начать осуществление далеко идущего плана, неизбежным следствием чего было бы фактическое распространение ядерного оружия внутри НАТО.
	
		\subsection[Путь к ядерному вооружению ФРГ]{\center ПУТЬ К ЯДЕРНОМУ ВООРУЖЕНИЮ ФРГ}
	
	
	
	
	Характерно, что никто из стран — участниц НАТО не поддержал с таким энтузиазмом план создания атомных сил НАТО, как это сделал Бонн.
	
	Почему так произошло?
	
	Превращение НАТО в «четвертую ядерную державу», как мы покажем ниже, «а практике означало бы превращение в такую державу Федеративной Республики Германии со всеми вытекавшими отсюда последствиями. Сразу же после появления нового американского предложения бывший в то время канцлер ФРГ Аденауэр безоговорочно поддержал этот план. Во время состоявшейся 16—18 декабря I960 года сессии Совета НАТО тогдашний министр обороны ФРГ Штраус также поспешил заявить о полной поддержке Бонном плана создания «четвертой ядерной державы» и готовности правительства ФРГ дать положительный ответ на предложение верховного главнокомандующего вооруженными силами НАТО предоставить подразделение западногерманского бундесвера в состав атомной ударной группировки НАТО. По сообщению агентства ДПА в начале января 1960 года, Штраус подтвердил это в специальном письме к тогдашнему верховному главнокомандующему вооруженными силами НАТО американскому генералу Норстэду. По сообщению газеты «Ди вельт», Штраус «горячо приветствовал» поставленное на обсуждение парижской сессии Совета НАТО предложение американцев предоставить в распоряжение НАТО пять атомных подводных лодок, каждая из которых была бы оснащена 16 ракетами «Поларис». «Министр, — писала газета «Ди вельт», — рассматривает это предложение, по его словам, как первую ступень к превращению союза в «четвертую ядерную державу».
	
	
	
	Тот факт, что боннские военные круги так рьяно ухватились за план превращения НАТО в «четвертую ядерную державу» весьма показателен. Еще бы, реализация этого плана вложила бы со временем в их руки готовое ракетно-ядерное оружие! Индийская газета «Нейшнл геральд» справедливо отмечала в связи с появлением плана, что в поддержке Аденауэром этого плана «никогда не приходилось сомневаться. Это больше его детище, чем кого-либо другого, поскольку план предполагает сделать Федеративную Республику ядерной державой без той затраты усилий и денег, которые потребовались от других ядерных держав».
	
	Боннские военные круги стремились осуществить атомное вооружение бундесвера посредством плана «четвертой ядерной державы» еще и потому, что в данном случае непосредственная командная власть над атомными боевыми соединениями принадлежала бы самим западногерманским генералам. Тоддашвий боннский министр обороны Штраус в интервью пресс-бюллетеню «Боннер информационен аус эрстер ханд», близкому к ХДС, заявил, что ФРГ посредством плана «четвертой ядерной державы» намерена была не только обладать атомным оружием, но и иметь право решающего голоса в вопросе применения этого оружия массового уничтожения. Штраус прямо заявил: «Речь идет о том, чтобы, с одной стороны, в разумной форме обеспечить европейским партнерам право решающего голоса и, с другой стороны преобразовать структуру командования в соответствии с (военными потребностями, то есть упростить ее».
	
	О том, как западногерманские военные круги намерены были заполучить посредством осуществления плана «четвертой ядерной державы» право решающего голоса в вопросах применения ракетно-ядерного оружия, давало сообщение газеты «Франкфуртер рундшау». Как указывала эта газета, «немецкий контингент», выделенный с такой поспешностью Штраусом в состав специальных мобильных войск НАТО, именуемых «пожарной командой», должен был (быть вооружен американскими атомными ракетами типа «Онест Джон» или же американскими атомными орудиями калибра 203,2 миллиметра. Этот западногерманский атомный дивизион НАТО должен был быть размещен на территории ФРГ и, следовательно, непосредственно подчиняться тогдашнему командующему сухопутными войсками НАТО в Центральной зоне Европы — западногерманскому генералу Шпейделю.
	
	Таким образом, в случае претворения в жизнь плана превращения НАТО в «четвертую ядерную державу» реваншистские элементы в ФРГ, открыто заявлявшие о своих территориальных притязаниях в отношении соседних европейских государств, смогли бы в любой момент применить ракетно-ядерное оружие и тем самым вовлечь европейские страны и весь мир в уничтожающую ракетно-ядерную схватку.
	
	О такой серьезной опасности, которую представляло бы собой вооружение бундесвера ракетно-ядерным оружием, убедительно свидетельствовало сообщение еженедельника «Трибюн де Насьон» от 23 декабря 1960 года. В статье под характерным заголовком «Могут ли генералы Хойзингер или Шпейдель вовлечь НАТО в ядерную войну?» еженедельник указывал, что, как стало известно из заседаний ассамблеи Западноевропейского союза, решение об использовании тактического атомного оружия находится в ведении местных командующих НАТО. Таким образом, каждый командующий сектором может оказаться в таком положении, когда ему придется решать, следует ли применить атомные гаубицы и ракеты ближнего действия типа «Онест Джон». Хотя ядерные боеголовки от этого оружия и находились иод контролем американских военных властей, однако достаточно было приказа генерала Норстэда, по особой просьбе одного из командующих зоной, чтобы деблокировать их и передать в распоряжение ракетных частей, например бундесвера.
	
	Передача же в случае осуществления плана «четвертой ядерной державы» ядерных боеголовок бундесверу дала бы западногерманским военным кругам возможность в любой момент, не дожидаясь даже деблокирования ядерных боеголовок верховным главнокомандующим вооруженными силами НАТО, применить ракетно-ядерное оружие. Как совершенно справедливо отмечала английская газета «Обсервер» от 11 декабря 1960 года, в плане «четвертой ядерной державы» ФРГ увидела «возможность располагать атомным оружием (или по крайней мере иметь некоторое право контроля над его применением), что ей в данное время запрещено парижскими соглашениями».
	
	Следует также иметь в виду, что с возрастанием роли ФРГ в НАТО ее возможности влиять на ракетно-ядерную стратегию блока пропорционально возрастали. Весной 1961 года западногерманский генерал Хойзингер был назначен на весьма ответственный пост в НАТО — на пост председателя Постоянного военного комитета блока. Газета «Дейли экспресс» писала в связи с этим, что «в Бонне считают, что Хойзингер будет занимать влиятельный пост для того, чтобы (направлять военное планирование в соответствии с германской политикой». Другая газета «Дейли скетч» указывала в связи с этим назначением, что на своем новом посту Хойзингер фактически будет давать советы даже верховному главнокомандующему вооруженными силами НАТО.
	
	Выступая в прениях в Палате общин по вопросам политики в области обороны, лейборист Антони Гринвуд сказал в то время: «Одна из главных угроз миру во всем мире — это ирредентистские притязания Германии на ее утраченные территории». Он выразил при этом опасение, что перевооруженная Германия может предпринять нападение на Восточную Европу, чтобы попытаться вернуть потерянные территории, и в этом случае на Англию неизбежно обрушится возмездие.
	
	В связи с этим нам хотелось бы подчеркнуть еще одно обстоятельство: план «четвертой ядерной державы» в случае его претворения в жизнь не только привел бы к тому, что ракетно-ядерное оружие оказалось бы непосредственно в руках боннских военных кругов, но он в значительной мере облегчил бы также создание Федеративной Республикой Германии своего собственного ядерного оружия.
	
	Почему это могло произойти?
	
	В случае принятия плана «четвертой ядерной державы» Соединенные Штаты Америки неизбежно'были бы вынуждены расширить каналы предоставляемой странам НАТО информации, касающейся ракетно-ядерного оружия. В самом деле, это произошло бы в силу того, что уже сам факт обладания европейскими участниками НАТО этим оружием настоятельно диктовал бы необходимость предоставления им и более подробных сведений относительно производства и применения этого оружия. Совершенно очевидно, что в случае претворения в жизнь плана «четвертой ядерной державы», когда страны — члены НАТО получили бы в свои руки полные комплекты ракетно-ядерного оружия, в значительной мере потеряло бы всякий смысл ограничивать предоставление этим странам секретной информации, касающейся производства и применения ракетно-ядерного оружия. А это в огромной степени способствовало бы созданию европейскими участниками НАТО и прежде всего Федеративной Республикой Германии своего собственного ядерного оружия.
	
			\subsection[От плана «четвертой ядерной державы» к «пакту Нассау»]{\center ОТ ПЛАНА «ЧЕТВЕРТОЙ ЯДЕРНОЙ ДЕРЖАВЫ» К «ПАКТУ НАССАУ»}

	
	Правительство демократической партии США во главе с бывшим президентом Кеннеди, победившим на выборах 1960 года, заняло более осторожную позицию по отношению к плану «четвертой ядерной державы» НАТО. Характерно, что ни до, ни после избрания его президентом Кеннеди не высказался в поддержку этого плана. Напротив, по сообщению французского журнала «Экспресс» от 22 декабря 1960 года, проект «четвертой ядерной державы», изложенный тогдашним главнокомандующим вооруженными силами НАТО Норстэдом на парламентской конференции стран НАТО осенью 1960 года, был сильно смягчен тем же Норстэдом в результате вмешательства Кеннеди, который отказался дать связать себе руки.
	
	В начале февраля 1961 года специальная группа под руководством Ачесона получила задание изучить вопрос о политике США в Североатлантическом блоке. И хотя итоги работы этой труппы не были опубликованы, в конце марта в американскую прессу тем не менее проникли сведения о том, что группа рекомендовала отложить рассмотрение вопроса о создании ядерных сил НАТО и обратить особое внимание на укрепление обычных вооруженных сил этого блока.
	
	О сдержанной позиции новой американской администрации в отношении передачи ракетно-ядерного оружия союзникам свидетельствовала и поездка тогдашнего посла по особым поручениям Гарримана в главные страны НАТО и прежде всего в ФРГ, предпринятая весной 1961 года. Гарриман опроверг сообщения печати о наличии «полного согласия» между ним и Штраусом по вопросам ракетно-ядерного вооружения бундесвера и стратегии НАТО. «Господин Штраус, — заявил Гарриман, — имеет, конечно, собственное мнение. Но мне не хотелось бы высказываться об этом». Газета «Штутгартер цайтунг» от 8 марта 1961 года писала по этому поводу: «Очевидно, в беседе между Гарриманом и министром обороны Штраусом единодушие не проявилось». Печать назвала «холодным душем» для Штрауса высказывания Гарримана «а пресс-конференции в Бонне по вопросу ракетно-ядерного вооружения бундесвера.
	
	«Как указывается в Вашингтоне, — писала в апреле 1961 года газета «Нью-Йорк таймc», — постепенно укрепляется мнение, что западная оборонная стратегия должна строиться на том принципе, что Соединенные Штаты — главный, если не единственный, хранитель ядерного оружия. Идея предоставления ядерного оружия союзникам потихоньку оттесняется на задний план».
	
	Чем же можно было объяснить это явно сдержанное отношение Кеннеди к плану «четвертой ядерной державы»?
	
	Дело в том, что в Вашингтоне решили прежде всего тесно «увязать» проблему ядерного вооружения НАТО с общим направлением так называемой «основной стратегии» и разработанным Соединенными Штатами «великим планом» укрепления Атлантического блока. В них, как и прежде, во главу угла ставилось ядерное оружие, при помощи которого правящие круги США намерены были и дальше проводить свою политику «с позиции силы». Однако правительство Кеннеди занялось поисками такой формы ядерных сил НАТО, которая сохранила бы монополию на ядерное оружие в НАТО за США и возможно крепче привязала бы других участников блока к американской военной машине.
	
	Правительство Кеннеди не могло не считаться с тем, что время, прошедшее после запуска первых советских искусственных шутников Земли с еще большей очевидностью подтвердило, что ни о каком военном превосходстве США, не говоря уже о решающем б области ракетно-ядерного оружия, не могло быть и речи. Советский Союз в интересах обеспечения своей безопасности добился новых больших успехов в области ракетостроения. Были создны и успешно испытаны новые виды мощных межконтинентальных баллистических ракет. Все это окончательно лишало всякого смысла прежнюю военную доктрину Эйзенхауэра — Даллеса с ее провокационной идеей «массированного возмездия».
	
	Успехи Советского Союза в области ракетно-ядерного оружия все больше вынуждали правящие круги США считаться с теми грозными последствиями, какие может иметь для них самих конфликт с применением термоядерного оружия. Поэтому «великим планом» укрепления НАТО, разработанным правительством Кеннеди, предусматривалось сохранение целиком за правящими кругами США решения вопроса о применении ядерного оружия. Эта точка зрения официальных американских кругов нашла, в частности, свое отражение в книге профессора Чикагского университета Р. Осгуда «Обязывающий союз», в которой автор потребовал от партнеров США по НАТО целиком «полагаться на американские ядерные силы», то есть безоговорочно принять руководство США в ядерной области, а не «пытаться дополнить или заменить эти силы своими собственными».
	
	Таким путем Соединенные Штаты намерены были фактически сохранить свою монополию на ядерное оружие в НАТО, положить конец поползновениям стран Западной Европы оспаривать эту монополию в капиталистическом мире, усилить военную зависимость своих союзников, чтобы и впредь заставлять их послушно следовать в фарватере американской политики. Эти изменения в американской ядерной политике в НАТО нашли свое отражение уже на сессии Совета блока, состоявшейся 8—10 мая 1961 года в Осло. Хотя государственный секретарь США Раек заявил на сессии о предоставлении НАТО пяти атомных подводных лодок, оснащенных ракетами «Поларис», он подчеркнул, однако, что в соответствии с американским законодательством ракеты «Поларис» останутся под односторонним контролем США (в ведении американского адмирала Деннисона — тогдашнего главнокомандующего вооруженными силами НАТО в Северной Атлантике; экипажи подлодок — исключительно американские). Лондонская «Тайме» писала по поводу этого решения: «Американцы намерены держать стратегическое ядерное оружие за исключением вклада Англии под сугубо американским контролем... Правительство США оставляет в силе свое решение об оснащении американских вооруженных сил, базирующихся в Европе, дополнительным числом американских подводных лодок, снабженных ракетами «Поларис». Но оно берет обратно свое предложение превратить НАТО в «четвертую ядерную державу»... Боеголовки для ядерного тактического оружия, размещенного сейчас в странах Западной Европы, тоже будут по-прежнему оставаться под американским контролем».
	
	Новая стратегическая «схема» США предусматривала, что их западноевропейские союзники по НАТО должны делать больший упор на наращивание обычных, а не ядерных сил. Особенно отчетливо эти требования США были изложены на сессиях Совета НАТО в Афинах и в Париже в мае и декабре 1962 года. Тогдашний министр обороны США Макнамара в ряде своих выступлений выдвинул даже новейшую трактовку прежней концепции «меча и щита», заявив, что «мечом» теперь являются сухопутные силы НАТО, а «щитом» американская ядерная мощь. Подобная трактовка вызвала резкое обострение противоречий между США и их западноевропейскими союзниками по НАТО, усмотревших в ней фактический отказ США от обязательств применить в любых условиях свое термоядерное оружие. Военный обозреватель газеты («Сент-Луис пост Диспетч» Томас Филиппе писал в этой газете 30 декабря 1962 года: «Это было широко истолковано в Европе в том смысле, что новая стратегия рассчитана на то, чтобы дать возможность Соединенным Штатам воздержаться от ядерных ударов... если только они сами не окажутся под угрозой». В этом нашло, по сути дела, свое выражение стремление американских военных кругов перенести центр тяжести на развязывание локальных конфликтов на территории западноевропейских участников НАТО, возникновение которых не ставило бы под угрозу безопасность самих Соединенных Штатов.
	
	После Карибского кризиса осенью 1962 года США стали еще более резко и непримиримо подходить к решению ядерной проблемы в НАТО. По-своему истолковав уроки этого кризиса, министр обороны США Макнамара, генерал Норстэд и ряд других высокопоставленных американских деятелей в своих выступлениях развивали мысль о том, что право нажимать или не нажимать на кнопку ядерной войны должно (принадлежать в НАТО только Соединенным Штатам. Наиболее образно эту точку зрения официальных кругов Вашингтона выразил известный американский обозреватель Уолтер Липпман в своем выступлении в Ассоциации англоамериканской печати в Париже в конце ноября 1962 года. «Управление атомной мощью западного мира, — прямо заявил Липпман, — можно сравнить с управлением автомобилем, который мчится с очень крутыми поворотами. У руля может сидеть только один человек. Те, кто с ним в автомобиле, могут быть полезны лишь при выборе маршрута да отъезда... Означает ли это, что Европа должна предоставить США заботу об атомном оружии Запада? Я думаю, что в ближайшие годы это неизбежно...»
	
	
	План «четвертой ядерной державы» в этих условиях уже не мог устроить правящие круги Соединенных Штатов.
	
	Почему?
	
	Да потому, что этот план, как отмечала газета «Нью-Йорк таймс», «дал бы НАТО собственный запас боеголовок, которые находились бы под контролем этого союза как самостоятельной корпоративной единицы...» А это могло бы привести к использованию ядерного оружия помимо воли и желания США в случае возникновения какого-либо конфликта между Западом и Востоком. Особенно сильно это укрепило бы позиции ФРГ в НАТО, поскольку она получила бы в свои руки готовое ракетно-ядерное оружие. Именно это обстоятельство заставляло Кеннеди в первый период его пребывания у власти с большой осторожностью относиться к планам создания ядерных сил НАТО. Естественно, что подобная позиция вызвала сильное недовольство у западногерманских милитаристов и реваншистов. Но руководящие круги ФРГ были вынуждены на некоторое время смириться и умерить свои ядерные претензии.
	
	Вместе с тем проведение политики, направленной против дальнейшего распространения ядерного оружия и, в частности, против создания ядерных сил НАТО было, конечно, невозможным без осуществления линии на общее смягчение международной напряженности, уменьшение гонки вооружений и прежде всего ядерных, на достижение соглашений с СССР по вопросам разоружения и международной безопасности. Между тем именно вскоре после своего прихода к власти правительство Кеннеди значительно увеличило военные ассигнования, обусловив это увеличение необходимостью преодолеть «ракетный разрыв», то есть отставание от Советского Союза в ракетной области. Обострению международной напряженности способствовала и предпринятая в апреле 1961 года правительством Кеннеди попытка интервенции на Кубу, а также взятый им летом 1961 года курс на значительное усиление военных приготовлений в НАТО в ответ на предложения СССР по урегулированию берлинского вопроса.
	
	В этих условиях в Соединенных Штатах и других странах НАТО заметно активизировались наиболее агрессивные реакционные круги. Правительство Кеннеди критиковалось за то, что, делая упор на важность обычного вооружения и проявляя осторожность в отношении использования ядерного оружия, оно якобы ослабляет военные и политические позиции США.
	
	В таких условиях со стороны европейских членов НАТО, и прежде всего со стороны западногерманских милитаристов и реваншистов, на правительство США был оказан сильнейший нажим с требованием об отказе от курса на укрепление американской ядерной монополии в НАТО, о допуске этих стран в той или иной форме к ядерному оружию. В западногерманской прессе появились заявления Аденауэра и Штрауса о том, что войска Западной Германии должны иметь такое же вооружение, каким располагает «будущий противник». В ноябре 1961 года Аденауэр вновь открыто выдвинул требование о том, чтобы Атлантическому блоку было передано ядерное оружие, которое могло бы быть использовано без предварительного согласия президента США.
	
	На сессии Совета НАТО в Париже в декабре 1961 года «Соединенные Штаты дали согласие возобновить обсуждение вопроса о создании ядерных сил Североатлантического союза. Но в течение почти всего 1962 года Соединенные Штаты Америки явно проводили тактику проволочек, стремясь избежать принятия каких-либо конкретных решений по этому вопросу. При этом представители США в НАТО всячески стремились доказать, что имевшегося американского ядерного оружия вполне достаточно для нужд Запада в случае войны и поэтому с военной точки зрения создание ядерных сил НАТО не являлось необходимым.
	
	Нажим европейских членов НАТО на Соединенные Штаты по вопросам ядерной политики достиг наивысшей точки во второй половине 1962 года. В декабре 1962 года произошло резкое изменение в позиции Соединенных Штатов Америки по вопросу создания ядерных сил Североатлантического блока.
	
	«Новый курс» правительства Кеннеди в политике ядерного вооружения НАТО означал прежде всего конец «особым «отношениям» в этой области, установившимся между США и Англией. Во время встречи Кеннеди и Макмиллана в Нассау на Багамских островах 18—21 декабря 1962 года Соединенные Штаты нанесли серьезный удар притязаниям Англии на «национальное ядерное сдерживающее средство», отказавшись предоставить обещанные Англии ракеты «Скайболт» класса «воздух — земля». (Англичане предполагали их снабдить собственными ядерными зарядами). На приобретении ракет («Скайболт» были основаны все планы создания так называемых ядерных сил Англии; с получением ракет «Скайболт» английское правительство связывало надежды «продлить жизнь» английской стратегической бомбардировочной авиации до 70-х годов. Без этой ракеты (радиусом действия в 1100 миль) английские стратегические бомбардировщики в значительной мере теряли свое значение, ибо они с каждым годом имели все меньше шансов преодолеть современную противовоздушную оборону и донести ядерные заряды до цели. Таким образом, в действительности на переговорах в Нассау, как справедливо отмечала англо-американская пресса, стоял «вопрос о статусе Англии как самостоятельной ядерной державы».
	
	Соединенные Штаты поставили перед собой цель превратить «пакт Нассау» в основу новой атомно-стратегической программы США для НАТО, в широкое соглашение держав НАТО в ядерной области, которое наиболее выгодным для США образом решало бы проблему ядерных сил Запада в целом. В результате совещания Кеннеди и Макмиллана было выработано заявление «о системах ядерной обороны».
	
	Наиболее важные и конкретные положения содержались в 6-м и 8-м пунктах этого заявления. Так, согласно пункту 8-му Соединенные Штаты взамен ракеты «Скайболт» предложили предоставить Англии «на постоянной основе»  ракеты «Поларис» (без боеголовок), предназначенные для дорогостоящих ракетных подводных лодок (которых у Англии еще не было), но непригодные для использования авиацией. Тем самым Англия практически лишалась возможности  совершенствовать в ближайшее время собственные ядерные силы. При этом Соединенные Штаты выдвинули  требование, чтобы английские подводные лодки с «Поларисами», (наряду с частью подводных лодок США, уже состоящих на вооружении) вошли в состав так называемых  «многосторонних ядерных сил» НАТО. Как указывалось в нассаусюом «Заявлении о системах ядерной обороны», целью участников переговоров в отношении обеспечения ракет «Поларис»  должно было <быть создание «многосторонних ядерных сил» НАТО при самых тесных консультациях с другими  союзниками НАТО. В повестку дня был поставлен вопрос об участии в «многосторонних ядерных силах» наряду с Англией также и Франции.
	
	Таким образом, правительство Кеннеди фактически разработало и выдвинуло на совещании в Нассау совершенно новый проект создания объединенных ядерных сил НАТО. «Соглашение в Нассау, — писала английская газета «Дейли мейл» 20 февраля 1963 года, — предусматривало создание ядерных сил НАТО, участниками которых фактически были бы только Америка, Англия и Франция».
	
	Остальные страны НАТО, в том числе и ФРГ, должны  были, по замыслу США, довольствоваться предоставлением объединенным ядерным силам НАТО принадлежавших им ракетных и авиационных подразделений, а также некоторым участием в обсуждении атомных планов блока.
	
	Нетрудно понять подлинную цель выдвижения  Соединенными Штатами нового проекта создания ядерных сил НАТО, (нашедшего свое выражение в «пакте Нассау». Соединенные Штаты, предлагая создать ядерные силы НАТО, основой  которых в ближайшие годы было бы американское ракетно-ядерное оружие, .рассчитывали сохранить таким образом на долгое время свою монополию на ядерное оружие в НАТО, закрепить свое руководящее положение в блоке.
	
	Соединенные Штаты прежде всего рассчитывали, что  осуществление «пакта Нассау» даст им возможность установить контроль над ядерным оружием Англии и Франции.  Во-вторых, США рассчитывали, что в обмен на признание Франции в качестве атомной державы и на допуск к руководству НАТО де Голль откажется от своих антиамериканских  планов в Европе. В-третьих, правящие круги США надеялись путем сговора между тремя атомными державами блока — США, Англией и Францией — заставить остальных членов этого пакта (в первую очередь подразумевалась ФРГ)  отказаться от претензий на атомное оружие.
	
	Таковы были далеко идущие расчеты Соединенных Штатов, связанные с «пактом Нассау».
	
	
		\newpage
		\section[Глава вторая. ПРОЕКТ МЯС И ЕГО ЦЕЛИ]{\center ГЛАВА ВТОРАЯ.\\ \textbf{ПРОЕКТ МЯС И ЕГО ЦЕЛИ}}	
	\subsection[Вашингтон раскрывает свои карты]{\center ВАШИНГТОН РАСКРЫВАЕТ СВОИ КАРТЫ}
	

Те изменения, которые были внесены правительством Кеннеди в политику ядерного вооружения НАТО и которые выразились в отказе от плана «четвертой ядерной державы», вначале были очень настороженно встречены в Бонне. В новом ядерном проекте США, преследовавшем цель в максимальной степени сохранить за Вашингтоном право принятия решения об использовании ядерного оружия в НАТО, Бонн усмотрел определенную помеху тому, чтобы заполучить в ближайшее время в свои руки готовое ракетно-ядерное оружие. Западногерманская печать с иронией писала о «рождественском сюрпризе», преподнесенном Соединенными Штатами своему верному партнеру. Близкая к правительству газета «Генераль-анцейгер» даже обвинила американцев в желании поставить ФРГ в положение союзника «второго класса», а бундесвер превратить в «пехоту, оснащенную обычным оружием». Касаясь соглашения в Нассау, направленного на создание атомного триумвирата в НАТО, западногерманская газета «Ди вельт» прямо писала, что «Бонн ожидает, что проблема будет решена в многостороннем плане, а не будет ограничена рамками трех великих держав». Тогдашний руководитель федерального ведомства печати фон Хазе, касаясь соглашений в Нассау, заявил в начале января 1963 года журналистам в Бонне, что, как надеется федеральное правительство, эти соглашения положат начало созданию «многосторонних ядерных сил» НАТО. «Создание атомной державы НАТО в составе лишь США, Англии и, возможно, Франции, — многозначительно заявила «Рейнише пост», — может привести к расколу Атлантического союза». Западногерманская печать прямо заявляла, что если ФРГ не будет включена в состав «атомного клуба», то она встанет на путь собственного производства ядерного оружия.

Но решающее значение для претворения в жизнь соглашений в Нассау имела позиция де Голля. Хотя Соединенные Штаты и старались всячески подчеркнуть, что в Нассау де Голлю была сделана существенная уступка и что в случае присоединения Франции к «пакту Нассау» США могут пойти навстречу Франции в вопросе предоставления ей «атомных секретов», — несмотря на все это, французский президент оставался непреклонным. По утверждению журнала «Пари-матч», президент де Голль, характеризуя американские предложения, заявил своим министрам, что «речь идет о ширме», которая по замыслу США должна дать им возможность наложить руку на французскую атомную силу, и что Франция не может «отказаться от своих усилий и бросить свое начинание потому, что Вашингтон решил внести свои предложения», 14 января 1963 года де Голль нанес серьезный удар по планам Соединенных Штатов, заявив иа пресс-конференции, что Франция не намерена присоединяться к соглашению в Нассау и что она выступает также против приема Англии в «Общий рынок».

Таким образом, разработанные первоначально в Нассау планы, состоявшие в том, чтобы создать ядерный триумвират в НАТО в составе США, Англии и Франции, по сути дела, провалились. Не сумев упрочить свои позиции в Западной Европе путем привлечения Франции к атомному сотрудничеству в НАТО, Соединенные Штаты в то же время вызвали недовольство и подозрения своего основного европейского партнера по НАТО — ФРГ. Все это поставило Соединенные Штаты в весьма сложное положение. Необходимы были срочные меры, чтобы найти выход из создавшейся ситуации.

Первые месяцы 1963 года ознаменовались целой серией переговоров о форсировании ядерных вооружений НАТО. 4—5 января генеральный секретарь НАТО Д. Стиккер вел по этому вопросу переговоры в Бонне. Этот же вопрос обсуждался 7—8 января во время поездки министра иностранных дел ФРГ Г. Шредера в Лондон и на открывшемся 11 января в Париже специальном заседании Постоянного совета НАТО. Вслед за этим последовала новая серия дополнительных переговоров: поездка заместителя государственного секретаря США Д. Болла в Бонн, Д. Стиккера в Рим, Лондон и другие европейские столицы; поездка заместителя министра обороны США Гилпатрика в Италию и ФРГ.

Важнейшее значение в серии этих переговоров и встреч имело выступление заместителя государственного секретаря США Д. Болла на заседании Постоянного совета НАТО. На этом заседании Д. Болл изложил перед союзниками план создания так называемых «многосторонних ядерных сил» НАТО. Он заявил, что Соединенные Штаты согласны продать Североатлантическому союзу атомные подводные лодки, вооруженные ракетами «Поларис» с ядерными боеголовками. Команды этих лодок должны были формироваться из представителей стран — участниц НАТО, пожелавших сотрудничать в «многосторонних ядерных силах». Стоимость подводного ядерного флота также должна была распределяться между ними.


Таким образом, Соединенные Штаты уже в середине января 1963 года резко изменили курс и фактически выдвинули перед союзниками совершенно новый проект ядерных сил НАТО. Этот проект был опаснее соглашений в Нассау, так как предусматривал значительно более широкое  распространение ядерного оружия и допуск к нему западногерманских милитаристских кругов. Выдвинув это новое предложение, правительство Кеннеди, по существу, вернулось к  первоначальным атомным планам, разработанным еще при  правительстве Эйзенхауэра.

Не удивительно поэтому, что новое американское предложение о создании «многосторонних ядерных сил» НАТО было воспринято в Бонне с нескрываемым удовлетворением. 14 января 1963 года после переговоров Болла с Аденауэром правительство ФРГ впервые после совещания в Нассау  официально заявило о своем полном одобрении атомных планов США. Представитель ФРГ в НАТО профессор Греве получил от канцлера Аденауэра соответствующее указание.  Западногерманская печать с нескрываемым удовольствием отмечала все это как «важную веху» на пути к оснащению  западногерманской армии атомным оружием. «Создается  впечатление, — писала 15 января 1963 года близкая к командной  верхушке бундесвера газета «Франкфуртер альгемайне», — что Вашингтон отдает дань политическому, экономическому и  военному авторитету Федеративной Республики».

И действительно, изложенный Соединенными Штатами в середине января 1963 года проект «многосторонних ядерных сил» НАТО на базе атомных подводных лодок, вооруженных ракетами «Поларис» с ядерными боеголовками, вел к  допуску западногерманских милитаристов к ядерному оружию, к фактическому обладанию этим оружием и к контролю над его использованием.

На основании чего можно было сделать подобный вывод?

Предоставим слово по этому вопросу самой американской прессе. Как сообщалось в корреспонденции из Вашингтона, распространенной уже в конце января 1963 года, пресс-бюро газеты «Нью-Йорк тайме», сославшейся на «осведомленных лиц» в Вашингтоне «дальнейшие фазы нынешней  предварительной «концепции» правительства (Соединенных Штатов. — \textit{Б. X.}) насчет «Поларисов» приведут к постепенному отказу от чисто «американских команд и чисто американского  контроля над этими подводными лодками». Далее в сообщении подробно излагались возможные «стадии» приобщения  союзников США к американскому ракетно-ядерному оружию. «На первых порах, — указывалось в сообщении, — ...команды подводных лодок будут укомплектованы только из американцев... Вторая фаза предполагает подготовку смешанных  команд. На это, как полагают, потребуется около полутора лет для каждой команды. Третья фаза предполагает передачу лодок смешанным командам, в которых американские  офицеры все еще держали бы в своих руках минимум один из двух «ключей», необходимых чтобы запустить ракету с  ядерной боеголовкой. Последняя фаза... предполагает полную передачу подводных лодок иод многонациональный контроль НАТО. Соединенные Штаты в этом случае уже не были бы обладателями ключей или же права вето над применением атомных боеголовок. Этот последний шаг потребует изменения закона Мак-Магона, ограничивающего предоставление другим странам сведений об атомном оружии и  запрещающего предоставлять кому бы то ни было контроль над ним».

Таким образом, уже первоначальный вариант создания «многосторонних ядерных сил» НАТО фактически открывал новые возможности для того, чтобы доступ к  ракетно-ядерному оружию и к спусковому механизму от этого оружия  получили многие государства — участники НАТО и прежде всего западногерманские милитаристские круги. Дальнейшее  развитие и конкретизация планов создания «многосторонних ядерных сил» НАТО лишь подтвердили это.

	\subsection[Бонн вносит коррективы]{\center БОНН ВНОСИТ КОРРЕКТИВЫ}


Хотя замысел Вашингтона, по которому основу  «многосторонних ядерных сил» НАТО должны были составить атомные подводные лодки, вел к тому, что доступ к  стратегическому ядерному оружию в НАТО в конечном итоге получили бы многие страны — участницы блока и в первую очередь милитаристские круги Бонна, последние усмотрели даже в этом плане недостаточно короткий путь к тому, чтобы  заполучить в свои руки готовое ракетно-ядерное оружие. После встречи в Нассау в кругах НАТО пошли явно  инспирированные Бонном разговоры о том, что поскольку «многосторонние силы» должны состоять из атомных подводных лодок,  строительство которых очень дорогое и требующее длительного времени предприятие, то весь проект «многосторонних  ядерных сил» относится-де к далекому будущему.

Стремясь переложить на западноевропейские страны бремя военных расходов, связанных с созданием  «многосторонних ядерных сил» НАТО, и одновременно удовлетворить возраставшие притязания Бонна, стремившегося скорее  получить доступ к ядерному оружию, Соединенные Штаты  выдвинули в середине февраля 1963 года модернизированный  вариант создания «многосторонних ядерных сил» НАТО,  предусматривавший установку ракет «Поларис» на обычных  надводных судах (двадцать пять единиц с восемью ракетами каждая, то есть всего 200 ракет). 21 февраля 1963 года  государственный департамент США опубликовал заявление, в  котором предполагалось «по крайней мере первоначально» ввести в состав «многосторонних ядерных сил» НАТО не атомные подводные лодки, а надводные корабли с ракетами «Поларис».

Сторонники такого решения проблемы ядерных сил НАТО исходили из того, что преимуществом надводного ядерного флота была бы его «неуязвимость». Близкая к Белому дому газета «Вашингтон пост» писала по поводу нового  предложения США: «Предлагаемый американцами флот из 25 специально сконструированных быстроходных торговых  судов, каждое из которых будет вооружено 8 ракетами  «Поларис», имеет гораздо большие шансы на выживание, чем это кажется на первый взгляд. Эти суда оперировали «бы  предпочтительно в ограниченных прибрежных водах стран НАТО, где их было бы трудно отличить от 3000 других торговых  судов, постоянно плавающих в водах Атлантики и  Средиземного моря». Руководители Пентагона, выдвинув этот вариант .создания ядерных сил НАТО на базе кораблей-хамелеонов с ракетно-ядерным оружием на борту рассчитывали и на то, что эти суда в случае войны должны были оттягивать на себя часть ракетно-ядерных средств, которые вынуждены были бы применить против агрессора государства, подвергнувшиеся нападению, что ослабило бы таким образом удары этого оружия по территории самих Соединенных Штатов. Но  помимо этих стратегических, экономических, а также технических причин (в частности, решительного возражения ВМС США против предоставления союзникам секретов, касающихся ядерных подводных лодок), изменить свои планы вскоре  после переговоров в Нассау Вашингтон побудили и  политические мотивы, и в первую очередь усилившееся давление  западногерманских милитаристских кругов. Министр обороны ФРГ фон Хассель прямо заявил на пресс-конференции, состоявшейся в Бойне 7 марта 1963 года, что план оснащения надводных кораблей ракетами «Поларис» возник именно по его инициативе.

Какие же преимущества должно было дать Бонну создание «многосторонних ядерных сил» НАТО на базе надводных кораблей?

Прежде всего в Бонне решили, что надводный ядерный флот НАТО, который должен был состоять из 25 специально сконструированных быстроходных судов, каждое из которых должно было быть вооружено 8 ракетами «Поларис», можно создать в 2 раза быстрее, и он обойдется значительно  дешевле, чем флот атомных подводных ракетоносцев. Надводные корабли ядерного флота НАТО, как специально подчеркнул на упоминавшейся выше пресс-конференции фон Хассель, могли строиться и на западногерманских верфях. Выдвинув новый вариант создания «многосторонних ядерных сил» на базе надводных кораблей, США рассчитывали переложить значительную часть бремени по созданию этих сил на своих западноевропейских союзников; это обстоятельство также учитывалось Бонном в качестве важного фактора,  призванного обеспечить ему командные позиции в «многосторонних ядерных силах» НАТО.

	\subsection[Первый шаг на пути создания ядерных сил НАТО]{\center ПЕРВЫЙ ШАГ НА ПУТИ СОЗДАНИЯ ЯДЕРНЫХ СИЛ НАТО}


В феврале — марте 1963 года для детального обсуждения предложения о создании «многосторонних ядерных сил» НАТО предпринял турне по западноевропейским столицам специальный представитель президента США Л. Мэрчент. Его поездка доказала, что, кроме Западной Германии, которая безоговорочно поддержала американский план создания многосторонних ядерных сил» НАТО, этот план встретил довольно холодный прием со стороны других  западноевропейских участников блока, в частности, со стороны Англии. «Идея создания в рамках НАТО многостороннего ядерного флота со смешанными экипажами кораблей не вызывает никакого энтузиазма у правительства Макмиллана», — писала газета «Нью-Йорк таймс». Не случайно в заявлении англичан по  поводу американского проекта «многосторонних ядерных сил» ни словом не упоминалось о финансовом вкладе Англии на  содержание судов или на оснащение их ракетами. Дело в том, что англичане стали энергично добиваться создания  объединенных ядерных сил НАТО другого типа, которые также были предусмотрены соглашением, подписанным в Нассау.  (Лондон в своих аргументах ссылался, в частности, на 6-й  параграф соглашения в Нассау).

	

	
	
	\newpage
	\tableofcontents
	
	\thispagestyle{empty} % 
	
	\newpage
	
	\setcounter{secnumdepth}{0}  
	
	\phantomsection
	
		\section*{Описание}
	
	{\bf Название:} НАТО И АТОМ (Ядерная политика Североатлантического блока) 
	
{\bf Автор:} Борис Михайлович Халоша
	
{\bf Издательство:} Москва: «Знание» 1975
	
		{\bf Редактор:} \textit{К. М. Чушкова}
	
		{\bf Художественный редактор:} \textit{В. И. Пантелеев}
	
		{\bf Технический редактор:} \textit{М. Т. Столярова}
	
		{\bf Корректор:} \textit{О. Ю. Мигун}
	
		{\bf Аннотация:} В книге рассматриваются различные аспекты политики ядерного вооружения Североатлантического блока (НАТО). В ней показывается, как менялась эта политика в связи с новым соотношением сил на мировой арене, рассматриваются попытки руководства блока создать объединенные ядерные силы НАТО, выявившиеся противоречия вокруг этого проекта и его провал. Значительное место в книге уделяется анализу деятельности нынешних органов ядерного планирования НАТО, раскрывается их опасный для дела мира и безопасности народов характер. Автор освещает широкий круг проблем, связанных с борьбой Советского Союза, всех миролюбивых сил за запрещение ядерного оружия, предотвращение его распространения, создание безъядерных зон в различных районах мира. 
		\thispagestyle{empty} % выключаем отображение номера для этой страницы

	
\end{document}


