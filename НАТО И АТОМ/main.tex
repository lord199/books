
%\documentclass[oneside,final,14pt]{extreport}
\documentclass[12pt, a4paper, openany]{book}
\usepackage[left=2.5cm,right=2.5cm,top=2cm,bottom=2cm,bindingoffset=0cm]{geometry}
%\usepackage[koi8-r]{inputenc}
\usepackage[russianb]{babel}
\usepackage{vmargin}
\setpapersize{A4}
\usepackage[T2A]{fontenc}
\usepackage[utf8x]{inputenc}  % more recent versions (at least>=2004-17-10)
%\usepackage[russian]{babel}
%\setmarginsrb{2cm}{1.5cm}{1cm}{1.5cm}{0pt}{0mm}{0pt}{13mm}
\usepackage{indentfirst}
\usepackage{nicefrac} % For comparison
%\usepackage{xfrac}    % Works better with other fonts
%\usepackage[unicode, pdftex]{hyperref}
\usepackage{lettrine}
\usepackage[usenames]{color}
\special{papersize=a5}
\usepackage{colortbl}
\usepackage[pagestyles]{titlesec}
\usepackage{xfrac}    % Works better with other fonts
\usepackage[colorlinks=true,linkcolor=black,urlcolor=black,bookmarksopen=true]{hyperref}


% Настройка вертикальных и горизонтальных отступов
\titlespacing{\chapter}{0pt}{5pt}{5pt}
\titlespacing{\section}{\parindent}{4mm}{4mm}
\titlespacing{\subsection}{\parindent}{3mm}{3mm}

\newcommand{\anonsection}[1]{ \section*{#1} \addcontentsline{toc}{section}{\numberline {}#1}} 

\makeatletter %%%%% <---- Starting chapter without a pagebreak
\renewcommand\chapter{\par%
	\thispagestyle{plain}% \global\@topnum\z@
	\@afterindentfalse \secdef\@chapter\@schapter}
\makeatother %%%%% <---- Starting chapter without a pagebreak
\titleformat{\chapter}[display]
{\normalfont\bfseries}{}{0pt}{\Large}

\newpagestyle{mystyle}{
	\sethead[\thepage][][]{}{}{\thepage}	
}

\pagestyle{mystyle}

\sloppy
\begin{document}

	
	\begin{titlepage}
		
		\begin{center}
			%\vfill
			
			%\vfill
			\topskip0pt
			\vspace*{\fill}
			
			
			{\large\bf Б. М. ХАЛОША\\}
			\ \\
			\ \\
			{\Huge\bf НАТО И АТОМ\\}
			\ \\
			\ \\
			\ \\
			(Ядерная политика Североатлантического блока) 
			\vspace*{\fill}    
			
			\vfill
			
				\begin{flushright}
			
			Издательство
		
			«Знание»
		
			Москва 1975
			
		\end{flushright}
		\end{center}
		
	\end{titlepage}
	
	\thispagestyle{empty} % выключаем отображение номера для этой страницы
	
	\newpage
	

\setcounter{secnumdepth}{0}  
	
	\section[Введение]{\center ВВЕДЕНИЕ}
	
Начало 70-х годов нынешнего столетия войдет в историю международных отношений как время глубоких  политических перемен на мировой арене.

Благодаря настойчивой борьбе Советского Союза, других социалистических государств, всех миролюбивых сил, а  также реализму, «проявленному в политике ряда   капиталистических стран, достигнуты серьезные сдвиги в сторону разрядки напряженности на международной арене. Совершается исторический поворот от «холодной войны» и опасной напряженности к совместным усилиям по укреплению мира и развитию взаимовыгодного сотрудничества.

Вся история советской внешней политики от Декрета о мире, принятого в 1917 году, и до Программы мира, выдвинутой XXIV съездом КПСС, и последующего успешного осуществления этой Программы свидетельствует о настойчивой и целеустремленной борьбе Советского Союза за мир, за широкое внедрение в практику международных отношений ленинских принципов мирного сосуществования.

В то же время ход мировых событий убедительно показывает, что на пути движения человечества к прочному миру и сотрудничеству между народами существуют еще немалые препятствия. Силы империализма, реакции и войны не   сложили оружие. Они упорно противятся процессу оздоровления международной обстановки, пытаются возродить времена «холодной войны» и опасной напряженности в мире. «Мы реалисты, — говорил Л. И. Брежнев в докладе «О   пятидесятилетии Союза Советских Социалистических республик», — и хорошо видим, что влиятельные круги в мире империализма все еще не отказались от попыток проводить политику «с позиции силы». Все еще продолжается развязанная ими гонка вооружений, создающая угрозу миру. Мы и наши союзники, естественно, не можем не делать из этого необходимых выводов. Но миролюбивый курс нашей внешней политики   неизменен, и в современной обстановке возможности миролюбивых сил в их борьбе против сил агрессии и войны велики как никогда»{\footnote{Л. И. Брежнев. О пятидесятилетии Союза Советских Социалистических Республик. М., Политиздат, 1972, стр. 42.}}.

Как известно, одним из главных орудий подрывной деятельности агрессивных кругов империализма против мира и безопасности народов был и остается Североатлантический блок (НАТО){\footnote{НАТО — сокращенное название, происходящее от начальных букв официального английского наименования этого военного блока —  «Организация Североатлантического договора». (North Atlantic Treaty Organization—NATO).}}.

Напомним, что договор о создании этой замкнутой военно-политической группировки западных держав был подписан 4 апреля 1949 года в Вашингтоне. Первоначально в НАТО вошли Соединенные Штаты Америки, Великобритания,   Франция, Бельгия, Голландия, Люксембург, Канада, Италия, Португалия, Норвегия, Дания и Исландия. В феврале 1952 года в Североатлантический союз были включены Турция и Греция, а в мае 1955 года — Федеративная Республика Германии.

НАТО занимает центральное место среди империалистических военных союзов, созданных для борьбы с мировой социалистической системой, национально-освободительным, рабочим и демократическим движением.

За двадцать пять лет существования НАТО наиболее агрессивные милитаристские круги этого блока использовали и используют его для того, чтобы превратить Западную Европу в вооруженный лагерь и военный плацдарм. Через систему НАТО они добивались и добиваются усиления гонки  вооружений, увеличения вооруженных сил в западноевропейских странах, раздувания ими военных бюджетов, милитаризации их экономики.

И сейчас влиятельные круги в странах — членах НАТО всячески стремятся помешать положительным изменениям, происходящим на международной арене, навязать подход к решению важнейших политических проблем современности с обанкротившейся «позиции силы». Процесс разрядки, все более развивающийся в Европе, тревожит представителей военно-промышленного комплекса, милитаристские круги НАТО, которые усматривают в нем угрозу самому   существованию этой военной группировки. Вооруженная интервенция НАТО против независимой суверенной Республики Кипр, предпринятая летом 1974 г., — новое свидетельство   агрессивного характера этой военной группировки.

	
	\newpage
	\tableofcontents
	
	\thispagestyle{empty} % 
	
	\newpage
	
	\setcounter{secnumdepth}{0}  
	
	\phantomsection
	
		\section*{Описание}
	
	{\bf Название:} НАТО И АТОМ  (Ядерная политика Североатлантического блока) 
	
{\bf Автор:} Борис Михайлович Халоша
	
{\bf Издательство:} Москва: «Знание» 1975
	
		{\bf Редактор:} \textit{К. М. Чушкова}
	
		{\bf Художественный редактор:} \textit{В. И. Пантелеев}
	
		{\bf Технический редактор:} \textit{М. Т. Столярова}
	
		{\bf Корректор:} \textit{О. Ю. Мигун}
	
		{\bf Аннотация:} В книге рассматриваются различные аспекты политики ядерного вооружения Североатлантического блока (НАТО). В ней показывается, как менялась эта политика в связи с новым  соотношением сил на мировой арене, рассматриваются попытки руководства блока создать объединенные ядерные силы НАТО, выявившиеся противоречия вокруг этого проекта и его провал. Значительное место в книге уделяется анализу деятельности нынешних органов ядерного планирования НАТО, раскрывается их опасный для дела мира и безопасности народов характер. Автор освещает широкий круг проблем, связанных с борьбой Советского Союза, всех миролюбивых сил за запрещение ядерного оружия, предотвращение его распространения, создание безъядерных зон в различных районах мира. 
		\thispagestyle{empty} % выключаем отображение номера для этой страницы

	
\end{document}


